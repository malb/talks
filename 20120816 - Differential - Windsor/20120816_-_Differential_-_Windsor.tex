\documentclass[11pt]{beamer}
\usepackage[utf8x]{inputenc}
\usepackage{xspace}
\usepackage{tikz}


\mode<presentation>
{
  \setbeamercovered{transparent}
  \setbeamercolor{normal text}{fg=white,bg=gray}
  \setbeamercolor{alerted text}{fg=white}
  \setbeamercolor{example text}{fg=white}
  \setbeamercolor{background canvas}{bg=darkgray} 
  \setbeamercolor{structure}{fg=white}

  \setbeamercolor{block title}{bg=orange,fg=white}
  \setbeamercolor{block body}{bg=white,fg=darkgray}

  \setbeamercolor{palette primary}{use=structure,fg=structure.fg}

  \setbeamercolor{math text}{}
  \setbeamercolor{math text inlined}{parent=math text}
  \setbeamercolor{math text displayed}{parent=math text}

  \setbeamercolor{normal text in math text}{}

  \setbeamercolor{local structure}{parent=structure}

  \setbeamercolor{titlelike}{parent=structure}

  \setbeamercolor{title}{parent=titlelike}
  \setbeamercolor{title in head/foot}{parent=palette quaternary}
  \setbeamercolor{title in sidebar}{parent=palette sidebar quaternary}

  \setbeamercolor{subtitle}{parent=title}
}

\title{An All-In-One Approach to Differential Cryptanalysis for Small Block Ciphers}
\author{Martin R.\ Albrecht~\inst{1} \and Gregor Leander~\inst{2}}
\institute{1 -- POLSYS Team, UPMC, France \and 2 -- DTU, Denmark}
\date{SAC 2012, Windsor, Canada}

\newcommand{\bemph}[1]{\textbf{#1}\xspace}

\newcommand{\SBox}[2]{\draw (#1-0.4,#2) -- (#1+0.4,#2) -- (#1+0.4,#2-0.75) -- (#1-0.4,#2-0.75) -- cycle; \node at (#1,#2-0.4) {$S$} }
\newcommand{\KeyAdd}[2]{
\draw (#1, #2) -- (#1+0.5,#2); \node at (#1+0.25,#2) {\tiny $\oplus$};
\draw (#1,#2-0.25) -- (#1+0.5,#2-0.25);  \node at (#1+0.25,#2-0.25) {\tiny $\oplus$};
\draw (#1,#2-0.50) -- (#1+0.5,#2-0.50);   \node at (#1+0.25,#2-0.50) {\tiny $\oplus$};
\draw (#1,#2-0.75) -- (#1+0.5,#2-0.75);  \node at (#1+0.25,#2-0.75) {\tiny $\oplus$};
 }

\newcommand{\SBoxes}[1]{\SBox{#1}{0}; \SBox{#1}{1}; \SBox{#1}{2}; \SBox{#1}{3};}
\newcommand{\KeyAdds}[1]{\KeyAdd{#1}{0}; \KeyAdd{#1}{1}; \KeyAdd{#1}{2}; \KeyAdd{#1}{3};}

\newcommand{\Permutation}[1]{
\draw (#1,3.00) -- (#1+0.25,3.00) -- (#1+1,3.00) -- (#1+1.25,3.00);
\draw (#1,2.75) -- (#1+0.25,2.75) -- (#1+1,2.00) -- (#1+1.25,2.00);
\draw (#1,2.50) -- (#1+0.25,2.50) -- (#1+1,1.00) -- (#1+1.25,1.00);
\draw (#1,2.25) -- (#1+0.25,2.25) -- (#1+1,0.00) -- (#1+1.25,0.00);

\draw (#1,2.00) -- (#1+0.25,2.00) -- (#1+1,2.75) -- (#1+1.25,2.75);
\draw (#1,1.75) -- (#1+0.25,1.75) -- (#1+1,1.75) -- (#1+1.25,1.75);
\draw (#1,1.50) -- (#1+0.25,1.50) -- (#1+1,0.75) -- (#1+1.25,0.75);
\draw (#1,1.25) -- (#1+0.25,1.25) -- (#1+1,-0.25) -- (#1+1.25,-0.25);

\draw (#1,1.00) -- (#1+0.25,1.00) -- (#1+1,2.50) -- (#1+1.25,2.50);
\draw (#1,0.75) -- (#1+0.25,0.75) -- (#1+1,1.50) -- (#1+1.25,1.50);
\draw (#1,0.50) -- (#1+0.25,0.50) -- (#1+1,0.50) -- (#1+1.25,0.50);
\draw (#1,0.25) -- (#1+0.25,0.25) -- (#1+1,-0.50) -- (#1+1.25,-0.50);

\draw (#1,0.00)  -- (#1+0.25,0.00)   -- (#1+1,2.25) -- (#1+1.25,2.25);
\draw (#1,-0.25) -- (#1+0.25,-0.25) -- (#1+1,1.25) -- (#1+1.25,1.25);
\draw (#1,-0.50) -- (#1+0.25,-0.50) -- (#1+1,0.25) -- (#1+1.25,0.25);
\draw (#1,-0.75) -- (#1+0.25,-0.75) -- (#1+1,-0.75) -- (#1+1.25,-0.75);

}

\newcommand{\cbox}[3]{\draw[fill=#3] (#1,#2) -- (#1,0) -- (#1+1,0) -- (#1+1,#2) -- cycle;}


\DeclareMathOperator{\D}{\mathcal{D}}
\DeclareMathOperator{\R}{\mathcal{R}}
\DeclareMathOperator{\W}{\mathcal{W}}
\DeclareMathOperator{\Var}{Var}
\DeclareMathOperator{\Cov}{Cov}
\DeclareMathOperator{\Prob}{\mathbf{Pr}}
\DeclareMathOperator{\N}{\mathcal{N}}
\DeclareMathOperator{\Bern}{\mathcal{B}}
\DeclareMathOperator{\average}{av}
\DeclareMathOperator{\e}{\mathbf{e}}
\DeclareMathOperator{\erf}{erf}
\DeclareMathOperator{\Dist}{Dist}
\DeclareMathOperator{\Multi}{Multi}

\AtBeginSection[] {
   \begin{frame}{Outline}
  \tableofcontents[currentsection]
  \end{frame}
}

\begin{document}

\begin{frame}
\titlepage
\end{frame}

\section{Motivation}

\begin{frame}
\frametitle{Lightweight Blockciphers}

\begin{itemize}
 \item Lightweight cryptography calls for block ciphers for very resource constrained devices.
 \item Designing secure ciphers for tiny devices requires \bemph{new design strategies} and perhaps \bemph{compromises} in the security level.
 \item One such constraint is the block size, typically
  \begin{itemize}
   \item $\leq 64$ for lightweight ciphers, and
   \item $\geq 128$ for standard ciphers.
  \end{itemize}
\end{itemize}

\end{frame}

\begin{frame}
\frametitle{Small Block Sizes} 

\begin{description}
 \item[{\bf\color{pink} attacker:}] distinguish the output faster when in counter mode
 \item[{\bf\color{pink} attacker:}] build the complete code-book
 \item[{\bf\color{pink} attacker:}] time-memory trade-off
 \item[{\bf\color{green} designer:}] statistical attacks more difficult due to less data
 \item[{\bf\color{orange} theory:}]   study attacks using entire code-book 
\end{description}

\begin{block}{}
For differential cryptanalysis, it is feasible to compute the exact expected probabilities for all differentials.\\
\end{block}
Yet, it is not obvious how to provide an optimal unified view even if this data is available.
\end{frame}


% \begin{frame}
% \frametitle{Contribution} 
% 
% \begin{itemize}
%  \item Differential cryptanalysis exposes a \bemph{non-uniform distribution} of \bemph{output differences}. 
%  \item Recovering key information is task of \bemph{distinguishing} between right-key and wrong-key \bemph{distributions}. 
%  \item Attacker does not have access to a \bemph{full description} of these distributions.
%  \item For small block sizes, we can compute the parameters precisely. 
%  \item $\Rightarrow$ distinguish distributions which are known completely.
% \end{itemize}
% 
% \begin{block}{Side Effect}
% Our framework unifies and generalises standard differential attacks, impossible differentials, improbable differentials and truncated differentials into one attack framework.
% \end{block}
% \end{frame}



\section{Differential}

\begin{frame}[fragile,allowframebreaks]
\frametitle{Rolling Example: 3 Rounds of \textsc{SmallPresent-[4]}}

\vspace{0.65em}
\begin{center}
\begin{tikzpicture}

\KeyAdds{-1};
\SBoxes{0};
\Permutation{0.5};

\KeyAdds{2};
\SBoxes{3};
\Permutation{3.5};

\KeyAdds{5};
\SBoxes{6};
\Permutation{6.5};

\KeyAdds{8};
\end{tikzpicture}
\end{center}

\framebreak

\begin{center}
\begin{tikzpicture}
\draw [fill=blue!50] (-1.0,4) -- (2.0,4) -- (2.0,-1) -- (-1.0,-1) -- cycle;
\node at (0.5,3.5) {$R_{k_0}$};

\KeyAdds{-1};
\SBoxes{0};
\Permutation{0.5};

\draw [fill=blue!50] (2.0,4) -- (5.0,4) -- (5.0,-1) -- (2.0,-1) -- cycle;
\node at (3.5,3.5) {$R_{k_1}$};

\KeyAdds{2};
\SBoxes{3};
\Permutation{3.5};

\draw [fill=blue!50] (5.0,4) -- (8.0,4) -- (8.0,-1) -- (5.0,-1) -- cycle;
\node at (6.5,3.5) {$R_{k_2}$};

\KeyAdds{5};
\SBoxes{6};
\Permutation{6.5};

\KeyAdds{8};
\end{tikzpicture}
\end{center}

\framebreak

\begin{center}
\begin{tikzpicture}
\draw [fill=blue!50] (-1.0,4) -- (2.0,4) -- (2.0,-1) -- (-1.0,-1) -- cycle;
\node at (0.5,3.5) {$R_{k_0}$};

\KeyAdds{-1};
\SBoxes{0};
\Permutation{0.5};

\draw [fill=blue!50] (2.0,4) -- (5.0,4) -- (5.0,-1) -- (2.0,-1) -- cycle;
\node at (3.5,3.5) {$R_{k_1}$};

\KeyAdds{2};
\SBoxes{3};
\Permutation{3.5};

\draw [fill=blue!50] (5.0,4) -- (8.0,4) -- (8.0,-1) -- (5.0,-1) -- cycle;
\node at (6.5,3.5) {$R_{k_2}$};

\KeyAdds{5};
\SBoxes{6};
\Permutation{6.5};

\draw [fill=red!50] (8.0,4) -- (8.5,4) -- (8.5,-1) -- (8.0,-1) -- cycle;

\KeyAdds{8};
\end{tikzpicture}
\end{center}

\framebreak

{\Large
$$E_k(x) := k_3 \oplus \left(R_{k_2} \circ R_{k_1} \circ  R_{k_0}(x)\right)$$
}
\end{frame}

\begin{frame}
\frametitle{Differential}

Let $$P_F(\delta,\gamma) := \Pr[F(x \oplus \delta) \oplus F(x) = \gamma)].$$

We are interested in pairs $(\delta,\gamma)$ such that $$P_{E_k}(\delta,\gamma) \textnormal{ is ``big enough''}.$$

\begin{center}
\begin{tikzpicture}[scale=0.8]

\node at (-1.5,1.25) {\Huge $\delta$};

\KeyAdds{-1};
\SBoxes{0};
\Permutation{0.5};

\KeyAdds{2};
\SBoxes{3};
\Permutation{3.5};

\KeyAdds{5};
\SBoxes{6};
\Permutation{6.5};

\KeyAdds{8};

\node at (9,1.25) {\Huge $\gamma$};
\end{tikzpicture}
\end{center}

\end{frame}


\begin{frame}
\frametitle{Markov Assumption} 

\begin{center}
\begin{tikzpicture}[scale=0.8,trail/.style={color=orange,very thick}]

\node at (-1.5,1.25) {\Huge $\delta$};

\node at (0.5,3.5) {$P_R(\delta,\gamma_1)$};

\KeyAdds{-1};
\SBoxes{0};
\Permutation{0.5};

\node at (2.4,1.25) {\Huge $\gamma_1$};

\node at (2.5,3.5) {$\cdot$};

\node at (4.5,3.5) {$P_R(\gamma_1,\gamma_2)$};

\KeyAdds{3};
\SBoxes{4};
\Permutation{4.5};

\node at (6.4,1.25) {\Huge $\gamma_2$};

\node at (6.5,3.5) {$\cdot$};

\node at (8.5,3.5) {$P_R(\gamma_2,\gamma)$};

\KeyAdds{7};
\SBoxes{8};
\Permutation{8.5};

\KeyAdds{10};

\node at (11,1.25) {\Huge $\gamma$};

\draw[trail] (-1,2-0.50) -- (-1+0.5,2-0.50);
\draw[trail] (0.5,2-0.25) -- (1.75,2-0.25);
\draw[trail] (0.5,2-0.75) -- (0.75,2-0.75) -- (1.5,-0.25) -- (1.75,-0.25);

\draw[trail] (3.0,1.75) -- (3.5,  1.75);
\draw[trail] (3.0,-0.25) -- (3.5,-0.25);

\draw[trail] (4.5,1.50) -- (4.75,1.50) -- (5.5,0.75) -- (5.75,0.75);
\draw[trail] (4.5,1.75) -- (4.75,1.75) -- (5.5,1.75) -- (5.75,1.75);

\draw[trail] (4.5,-.50) -- (4.75,-.50) -- (5.5, 0.25) -- (5.75, 0.25);
\draw[trail] (4.5,-.75) -- (4.75,-.75) -- (5.5,-.75) -- (5.75,-.75);

\draw[trail] (7.0,0.75) -- (7.5,  0.75);
\draw[trail] (7.0,1.75) -- (7.5,  1.75);
\draw[trail] (7.0,0.25) -- (7.5,  0.25);
\draw[trail] (7.0,-.75) -- (7.5,  -.75);

\draw[trail] (8.5,2.00) -- (8.75, 2.00) -- (9.50, 2.75) -- (9.75,2.75);
\draw[trail] (8.5,1.75) -- (8.75, 1.75) -- (9.50, 1.75) -- (9.75,1.75);
\draw[trail] (8.5,0.75) -- (8.75, 0.75) -- (9.50, 1.50) -- (9.75,1.50);
\draw[trail] (8.5,-.50) -- (8.75, -.50) -- (9.50, 0.25) -- (9.75,0.25);

\draw[trail] (10.00, 2.75) -- (10.50,2.75);
\draw[trail] (10.00, 1.75) -- (10.50,1.75);
\draw[trail] (10.00, 1.50) -- (10.50,1.50);
\draw[trail] (10.00, 0.25) -- (10.50,0.25);

\end{tikzpicture}
\end{center}

\begin{block}{}
We sum over all such trails $\delta,\gamma_1,\gamma_2,\gamma$ and assume $$P_{E_k}(\delta,\gamma) = \sum_{\gamma_1,\gamma_2} P_R(\delta,\gamma_1) \cdot  P_R(\gamma_1.\gamma_2) \cdot  P_R(\gamma_2,\gamma).$$
\end{block}
\end{frame}

\begin{frame}[allowframebreaks]
\frametitle{From Differences to Distributions}

If the block size is small, we can compute the probabilities for all output differences.

\framebreak

\begin{center}
\begin{tikzpicture}[scale=0.7]

\node at (-1.5,1.25) {\Huge $\delta$};

\KeyAdds{-1};
\SBoxes{0};
\Permutation{0.5};

\node at (3.0,1.25) {\Huge $\gamma_1$};

\KeyAdds{4};
\SBoxes{5};
\Permutation{5.5};

\node at (8.0,1.25) {\Huge $\gamma_2$};

\KeyAdds{9};
\SBoxes{10};
\Permutation{10.5};

\KeyAdds{12};

\node at (13,1.25) {\Huge $\gamma$};
\end{tikzpicture}
\end{center}

\framebreak

\begin{center}
\begin{tikzpicture}[scale=0.7]

\node at (-1.4,1.25) {\Large $\D_{I}$};

\KeyAdds{-1};
\SBoxes{0};
\Permutation{0.5};

\node at (3.0,1.25) {\Large $\D_1$};

\KeyAdds{4};
\SBoxes{5};
\Permutation{5.5};

\node at (8.0,1.25) {\Large $\D_2$};

\KeyAdds{9};
\SBoxes{10};
\Permutation{10.5};

\KeyAdds{12};

\node at (13.5,1.25) {\Large $\D_{O}$};
\end{tikzpicture}
\end{center}

\framebreak
\begin{scriptsize}
\begin{equation*}
\begin{array}{c}\Delta_0\\\Delta_1\\ \dots \\ \\ \delta\\ \\ \dots \\ \Delta_{2^n-2} \\ \Delta_{2^n-1}\end{array}:
\left(\begin{array}{c}0\\0\\ \dots\\ 0\\ 1\\ 0\\ \dots \\ 0 \\ 0\end{array}\right) 
\rightarrow A \rightarrow
\left(\begin{array}{c}0\\1/4\\ \dots\\ 1/8\\ 1/16\\ 0\\ \dots \\ 1/4 \\ 0\end{array}\right)
\rightarrow A \rightarrow
\left(\begin{array}{c}0\\1/16\\ \dots\\ 7/256\\ 3/32\\ 1/4\\ \dots \\ 1/16 \\ 1/4\end{array}\right)
\rightarrow A \rightarrow
\left(\begin{array}{c}0\\3/32\\ \dots\\ 19/1024\\ 3/128\\ 7/64\\ \dots \\ 1/128 \\ 1/16\end{array}\right)
 \end{equation*}
\end{scriptsize}

where $A$ is the transformation matrix composed of a block diagonal matrix with difference distribution matrices on the main diagonal and a matrix representing the linear layer.

\end{frame}


\section{Model}

\begin{frame}[allowframebreaks,fragile]
\frametitle{Multi-dimensional Distribution of $D^{(N)}_{E_K}(\delta,\gamma)$}
We denote by
\[ \D^{(N)}_{E_K}(\delta) = \left(D^{(N)}_{E_K}(\delta,1),D^{(N)}_{E_K}(\delta,2),\dots,D^{(N)}_{E_K}(\delta,2^n-1)\right) \]
the vector of all counters corresponding to all output differences after encrypting $N$ plaintext-pairs with difference $\delta$.

\begin{center}
\begin{tikzpicture}[yscale=0.5]
\cbox{0}{2}{lightgray} \node at (0.5,-1) {\tiny $D^{(N)}_{E_K}(\delta,1)$};
\cbox{1}{3}{lightgray} \node at (1.5,-1) {\tiny $D^{(N)}_{E_K}(\delta,2)$};
\cbox{2}{0}{lightgray} \node at (2.5,-1) {\tiny $D^{(N)}_{E_K}(\delta,3)$};
\cbox{3}{1}{lightgray} \node at (3.5,-1) {\tiny $D^{(N)}_{E_K}(\delta,4)$};
\cbox{4}{5}{lightgray} \node at (4.5,-1) {\tiny $D^{(N)}_{E_K}(\delta,5)$};
\cbox{5}{3}{lightgray} \node at (5.5,-1) {\tiny $D^{(N)}_{E_K}(\delta,6)$};
\cbox{6}{2}{lightgray} \node at (6.5,-1) {\tiny $D^{(N)}_{E_K}(\delta,7)$};
\cbox{7}{3}{lightgray} \node at (7.5,-1) {\tiny $D^{(N)}_{E_K}(\delta,8)$};
\end{tikzpicture}
\end{center}

\framebreak

\begin{block}{Assumption}
$\D^{(N)}_{E_K}(\delta)$ follows a multinomial distribution where each component is binomially distributed and
$$\sum_{\gamma}  D^{(N)}_{E_K}(\delta,\gamma) =N.$$
\end{block}


\end{frame}

\begin{frame}[allowframebreaks,fragile]
 \frametitle{The Attack Algorithm}

We assume that the attacker has computed the parameters of two distributions. Namely, vectors of parameters $p$ and $q$ such that
\begin{eqnarray}
p_i&=& P_{E}(\delta,i) \label{eqn:pi}\\
q_i&=& P_{R^{-1} \circ R \circ E}(\delta,i). \label{eqn:qi}
\end{eqnarray}

\framebreak

\begin{tikzpicture}[scale=0.5]

\node at (-1.5,1.25) {\Huge $\delta$};

\KeyAdds{-1};
\SBoxes{0};
\Permutation{0.5};

\KeyAdds{2};
\SBoxes{3};
\Permutation{3.5};

\KeyAdds{5};
\SBoxes{6};
\Permutation{6.5};

\KeyAdds{8};

\end{tikzpicture}
%\end{center}

\begin{tikzpicture}[scale=0.5]

\node at (-1.5,1.25) {\Huge $\delta$};

\KeyAdds{-1};
\SBoxes{0};
\Permutation{0.5};

\KeyAdds{2};
\SBoxes{3};
\Permutation{3.5};

\KeyAdds{5};

\node at (10,1.25) {\Large $\Dist_1 = \Multi(N,p)$};
\end{tikzpicture}


%\begin{center}
\begin{tikzpicture}[scale=0.5]

\node at (-1.5,1.25) {\Huge $\delta$};

\KeyAdds{-1};
\SBoxes{0};
\Permutation{0.5};

\KeyAdds{2};
\SBoxes{3};
\Permutation{3.5};

\KeyAdds{5};
\SBoxes{6};
\Permutation{6.5};

\KeyAdds{8};

\KeyAdds{8.75};

\Permutation{9.50};
\SBoxes{11.25};

\node at (16.5,1.25) {\Large $\Dist_2 = \Multi(N,q)$};
\end{tikzpicture}
%\end{center}


\framebreak

Given a guess for the last round key, the attacker 

\begin{enumerate}
 \item partially decrypts every ciphertext and for all output differences $\gamma$ computes the number of pairs fulfilling the differential $\delta \rightarrow \gamma$; 
 \item estimates the likelihood that the vector was sampled from $\Dist_1$. 
\end{enumerate}

\vspace{1em}

\begin{block}{}
Step two is equivalent to computing the difference of the log-likelihood of the vector with respect to $\Dist_1$ and with respect to $\Dist_2$.
\end{block}

\framebreak

\begin{center}
\begin{tikzpicture}[yscale=0.5]
\node at (-1,1) {$\D^{(N)}_{E_K}(\delta)$};
\cbox{0}{2}{lightgray}
\cbox{1}{3}{lightgray}
\cbox{2}{0}{lightgray}
\cbox{3}{1}{lightgray}
\cbox{4}{6}{lightgray}
\cbox{5}{3}{lightgray}
\cbox{6}{2}{lightgray}
\cbox{7}{3}{lightgray}
\end{tikzpicture}
\end{center}

\begin{center}
\begin{tikzpicture}[yscale=0.5]
\node at (-1.4,1) {$p$};
\cbox{0}{2}{blue!50}
\cbox{1}{4}{blue!50}
\cbox{2}{0}{blue!50}
\cbox{3}{2}{blue!50}
\cbox{4}{5}{blue!50}
\cbox{5}{3}{blue!50}
\cbox{6}{1}{blue!50}
\cbox{7}{2}{blue!50}

\end{tikzpicture}
\end{center}

\begin{center}
\begin{tikzpicture}[yscale=0.5]
\node at (-1.4,1) {$q$};
\cbox{0}{2}{red!50}
\cbox{1}{3}{red!50}
\cbox{2}{3}{red!50}
\cbox{3}{2}{red!50}
\cbox{4}{3}{red!50}
\cbox{5}{3}{red!50}
\cbox{6}{2}{red!50}
\cbox{7}{3}{red!50}
\end{tikzpicture}
\end{center}


\framebreak

\begin{block}{}
Denoting $w_i=\log \left(\frac{p_i}{q_i}\right)$
the attacker computes
$l_{k'}= \sum_{i} w_i c_i$ where $c_i$ is the counter value for the output difference $i$.
\end{block}

\framebreak

This captures the following variants of differential cryptanalysis

\framebreak

Standard Differential Cryptanalysis:

\begin{center}\begin{tikzpicture}[yscale=0.5]
\node at (-1.4,1) {$p$};
\draw[fill=blue!50] (0,0) rectangle (0+1,0+2);
\draw[fill=blue!50] (1,0) rectangle (1+1,0+4);
\draw[fill=blue!50] (2,0) rectangle (2+1,0+0);
\draw[fill=blue!50] (3,0) rectangle (3+1,0+2);
\draw[fill=blue!50] (4,0) rectangle (4+1,0+5);
\draw[fill=blue!50] (5,0) rectangle (5+1,0+3);
\draw[fill=blue!50] (6,0) rectangle (6+1,0+1);
\draw[fill=blue!50] (7,0) rectangle (7+1,0+2);

\node at (-1.4,-3) {$q$};
\draw[fill=red!50] (0,-4) rectangle (0+1,-4+2);
\draw[fill=red!50] (1,-4) rectangle (1+1,-4+3);
\draw[fill=red!50] (2,-4) rectangle (2+1,-4+3);
\draw[fill=red!50] (3,-4) rectangle (3+1,-4+2);
\draw[fill=red!50] (4,-4) rectangle (4+1,-4+3);
\draw[fill=red!50] (5,-4) rectangle (5+1,-4+3);
\draw[fill=red!50] (6,-4) rectangle (6+1,-4+2);
\draw[fill=red!50] (7,-4) rectangle (7+1,-4+3);

\draw[draw=green,very thick] (4.5,0.5) ellipse (1 and 5.5);
\end{tikzpicture}\end{center}

\framebreak

Truncated Differential Cryptanalysis:

\begin{center}\begin{tikzpicture}[yscale=0.5]
\node at (-1.4,1) {$p$};
\draw[fill=blue!50] (0,0) rectangle (0+1,0+2);
\draw[fill=blue!50] (1,0) rectangle (1+1,0+4);
\draw[fill=blue!50] (2,0) rectangle (2+1,0+0);
\draw[fill=blue!50] (3,0) rectangle (3+1,0+2);
\draw[fill=blue!50] (4,0) rectangle (4+1,0+5);
\draw[fill=blue!50] (5,0) rectangle (5+1,0+3);
\draw[fill=blue!50] (6,0) rectangle (6+1,0+1);
\draw[fill=blue!50] (7,0) rectangle (7+1,0+2);

\node at (-1.4,-3) {$q$};
\draw[fill=red!50] (0,-4) rectangle (0+1,-4+2);
\draw[fill=red!50] (1,-4) rectangle (1+1,-4+3);
\draw[fill=red!50] (2,-4) rectangle (2+1,-4+3);
\draw[fill=red!50] (3,-4) rectangle (3+1,-4+2);
\draw[fill=red!50] (4,-4) rectangle (4+1,-4+3);
\draw[fill=red!50] (5,-4) rectangle (5+1,-4+3);
\draw[fill=red!50] (6,-4) rectangle (6+1,-4+2);
\draw[fill=red!50] (7,-4) rectangle (7+1,-4+3);

\draw[draw=green,very thick] (1.5,0.5) ellipse (1 and 5.5);
\draw[draw=green,very thick] (4.5,0.5) ellipse (1 and 5.5);
\end{tikzpicture}\end{center}

\framebreak

Improbable Differential Cryptanalysis:

\begin{center}\begin{tikzpicture}[yscale=0.5]
\node at (-1.4,1) {$p$};
\draw[fill=blue!50] (0,0) rectangle (0+1,0+2);
\draw[fill=blue!50] (1,0) rectangle (1+1,0+4);
\draw[fill=blue!50] (2,0) rectangle (2+1,0+0);
\draw[fill=blue!50] (3,0) rectangle (3+1,0+2);
\draw[fill=blue!50] (4,0) rectangle (4+1,0+5);
\draw[fill=blue!50] (5,0) rectangle (5+1,0+3);
\draw[fill=blue!50] (6,0) rectangle (6+1,0+1);
\draw[fill=blue!50] (7,0) rectangle (7+1,0+2);

\node at (-1.4,-3) {$q$};
\draw[fill=red!50] (0,-4) rectangle (0+1,-4+2);
\draw[fill=red!50] (1,-4) rectangle (1+1,-4+3);
\draw[fill=red!50] (2,-4) rectangle (2+1,-4+3);
\draw[fill=red!50] (3,-4) rectangle (3+1,-4+2);
\draw[fill=red!50] (4,-4) rectangle (4+1,-4+3);
\draw[fill=red!50] (5,-4) rectangle (5+1,-4+3);
\draw[fill=red!50] (6,-4) rectangle (6+1,-4+2);
\draw[fill=red!50] (7,-4) rectangle (7+1,-4+3);

\draw[draw=green,very thick] (6.5,0.5) ellipse (1 and 5.5);
\end{tikzpicture}\end{center}

\framebreak
Impossible Differential Cryptanalysis:

\begin{center}\begin{tikzpicture}[yscale=0.5]
\node at (-1.4,1) {$p$};
\draw[fill=blue!50] (0,0) rectangle (0+1,0+2);
\draw[fill=blue!50] (1,0) rectangle (1+1,0+4);
\draw[fill=blue!50] (2,0) rectangle (2+1,0+0);
\draw[fill=blue!50] (3,0) rectangle (3+1,0+2);
\draw[fill=blue!50] (4,0) rectangle (4+1,0+5);
\draw[fill=blue!50] (5,0) rectangle (5+1,0+3);
\draw[fill=blue!50] (6,0) rectangle (6+1,0+1);
\draw[fill=blue!50] (7,0) rectangle (7+1,0+2);

\node at (-1.4,-3) {$q$};
\draw[fill=red!50] (0,-4) rectangle (0+1,-4+2);
\draw[fill=red!50] (1,-4) rectangle (1+1,-4+3);
\draw[fill=red!50] (2,-4) rectangle (2+1,-4+3);
\draw[fill=red!50] (3,-4) rectangle (3+1,-4+2);
\draw[fill=red!50] (4,-4) rectangle (4+1,-4+3);
\draw[fill=red!50] (5,-4) rectangle (5+1,-4+3);
\draw[fill=red!50] (6,-4) rectangle (6+1,-4+2);
\draw[fill=red!50] (7,-4) rectangle (7+1,-4+3);

\draw[draw=green,very thick] (2.5,0.5) ellipse (1 and 5.5);
\end{tikzpicture}\end{center}

\end{frame}

\begin{frame}[fragile,allowframebreaks]
\frametitle{Computation of the Gain}

The distribution of $l_{k'}$ can be well approximated by a normal distribution. 

Denote the random variable for the log-likelihood-ratio of the right key $\R$. We have
\[ E(\R)=N \sum w_ip_i  .\]
and
\[ \Var(\R)=N\left( \left(\sum_i w_i^2 p_i \right) -\left(\sum_i w_i p_i \right)^2 \right).\]

\framebreak

Therefore, denoting by $\N\left(E,V\right)$ the normal distribution with expected value $E$ and variance $V$, we will use the following approximation
\[ \R \sim \N\left(N \sum w_ip_i, N\left( \left(\sum_i w_i^2 p_i \right) -\left(\sum_i w_i p_i \right)^2 \right)\right).\]

Similarly, for wrong keys: $\W$.

\framebreak

We assume that the right key value is sampled according to ${\R}$, the wrong keys values are sampled from ${\W}$. Computing the gain for 50\% of the keys is computing the probability that $\W \ge E(\R)$.

\begin{center}
\newcommand{\normdist}[5]{\draw[#3] plot[id=#4,samples=100,domain=#5] function {  1/sqrt(2*pi*#2) * exp(- ((x-#1)**2 / (2*#2)) ) };}
\begin{tikzpicture}[scale=2.7]
    \draw[red,fill] plot[id=f1] file{integral.points};
    \normdist{-0.256}{0.112}{red,smooth,thick}{f2}{-1.5:1.5}
    \draw[white,thick] (0.257,0) -- (0.257,1.18) node[above] {$E(\R)$};
    \normdist{0.257}{0.113}{green,smooth,thick}{f3}{-1.5:1.5}

    \draw[->] (-1.6,0) -- (1.6,0) node[right] {$x$};
    \draw[->] (0,-0.0) -- (0,1.4) node[above] {$\Pr(x)$};
\end{tikzpicture}
\end{center}

\end{frame}

\section{Applications}

\begin{frame}
\frametitle{SmallPresent-[4]} 

\begin{small}
\begin{center}
\begin{tabular}{|l||c|c|c|c|c|}   
\hline
\#rounds &7 & 7 & 8 & 8 & 9 \\
\hline
Data used     & $2^{16}$ & $2^{9}$   & $2^{16}$  & $2^{16}$  & $2^{16}$  \\ \hline
\#$\Delta$    & $1$      & $1$       & $1$       & $5$       & $60$      \\ \hline
$E(\R)$       & $53.82$  & $0.84$    & $2.22$    & $9.76$    & $1.51$    \\ 
$V(\R)$       & $124.08$ & $1.93$    & $4.61$    & $20.15$   & $3.04$    \\ 
$E(\W)$       & $-47.28$ & $-0.73$   & $-2.14$   & $-9.46$   & $-1.51$   \\ 
$V(\W)$       & $84.23$  & $1.31$    & $4.15$    & $18.35$   & $3.02$    \\ \hline
\#r $< 2^{0}$ &      100 &  4 ( 1.1) &  1 ( 2.7) & 57 (61.9) &  1 ( 0.56)\\ 
\#r $< 2^{1}$ &          &  4 ( 1.5) &  2 ( 3.9) & 65 (67.6) &  2 ( 0.84)\\ 
\#r $< 2^{2}$ &          &  4 ( 2.1) &  3 ( 5.4) & 73 (73.1) &  2 ( 1.26)\\ 
\#r $< 2^{3}$ &          &  4 ( 2.9) &  5 ( 7.4) & 79 (78.3) &  2 ( 1.96)\\ 
\#r $< 2^{4}$ &          &  4 ( 4.1) &  8 (10.2) & 82 (83.0) &  2 ( 2.87)\\ 
\#r $< 2^{5}$ &          &  4 ( 5.7) & 11 (13.7) & 84 (87.2) &  5 ( 4.27)\\ 
\#r $< 2^{6}$ &          &  5 ( 7.8) & 16 (18.3) & 88 (90.7) &  8 ( 6.23)\\ 
\#r $< 2^{7}$ &          &  9 (10.7) & 25 (24.2) & 91 (93.6) & 14 ( 8.96)\\ \hline
\end{tabular}         
\end{center}
\end{small}
\end{frame}

\begin{frame}
\frametitle{KATAN-32} 

\begin{small}
\begin{center}
\begin{tabular}{|l|c|c|c|c|}
\hline
\#rounds                    & 71+24 & 88+24 & 90+24 & 91+24\\
\hline
$\delta$                    & ${\tt 0x1006a880}$ & ${\tt 0x1006a880}$ & ${\tt 0x1006a880}$ & ${\tt 0x1006a880}$\\
best $\gamma$               & ${\tt 0x00000008}$ & ${\tt 0x02000004}$ & ${\tt 0x08080015}$ & ${\tt 0x00400000}$\\
$\tilde P_E(\delta,\gamma)$ & $2^{-29.52}$       & $2^{-31.97}$       & $2^{-31.99}$       & $2^{-31.98}$\\
\hline
$E(\R)$                     &  2505.2 &  0.7203 &  0.3750 &  0.3695\\ 
$V(\R)$                     &  5096.6 &  1.4407 &  0.7500 &  0.7390\\ 
$E(\W)$                     & -2467.4 & -0.7203 & -0.3750 & -0.3695\\ 
$V(\W)$                     &  4868.2 &  1.4406 &  0.7500 &  0.7390\\
\hline
exp. gain                   & $>50$ & 3.12 & 2.37 & 2.35\\
\hline
\end{tabular}
\end{center}
\end{small}

\vspace{0.5em}

The expected gain for 50\% of the keys for 92+24 and 94+24 rounds is $1.92$ and $1.23$ respectively.

\end{frame}

\begin{frame}{Thank you for your attention}
\begin{center}
\begin{Huge}

\begin{center}
Questions?
\end{center}
\end{Huge}
\end{center}
\end{frame}


\end{document}