% Created 2019-02-16 Sat 22:29
% Intended LaTeX compiler: pdflatex
\documentclass[presentation,smaller]{beamer}
\RequirePackage{etex}
\RequirePackage[l2tabu,orthodox]{nag}            %% Warn about obsolete commands and packages
\RequirePackage{amsmath,amsfonts,amssymb,amsthm} %% Math
\RequirePackage{ifxetex,ifluatex}                %% Detect XeTeX and LuaTeX
\RequirePackage{fixltx2e}                        %% provides \textsubscript
\RequirePackage{xspace}
\RequirePackage{graphicx}
\RequirePackage{comment}
\RequirePackage{url}
\RequirePackage{relsize}
\RequirePackage{booktabs}
\RequirePackage{tabularx}
\RequirePackage[normalem]{ulem}
\RequirePackage[all]{xy}
\RequirePackage{etoolbox}

%%%
%%% Code Listings
%%%

\RequirePackage{listings}
\lstdefinelanguage{Sage}[]{Python}{morekeywords={True,False,sage,cdef,cpdef,ctypedef,self},sensitive=true}

\lstset{frame=none,
  showtabs=False,
  showspaces=False,
  showstringspaces=False,
  commentstyle={\color{gray}},
  keywordstyle={\color{mLightBrown}\textbf},
  stringstyle ={\color{mDarkBrown}},
  frame=single,
  basicstyle=\tt\scriptsize\relax,
  backgroundcolor=\color{gray!190!black},
  inputencoding=utf8,
  literate={…}{{\ldots}}1,
  belowskip=0.0em,
}

\makeatletter
\patchcmd{\@verbatim}
  {\verbatim@font}
  {\verbatim@font\scriptsize}
  {}{}
\makeatother

%%%
%%% Tikz
%%%

\RequirePackage{tikz,pgfplots}

\usetikzlibrary{calc}
\usetikzlibrary{arrows}
\usetikzlibrary{automata}
\usetikzlibrary{positioning}
\usetikzlibrary{decorations.pathmorphing}
\usetikzlibrary{backgrounds}
\usetikzlibrary{fit,}
\usetikzlibrary{shapes.symbols}
\usetikzlibrary{chains}
\usetikzlibrary{shapes.geometric}
\usetikzlibrary{shapes.arrows}
\usetikzlibrary{graphs}

%% Cache

\ifdefined\tikzcaching  % chktex 1
  \usetikzlibrary{external}
  \tikzexternalize[prefix=build/]
  \tikzset{external/up to date check=diff}  %% MD5 fails from within emacs
\fi

%%%
%%% SVG (Inkscape)
%%%

\ifxetex % chktex 1
\newcommand{\executeiffilenewer}[3]{%
  {\immediate\write18{#3}} % hack
}
\else
\newcommand{\executeiffilenewer}[3]{%
  \ifnum\pdfstrcmp{\pdffilemoddate{#1}}%
    {\pdffilemoddate{#2}}>0%
    {\immediate\write18{#3}}
  \fi%
}
\fi

\newcommand{\includesvg}[2][1.0\textwidth]{%
 \executeiffilenewer{#1.svg}{#1.pdf}%
 {inkscape -z -D --file=#2.svg --export-pdf=#2.pdf --export-latex --export-area-page}%
 \def\svgwidth{#1} 
 \input{#2.pdf_tex}%
} 

%%%
%%% Metropolis Theme
%%%

\usetheme{metropolis}
\metroset{color/block=fill}
\metroset{numbering=none}
\metroset{outer/progressbar=foot}
\metroset{titleformat=smallcaps}

\setbeamercolor{description item}{fg=mLightBrown}
% \setbeamerfont{alerted text}{series=\bfseries}
\setbeamerfont{footnote}{size=\scriptsize}
\setbeamercolor{example text}{fg=mDarkBrown}
\setbeamercolor{block title alerted}{fg=white, bg=mDarkBrown}
\setbeamertemplate{bibliography item}[text]

\renewcommand*{\UrlFont}{\ttfamily\relax}

%%%
%%% UTF-8 & Fonts
%%% 

\RequirePackage{unicodesymbols} % after metropolis which loads fontspec

\setmonofont[BoldFont={Cousine Bold},
             ItalicFont={Cousine Italic},
             BoldItalicFont={Cousine Bold Italic},
             Scale=0.9]{Cousine}             
%%%
%%% BibLaTeX
%%%

\RequirePackage[backend=bibtex,
            style=alphabetic,
            maxnames=4,
            citestyle=alphabetic]{biblatex}

\bibliography{local.bib,abbrev3.bib,crypto_crossref.bib,rfc.bib,jacm.bib}

\DeclareFieldFormat{title}{\alert{#1}}
\DeclareFieldFormat[book]{title}{\alert{#1}}
\DeclareFieldFormat[thesis]{title}{\alert{#1}}
\DeclareFieldFormat[inproceedings]{title}{\alert{#1}}
\DeclareFieldFormat[incollection]{title}{\alert{#1}}
\DeclareFieldFormat[article]{title}{\alert{#1}}
\DeclareFieldFormat[misc]{title}{\alert{#1}}

%%% 
%%% Microtype
%%%

\IfFileExists{upquote.sty}{\RequirePackage{upquote}}{}
\IfFileExists{microtype.sty}{\RequirePackage{microtype}}{}

\setlength{\parindent}{0pt}                   %%
\setlength{\parskip}{6pt plus 2pt minus 1pt}  %%
\setlength{\emergencystretch}{3em}            %% prevent overfull lines
\setcounter{secnumdepth}{0}                   %%

%%% Local Variables:
%%% mode: latex
%%% End:
\usepackage{graphicx}
\usepackage{grffile}
\usepackage{longtable}
\usepackage{wrapfig}
\usepackage{rotating}
\usepackage[normalem]{ulem}
\usepackage{amsmath}
\usepackage{textcomp}
\usepackage{amssymb}
\usepackage{capt-of}
\usepackage{hyperref}
\usepackage{microtype}
\usepackage{newunicodechar}
\usepackage[notions,operators,sets,keys,ff,adversary,primitives,complexity,asymptotics,lambda,landau,advantage]{cryptocode}
\usepackage{xspace}
\usepackage{units}
\usepackage{nicefrac}
\usepackage{gensymb}
\usepackage{amsthm}
\usepackage{amsmath}
\usepackage{amssymb}
\usepackage{xcolor}
\usepackage{listings}
\usepackage[color=yellow!40]{todonotes}
\newcommand{\mat}[1]{\ensuremath{\mathbf{#1}}\xspace}
\renewcommand{\vec}[1]{\ensuremath{\mathbf{#1}}\xspace}
\usepackage[]{algorithm2e}
\usetheme{default}
\author{Martin R. Albrecht}
\date{15/03/2018}
\title{Lattice Reduction Attacks on HE Schemes}
\hypersetup{
pdfauthor={Martin R. Albrecht},
pdftitle={Lattice Reduction Attacks on HE Schemes},
pdfkeywords={},
pdfsubject={},
pdfcreator={Emacs 26.1 (Org mode 9.2.1)},
pdflang={English},
colorlinks,
citecolor=gray,
filecolor=gray,
linkcolor=gray,
urlcolor=gray
}
\begin{document}

\maketitle

\begin{frame}[label={sec:org8591b3a}]{Learning with Errors}
The Learning with Errors (LWE) problem was defined by Oded Regev.\footfullcite{STOC:Regev05}

Given \((\vec{A},\vec{c})\) with uniform \(\vec{A} \in \ZZ_q^{m × n}\), uniform \(\vec{s} \in \ZZ_q^{n}\) and small \(\vec{e} \in \ZZ^{m}\) is \(\vec{c} \sample \mathcal{U}({\ZZ_q^{m}})\) or

\[
\left(\begin{array}{c}
\\
\\
\\ 
\vec{c} \\
\\
\\
\\
\end{array} \right) = \left(
\begin{array}{ccc}
\leftarrow & n & \rightarrow \\
\\
\\ 
& \vec{A} & \\
\\
\\
\\
\end{array} \right) \cdot \left( \begin{array}{c}
\\
\vec{s} \\
\\
\end{array} \right) + \left(
\begin{array}{c}
\\
\\
\\ 
\vec{e} \\
\\
\\
\\
\end{array} 
\right).
\]
\end{frame}


\begin{frame}[label={sec:orgd658f5a}]{Introduction}
\begin{block}{Where it all began …}
\fullcite{EPRINT:ACFP14}
\end{block}

\begin{itemize}
\item We were writing a paper on using Gröbner bases for solving LWE instances.
\item Ludovic Perret asked me to write the related work section.
\item Our paper on using Gröbner bases for solving LWE still has not been published.
\end{itemize}


I am still working on that related work section.
\end{frame}

\begin{frame}[label={sec:org01d8826},fragile]{“Related Work”}
 \alert{Primal Attack} (\texttt{primal\_usvp}, \texttt{primal\_decode})

Solve Bounded Distance Decoding problem (BDD), i.e. \[
\textnormal{ find } \vec{s'} \textnormal{ s.t. } \|\vec{w} - \vec{c}\| \textnormal{is minimised, with } \vec{w} = \vec{A} ⋅ \vec{s'} \textnormal{ using}\]
uSVP embedding or Babai's nearest planes resp. enumeration.

\alert{Dual Attack} (\texttt{dual}, \texttt{dual\_scale})

Solve Short Integer Solutions problem (SIS) in the left kernel of \(\vec{A}\), i.e. \[\textnormal{ find a short } \vec{w} \textnormal{ such that } \vec{w} ⋅ \vec{A} = 0\]
and check if \(\Angle{\vec{w},\vec{c}} = \vec{w} ⋅ \left(\vec{A} ⋅ \vec{s} + \vec{e}\right) = \Angle{\vec{w},\vec{e}}\) is short.
\end{frame}

\begin{frame}[label={sec:orgb0f8d4b}]{Bounded Distance Decoding and unique SVP}
Given \(\vec{A}, \vec{c}\) with \(\vec{c} = \vec{A} ⋅ \vec{s} + \vec{e}\), we know that for some \(\vec{s}'\) we have that \(\vec{A}⋅\vec{s}' - \vec{c} \pmod q\) is rather small.

\(\Rightarrow\) we know there is an unusually short vector in the \(q\)-ary lattice \[\vec{B}=\left(\begin{array}{cc}
          \vec{A}^T &  0 \\
          \vec{c}^T   & t \\
        \end{array} \right) \in \ZZ_q^{(n+1) \times (m+1)}\] since \[(\vec{s} \mid -1) ⋅ \vec{B} = (\vec{e} \mid -t) \bmod q\]
and use lattice reduction to find it.
\end{frame}

\begin{frame}[label={sec:org3793661}]{Success Condition (ADPS16)}
\begin{tikzpicture}
\begin{axis}[/pgf/number format/.cd,fixed, grid=both,ymin = 1,legend pos=north east, xlabel=index $i$ ,ylabel=$\log_2(\norm \cdot)$,width=\columnwidth, height=0.6\columnwidth, xmin = 1, xmax = 183,legend cell align=left,]
%      \draw[->] (-3,0) -- (4.2,0) node[right] {$x$};
%      \draw[->] (0,-3) -- (0,4.2) node[above] {$y$};
\addplot[domain=1:183,smooth,variable=\x,black] plot ({\x},{log2(1.01170246711949^(-2*(\x-1)+183)*54.5751087741536)});
\addlegendentry{GSA for $\norm{\vec b_i^*}$}

\addplot[gray,thick,x filter/.code={\pgfmathparse{\pgfmathresult+1.0}}] coordinates {
   (  0,  8.78) (  1,  8.78) (  2,  8.77) (  3,  8.72) (  4,  8.71) (  5,  8.69) (  6,  8.66) (  7,  8.63) (  8,  8.62) (  9,  8.59) ( 10,  8.54) ( 11,  8.53) ( 12,  8.51) ( 13,  8.47) ( 14,  8.43) ( 15,  8.39) ( 16,  8.36) ( 17,  8.34) ( 18,  8.30) ( 19,  8.28) ( 20,  8.24) ( 21,  8.20) ( 22,  8.16) ( 23,  8.13) ( 24,  8.10) ( 25,  8.07) ( 26,  8.04) ( 27,  7.99) ( 28,  7.96) ( 29,  7.94) ( 30,  7.91) ( 31,  7.88) ( 32,  7.84) ( 33,  7.79) ( 34,  7.76) ( 35,  7.73) ( 36,  7.69) ( 37,  7.65) ( 38,  7.61) ( 39,  7.59) ( 40,  7.55) ( 41,  7.52) ( 42,  7.48) ( 43,  7.44) ( 44,  7.39) ( 45,  7.37) ( 46,  7.33) ( 47,  7.31) ( 48,  7.27) ( 49,  7.24) ( 50,  7.21) ( 51,  7.18) ( 52,  7.15) ( 53,  7.09) ( 54,  7.07) ( 55,  7.03) ( 56,  7.00) ( 57,  6.97) ( 58,  6.95) ( 59,  6.91) ( 60,  6.87) ( 61,  6.83) ( 62,  6.79) ( 63,  6.74) ( 64,  6.72) ( 65,  6.67) ( 66,  6.64) ( 67,  6.62) ( 68,  6.59) ( 69,  6.55) ( 70,  6.52) ( 71,  6.46) ( 72,  6.44) ( 73,  6.40) ( 74,  6.38) ( 75,  6.34) ( 76,  6.31) ( 77,  6.28) ( 78,  6.24) ( 79,  6.21) ( 80,  6.15) ( 81,  6.13) ( 82,  6.09) ( 83,  6.06) ( 84,  6.02) ( 85,  6.00) ( 86,  5.97) ( 87,  5.92) ( 88,  5.88) ( 89,  5.86) ( 90,  5.82) ( 91,  5.78) ( 92,  5.75) ( 93,  5.73) ( 94,  5.71) ( 95,  5.66) ( 96,  5.64) ( 97,  5.59) ( 98,  5.55) ( 99,  5.51) (100,  5.47) (101,  5.43) (102,  5.41) (103,  5.36) (104,  5.36) (105,  5.31) (106,  5.28) (107,  5.25) (108,  5.23) (109,  5.18) (110,  5.13) (111,  5.09) (112,  5.04) (113,  5.01) (114,  5.00) (115,  4.96) (116,  4.92) (117,  4.86) (118,  4.83) (119,  4.79) (120,  4.77) (121,  4.72) (122,  4.68) (123,  4.66) (124,  4.63) (125,  4.60) (126,  4.56) (127,  4.52) (128,  4.50) (129,  4.45) (130,  4.43) (131,  4.40) (132,  4.36) (133,  4.34) (134,  4.30) (135,  4.27) (136,  4.24) (137,  4.22) (138,  4.18) (139,  4.16) (140,  4.12) (141,  4.09) (142,  4.06) (143,  4.03) (144,  4.01) (145,  3.95) (146,  3.91) (147,  3.89) (148,  3.85) (149,  3.81) (150,  3.77) (151,  3.75) (152,  3.71) (153,  3.66) (154,  3.62) (155,  3.59) (156,  3.55) (157,  3.51) (158,  3.47) (159,  3.43) (160,  3.39) (161,  3.37) (162,  3.29) (163,  3.27) (164,  3.23) (165,  3.19) (166,  3.13) (167,  3.08) (168,  3.03) (169,  2.99) (170,  2.94) (171,  2.89) (172,  2.84) (173,  2.79) (174,  2.76) (175,  2.72) (176,  2.68) (177,  2.65) (178,  2.61) (179,  2.58) (180,  2.51) (181,  2.54) (182,  2.56) };
\addlegendentry{Average for $\norm{\vec b_i^*}$}

\addplot[domain=1:183,samples=1000, smooth,variable=\x,darkgray,dotted,thick] plot ({\x},{log2( 3.19153824321146 * sqrt(183 - \x + 1) )});

\addlegendentry{Expectation for $\norm{\pi_i(\vec v)}$}

\draw[dashed] (127,1) -- (127,820) node[pos = 0.06, right] {$d-\beta+1$};
\end{axis}
\end{tikzpicture}
\end{frame}

\begin{frame}[label={sec:org1fb986d}]{Don’t treat block-wise lattice reduction as a black box}
\begin{itemize}
\item \fullcite{USENIX:ADPS16}

\item \fullcite{AC:AGVW17}
\end{itemize}
\end{frame}

\begin{frame}[label={sec:org314c535}]{Dual Attack}
Given samples \(\vec{A}, \vec{c}\):

\begin{enumerate}
\item Find a short \(\vec{y}\) solving SIS on \(\vec{A}\).
\item Compute \(\Angle{\vec{y}, \vec{c}}\).
\end{enumerate}

Either \(\vec{c} = \vec{A}\cdot \vec{s} + \vec{e}\) or \(\vec{c}\) uniformly random:

\begin{itemize}
\item If \(\vec{c}\) is uniformly random, so is \(\Angle{\vec{y}, \vec{c}}\).
\item If \(\vec{c} = \vec{A} \cdot \vec{s} + \vec{e}\), then \(\Angle{\vec{y}, \vec{c}} = \Angle{\vec{y} \cdot \vec{A}, \vec{s}} + \Angle{\vec{y}, \vec{e}} \equiv \Angle{\vec{y}, \vec{e}} \pmod{q}\). If \(\vec{y}\) is sufficiently short, then \(\Angle{\vec{y}, \vec{e}}\) will also be short, since \(\vec{e}\) is also small.
\end{itemize}
\end{frame}

\begin{frame}[label={sec:org092bf2b},fragile]{Algorithm Sketch}
\begin{small}
\begin{algorithm}[H]
  \SetKwFor{MRepeat}{repeat}{}{}
  \(ε_{d} \gets \exp(-π{({\mathrm{Exp}[\vecnorm{\vec{y}_{i}}]}⋅α)}^{2})\)\; 
  \(m \gets \lceil2\,\log(2 - 2\,\alert{ε_{t}})/\log(1 - 4\, ε_{d}^{2})\rceil\); 
 
  \(\mat{P} \sample\) \(n \times n\) permutation matrices\;
  \([\mat{A}_{0} \mid \mat{A}_{1}] \gets \mat{A} ⋅ \mat{P}\) with \(\mat{A}_{0} \in \ZZ_{q}^{m \times (n-\alert{k})}\)\;
  \(\mat{L} \gets\) basis for \(\{(\vec{y}, \vec{x}/\alert{c}) \in \ZZ^m × {({1}/{c} ⋅ \ZZ)}^n : \vec{y} ⋅ \mat{A}_{0} ≡ \vec{x} \bmod q\}\)\;
  \(\mat{L}' \gets\) BKZ-\(\alert{β}\) reduced basis for \(\mat{L}\)\;
  \For{\(i \gets 0\) \KwTo{} \(m-1\)}{
    \(\mat{U} \sample \) a sparse unimodular matrix with small entries\;
    \(\mat{L}_{i} \gets \) \(\mat{U} ⋅ \mat{L}'\)\;
    \(\mat{L}'_{i} \gets \) BKZ-\(\alert{β'}\) reduced basis for \(\mat{L}_{i}\)\;
    \((\vec{w}_{i},\vec{v}_{i}) \gets\) shortest row vector in \(\mat{L}'_{i}\)\;
    \(e'_{i} \gets \langle{\vec{w}_{i}},{\vec{c}}\rangle\)\;      
  }
  \lIf{\(e'_{i}\) follow discrete Gaussian distribution}{\Return\(\top\)}
  \Return\(\bot\)\;
\end{algorithm}
\end{small}
\end{frame}

\begin{frame}[label={sec:org048637e}]{Opening Black Boxes}
\begin{itemize}
\item Lattice reduction returns more than one somewhat short vector

\item Inner products have algebraic meaning beyond returning somewhat short elements
\end{itemize}

\begin{block}{}
\fullcite{EC:Albrecht17}
\end{block}
\end{frame}

\begin{frame}[label={sec:org2b4b659}]{Sources for future refinements}
\begin{itemize}
\item There are more black boxes to be opened, e.g.:
\begin{itemize}
\item enumeration/sieving inside BKZ \footfullcite{EPRINT:Ducas17}
\item BDD enumeration and small/sparse secrets
\end{itemize}
\item Cost of lattice reduction not fully understood
\end{itemize}

\begin{block}{Note}
Estimates in standards document are quite conservative and price some of these anticipated improvements in.
\end{block}
\end{frame}

\begin{frame}[label={sec:orgf6b4de7}]{Code = Research}
\begin{center}
\url{https://bitbucket.org/malb/lwe-estimator}
\end{center}

\begin{description}
\item[{relied upon}] NIST PQC submissions and HE standard security document
\item[{one man show}] about 300 commits, mostly by me
\item[{quality control}] tests, documentation but \textbf{no peer review}
\item[{bugs}] there have been bugs leading to false security estimates and plenty of potential for more: numerical stability, heuristics for pruning branches in a search tree, …
\end{description}
\end{frame}

\begin{frame}[label={sec:orge8cc5d8},standout]{Fin}
\begin{center}
\Huge \alert{Thank You}
\end{center}
\end{frame}
\end{document}