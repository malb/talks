\documentclass[10pt]{beamer}

\definecolor{darkred}{HTML}{A60000}
\definecolor{lightblue}{HTML}{366690}
\definecolor{darkgreen}{HTML}{009000}
\definecolor{lightbrown}{HTML}{AB6402}
\definecolor{yellow9}{HTML}{E1B640}

\usepackage{amsmath,amsfonts,amssymb}
\usepackage{xspace}
\usepackage[utf8]{inputenc}

% Α α Β β Γ γ Δ δ Ε ε Ζ ζ Η η Θ θ Ι ι Κ κ Λ λ Μ μ Ν ν Ξ ξ Ο ο Π π Ρ ρ Σ σ Τ τ Υ υ Φ φ Χ χ Ψ ψ Ω ω

\DeclareUnicodeCharacter{03B3}{\ensuremath{\gamma}\xspace}
\DeclareUnicodeCharacter{03B4}{\ensuremath{\delta}\xspace}
\DeclareUnicodeCharacter{03B7}{\ensuremath{\eta}\xspace}
\DeclareUnicodeCharacter{03BA}{\ensuremath{\kappa}\xspace}
\DeclareUnicodeCharacter{03BB}{\ensuremath{\lambda}\xspace}
\DeclareUnicodeCharacter{03BC}{\ensuremath{\mu}\xspace}
\DeclareUnicodeCharacter{03BD}{\ensuremath{\nu}\xspace}
\DeclareUnicodeCharacter{03C0}{\ensuremath{\pi}\xspace}
\DeclareUnicodeCharacter{03C3}{\ensuremath{\sigma}\xspace}
\DeclareUnicodeCharacter{03C4}{\ensuremath{\tau}\xspace}
\DeclareUnicodeCharacter{2208}{\ensuremath{\in}\xspace}
\DeclareUnicodeCharacter{2264}{\ensuremath{\le}\xspace}
\DeclareUnicodeCharacter{2265}{\ensuremath{\ge}\xspace}
\DeclareUnicodeCharacter{00B7}{\ensuremath{\cdot}\xspace}
\DeclareUnicodeCharacter{2261}{\ensuremath{\equiv}\xspace}


%%% Macros

\newcommand{\ZZ}[1][blank]{\ensuremath{
    \ifthenelse{\equal{#1}{blank}}
    {\mathbb{Z}}
    {\mathbb{Z}\left[#1\right]}\xspace}}
\newcommand{\QQ}[1][blank]{\ensuremath{
        \ifthenelse{\equal{#1}{blank}}
        {\mathbb{Q}}
        {\mathbb{Q}\left[#1\right]}\xspace}}
\newcommand{\ZZq}[1][blank]{\ensuremath{
        \ifthenelse{\equal{#1}{blank}}
        {\mathbb{Z}_q}
        {\mathbb{Z}_q\left[#1\right]}\xspace}}

\newcommand{\mat}[1]{\ensuremath{\mathbf{#1}}\xspace}
\renewcommand{\vec}[1]{\ensuremath{\mathbf{#1}}\xspace}
        

\usepackage{algorithm2e}
\usepackage{listings}

\lstdefinelanguage{Sage}[]{Python}{morekeywords={True,False,sage,with},sensitive=true}

\lstset{ %
  backgroundcolor=\color[rgb]{0.95,0.95,0.90},   % choose the background color; you must add \usepackage{color} or \usepackage{xcolor}
  basicstyle=\color{darkgray}\scriptsize\ttfamily, % the size of the fonts that are used for the code
  breakatwhitespace=false,         % sets if automatic breaks should only happen at whitespace
  breaklines=true,                 % sets automatic line breaking
  captionpos=none,                    % sets the caption-position to none
  commentstyle=\color[rgb]{0.133,0.545,0.133},
  deletekeywords={...},            % if you want to delete keywords from the given language
  escapeinside={\%*}{*)},          % if you want to add LaTeX within your code
  extendedchars=true,              % lets you use non-ASCII characters; for 8-bits encodings only, does not work with UTF-8
  frame=none,                    % adds a frame around the code
  keywordstyle=\bf\ttfamily\color[rgb]{0,.3,.7},
  morekeywords={},                 % if you want to add more keywords to the set
  numbers=none,                    % where to put the line-numbers; possible values are (none, left, right)
  numbersep=5pt,                   % how far the line-numbers are from the code
  numberstyle=\tiny\color{gray}, % the style that is used for the line-numbers
  rulecolor=\color{black},         % if not set, the frame-color may be changed on line-breaks within not-black text (e.g. comments (green here))
  showspaces=false,                % show spaces everywhere adding particular underscores; it overrides 'showstringspaces'
  showstringspaces=false,          % underline spaces within strings only
  showtabs=false,                  % show tabs within strings adding particular underscores
  stepnumber=1,                    % the step between two line-numbers. If it's 1, each line will be numbered
  stringstyle={\color[rgb]{0.75,0.49,0.07}},
  tabsize=4,                       % sets default tabsize to 2 spaces
  title=\lstname,                   % show the filename of files included with \lstinputlisting; also try caption instead of title
  language=Sage,                   % the language of the code
}
        

\usepackage{relsize}
\usepackage{comment}

\usepackage{tikz,pgfplots}
\pgfplotsset{compat=1.9, width=1.0\textwidth, height=0.8\textheight}


\mode<presentation>
{
  \setbeamercovered{transparent}
  \setbeamercolor{normal text}{fg=white,bg=gray}
  \setbeamercolor{alerted text}{fg=white}
  \setbeamercolor{example text}{fg=white}
  \setbeamercolor{background canvas}{bg=darkgray} 
  \setbeamercolor{structure}{fg=white}

  \setbeamercolor{block title}{bg=lightblue,fg=white}
  \setbeamercolor{block body}{bg=white,fg=darkgray}

  \setbeamercolor{block title example}{bg=lightblue,fg=white}
  \setbeamercolor{block body example}{bg=white,fg=darkgray}

  \setbeamercolor{palette primary}{use=normal text,fg=normal text.fg}
  \setbeamercolor{palette quaternary}{use=structure,fg=structure.fg}
  \setbeamercolor{palette secondary}{use=structure,fg=structure.fg}
  \setbeamercolor{palette tertiary}{use=normal text,fg=normal text.fg}

  \setbeamercolor{palette primary}{use=structure,fg=structure.fg}

  \setbeamercolor{math text}{}
  \setbeamercolor{math text inlined}{parent=math text}
  \setbeamercolor{math text displayed}{parent=math text}

  \setbeamercolor{enumerate item}{fg=lightgray}
  \setbeamercolor{itemize item}{fg=lightgray}
  \setbeamercolor{itemize subitem}{fg=lightgray}

  \setbeamercolor{normal text in math text}{}

  \setbeamercolor{local structure}{parent=structure}
  
  \setbeamercolor{titlelike}{parent=structure}

  \setbeamercolor{title}{parent=titlelike}
  \setbeamercolor{title in head/foot}{parent=palette quaternary}
  \setbeamercolor{title in sidebar}{parent=palette sidebar quaternary}

  \setbeamercolor{subtitle}{parent=title}

  \setbeamertemplate{navigation symbols}{}
}

\newcommand{\Z}{\ensuremath{\mathbb{Z}}\xspace}


%%%%%%%%%%%%%%%%%%%%%%%%%%%%%%
% Presentation Title Content %
%%%%%%%%%%%%%%%%%%%%%%%%%%%%%%
\title{Heartbleed}
\subtitle{or: I read the news, too}
\author[Martin R.\ Albrecht]{Martin R.\ Albrecht}
\institute{Information Security Group, Royal Holloway, University of London}

\date{}

\AtBeginSection[] {
	\begin{frame}
		\frametitle{Contents}
		\tableofcontents[sectionstyle=show/shaded,subsectionstyle=show/shaded]
	\end{frame}
}

\AtBeginSubsection[] {
    \begin{frame}
        \frametitle{Contents}
        \tableofcontents[sectionstyle=show/shaded,subsectionstyle=show/shaded]
    \end{frame}
}

\begin{document}

\begin{frame}[plain] % frame of type 'plain' is an empty frame
\begin{columns}
  \column{0.3\textwidth}
  \begin{center}
  \includegraphics[width=1\textwidth]{./heartbleed-logo.png}
  % heartbleed-logo.png: 341x413 pixel, 72dpi, 12.03x14.57 cm, bb=0 0 341 413
\end{center}

  \column{0.7\textwidth}\titlepage
\end{columns}
\end{frame}


%%%%%%%%%%%%%%%%%%%%%%%%%%%
% Table of Contents Slide %
%%%%%%%%%%%%%%%%%%%%%%%%%%%

\begin{frame}{XKCD \#1354}

\includegraphics<1>[clip,trim = 0 1118 0   34,width=\textwidth]{xkcd.png}
\includegraphics<2>[clip,trim = 0  894 0  258,width=\textwidth]{xkcd.png}
\includegraphics<3>[clip,trim = 0  671 0  481,width=\textwidth]{xkcd.png}
\includegraphics<4>[clip,trim = 0  448 0  704,width=\textwidth]{xkcd.png}
\includegraphics<5>[clip,trim = 0  224 0  928,width=\textwidth]{xkcd.png}
\includegraphics<6>[clip,trim = 0    0 0 1152,width=\textwidth]{xkcd.png}

\end{frame}

\begin{frame}[allowframebreaks]{RFC 6520: Transport Layer Security (TLS) and Datagram Transport Layer Security (DTLS) Heartbeat Extension}

\begin{exampleblock}{}
  {\em This document describes the Heartbeat Extension for the Transport
   Layer Security (TLS) and Datagram Transport Layer Security (DTLS)
   protocols.

   The Heartbeat Extension provides a new protocol for TLS/DTLS allowing
   the usage of keep-alive functionality without performing a
   renegotiation and a basis for path MTU (PMTU) discovery for DTLS.}
  \vskip1mm
  \hspace*\fill{\small--- RFC 6520}
\end{exampleblock} 

\framebreak

 \begin{itemize}
   \item The heartbeat extension implements a simple ping:\\``Are you still there?''
   \item This is a common feature for communication protocols:
   \begin{itemize}
    \item ICMP
    \item HTTP keep-alive
    \item SSH KeepAlive
    \item \dots
    \end{itemize}
 \end{itemize}

 \framebreak
 
\begin{exampleblock}{}
  {\em The larger takeaway actually isn't ``This wouldn't have happened if we didn't add Ping'', the takeaway is ``We can't even add Ping, how the heck are we going to fix everything else?''.}
  \vskip1mm
  \hspace*\fill{\small--- Dan Kaminsky}
\end{exampleblock} 
 
\begin{quote}
  
\end{quote}

\end{frame}


\begin{frame}{The bug \dots}

\begin{itemize}
 \item \dots is not in the standard: RFC 6520.
 \item \dots is in an implementation of RFC 6520 in OpenSSL.
 \item \dots affectes OpenSSL versions 
  \begin{itemize}
    \item 1.0.1 (up until and including 1.0.1f) and 
    \item 1.0.2 (beta).
  \end{itemize}
  
 \item \dots does not affect other SSL implementations as far as we know.
\end{itemize}
\end{frame}

\begin{frame}{What is OpenSSL?}

\begin{itemize}
  \item TLS (formerly known as SSL) is the most used protocol to encrypt traffic on the Internet.
  \item OpenSSL is an implementation of the TLS standard.
  \item OpenSSL is widely used.
\end{itemize}
\end{frame}

\begin{frame}[fragile]{Show me simplified code}
\framesubtitle{{\scriptsize \textbf{source:} \url{http://www.tedunangst.com/flak/post/heartbleed-vs-mallocconf}}}

\begin{lstlisting}[language=c,morekeywords={malloc,read,memcpy,write}]
struct { 
    unsigned short len;
    char payload[]; 
} *packet; 

packet = malloc(amt);
read(s, packet, amt);
buffer = malloc(packet->len);
/* malb: packet->len == amt? */
memcpy(buffer, packet->payload, packet->len);
write(s, buffer, packet->len);
\end{lstlisting}

\end{frame}

\begin{frame}[fragile,allowframebreaks]{Show me the actual code}
\framesubtitle{{\scriptsize \textbf{main source:} \url{http://blog.existentialize.com/diagnosis-of-the-openssl-heartbleed-bug.html/}}}

Here is the HeartbeatMessage packet definition:

\begin{lstlisting}[language=c,morekeywords={opaque,uint16}]
struct {
     HeartbeatMessageType type;
     uint16 payload_length;
     opaque payload[HeartbeatMessage.payload_length];
     opaque padding[padding_length];
  } HeartbeatMessage;
\end{lstlisting}

\framebreak

and here is the function processing heartbeats in OpenSSL:

\begin{lstlisting}[language=c,morekeywords={}]
int  dtls1_process_heartbeat(SSL *s) {          
  unsigned char *p = &s->s3->rrec.data[0], *pl;
  unsigned short hbtype;
  unsigned int payload;
  unsigned int padding = 16; /* Use minimum padding */
\end{lstlisting}

so we get a pointer to the data within an SSLv3 record:

\begin{lstlisting}[language=c,morekeywords={}]
typedef struct ssl3_record_st {
  int type;               /* type of record */
  unsigned int length;    /* How many bytes available */
  unsigned int off;       /* read/write offset into 'buf' */
  unsigned char *data;    /* pointer to the record data */
  unsigned char *input;   /* where the decode bytes are */
  unsigned char *comp;    /* only used with decompression  ... */
  unsigned long epoch;    /* epoch number, needed by DTLS1 */
  unsigned char seq_num[8]; /* sequence number, needed by DTLS1 */
} SSL3_RECORD;
\end{lstlisting}

\framebreak

Back to \texttt{dtls1\_process\_heartbeat}:

\begin{lstlisting}[language=c,morekeywords={memcpy,n2s,s2n}]
/* Read type and payload length first */
hbtype = *p++; // malb: p points to start of the message
n2s(p, payload); //malb: read 16-bit length into payload and p+=2
pl = p; // malb: points to payload sent by user
\end{lstlisting}

Later on, the reply is constructed, but first sufficient memory is allocated:

\begin{lstlisting}[language=c,morekeywords={memcpy,n2s,s2n}]
unsigned char *buffer, *bp;
int r;

/* Allocate memory for the response, size is 1 byte
 * message type, plus 2 bytes payload length, plus
 * payload, plus padding
 */
buffer = OPENSSL_malloc(1 + 2 + payload + padding);
bp = buffer;
\end{lstlisting}

and then the message is constructed:

\begin{lstlisting}[language=c,morekeywords={memcpy,n2s,s2n}]
/* Enter response type, length and copy payload */
*bp++ = TLS1_HB_RESPONSE; //malb: set the type
s2n(payload, bp); //malb: write length and bp+=2
memcpy(bp, pl, payload); //malb: fire!
\end{lstlisting}

\framebreak

\begin{itemize}
  \item The offending line is
\begin{lstlisting}[language=c,morekeywords={memcpy,n2s,s2n}]
memcpy(bp, pl, payload);
\end{lstlisting}
where we copy \texttt{payload} bytes from {\tt pl}.

 \item The variable \texttt{payload} is controlled by the user as is the actual length of data in \texttt{pl}. 
 \item The user can hence request a read from {\tt pl} requesting more data than {\tt pl} has -- up to 64kb. 
\end{itemize}

\vspace{1em}

What is after {\tt pl}? A pot of gold!
\end{frame}

\begin{frame}[allowframebreaks,fragile]{Memory}

\begin{itemize}
  \item {\tt pl} lives on the heap.
  \item Memory from the heap is requested with {\tt malloc()} and returned to the OS with {\tt free()}.
  \item When asked for a certain number of bytes, the OS will find a bit of unused memory and return that (if {\tt mmap()} isn't called)
\begin{lstlisting}
-------- | -------- | -------- | -------- 
-------- | -------- | --xxxxxx | -------- 
-------- | -------- | -------- | -------- <- many bytes free
xxxxxxxx | -------- | -------- | -------- 
xxxxxxxx | xxxxxxxx | xxxxxxxx | xxxxxxxx 
xxxxxxxx | xxxxxxxx | xxx--xxx | xxxxxxxx <- 2 bytes free 
xxxxxxxx | xx------ | xxxxxxxx | xxxxxxxx <- 4 bytes free
\end{lstlisting}
 \item depending on where the OS put {\tt pl} different data is after it
\end{itemize}

\framebreak

\begin{itemize}
  \item If the data behind {\tt pl} is in use, the attacker gets to see that
  \item If the data behind {\tt pl} is not in use and was returned to the OS, it depends on the OS if it is overwritten with dummy data or not.
\end{itemize}

However,
\begin{itemize}
  \item OpenSSL seldomly returns data to the OS.
  \item Instead, unused memory is re-used internally for performance reasons.
  \item This renders exploit mitigation techniques (such as overwriting free'd data) useless.
\end{itemize}

\begin{block}{Comment}
Some people\footnote{e.g\ \url{http://www.tedunangst.com/flak/post/heartbleed-vs-mallocconf}} claim that overwriting data in calls to {\tt malloc()} and disabling OpenSSL's {\tt malloc()} wrapper would have been a mitigation to this bug, but I don't see how.
\end{block}


\end{frame}

\begin{frame}[fragile]{The fix}
\begin{lstlisting}[language=c,morekeywords={memcpy,n2s,s2n}]
/* Read type and payload length first */
if (1 + 2 + 16 > s->s3->rrec.length)
   return 0; /* silently discard */
hbtype = *p++;
n2s(p, payload);
if (1 + 2 + payload + 16 > s->s3->rrec.length)
    return 0; /* silently discard per RFC 6520 sec. 4 */
pl = p;
\end{lstlisting}

This does two things: 
\begin{itemize}
  \item The first check stops zero-length heartbeats.
  \item The second check checks to make sure that the actual record length is sufficiently long.
\end{itemize}

  
\end{frame}



\begin{frame}
\frametitle{Fin}
\begin{center}
\large{Questions?}

\vspace{2em}

\url{http://heartbleed.com/}

\end{center}
\end{frame}

\end{document}


