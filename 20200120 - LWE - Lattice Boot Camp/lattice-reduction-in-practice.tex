% Created 2020-01-23 Thu 18:48
% Intended LaTeX compiler: pdflatex
\documentclass[xcolor=table,10pt,aspectratio=169]{beamer}
\usepackage{graphicx}
\usepackage{grffile}
\usepackage{longtable}
\usepackage{wrapfig}
\usepackage{rotating}
\usepackage[normalem]{ulem}
\usepackage{amsmath}
\usepackage{textcomp}
\usepackage{amssymb}
\usepackage{capt-of}
\usepackage{hyperref}
\usepackage{microtype}
\usepackage{newunicodechar}
\usepackage[notions,operators,sets,keys,ff,adversary,primitives,complexity,asymptotics,lambda,landau,advantage]{cryptocode}
\usepackage{xspace}
\usepackage{units}
\usepackage{nicefrac}
\usepackage{gensymb}
\usepackage{amsthm}
\usepackage{amsmath}
\usepackage{amssymb}
\usepackage{xcolor}
\usepackage{listings}
\usepackage[color=yellow!40]{todonotes}
%
% Config
%

\usepackage{amsmath,amsfonts,amssymb,amsthm}
\usepackage{ifxetex,ifluatex}
\usepackage{fixltx2e} % provides \textsubscript
\usepackage{xspace}
\usepackage{graphicx}
\usepackage{comment}
\usepackage{url}
\usepackage{relsize}
\usepackage{booktabs}
\usepackage{tabularx}
\renewcommand*{\UrlFont}{\ttfamily\smaller\relax}
\usepackage{subfigure}
\usepackage[normalem]{ulem}
%
% Pseudocode
%

\usepackage[linesnumbered,ruled]{algorithm2e}

%
% Source Code Listings
%

\usepackage{listings}
\lstdefinelanguage{Sage}[]{Python}{morekeywords={True,False,sage,cdef,cpdef,ctypedef,self},sensitive=true}
\lstset{frame=none,
          showtabs=False,
          showspaces=False,
          showstringspaces=False,
          commentstyle={\color{gray}},
          keywordstyle={\color{mLightBrown}\textbf},
          stringstyle ={\color{mDarkBrown}},
          frame=single,
          basicstyle=\tt\scriptsize\relax,
          backgroundcolor=\color{gray!190!black},
          inputencoding=utf8,
          literate={…}{{\ldots}}1,
          belowskip=0.0em,
          }
\usepackage{lstlinebgrd}
%
% Tikz
%

% from pgfplotsthemetol.sty
\definecolor{DarkPurple}{HTML}{332288}
\definecolor{DarkBlue}{HTML}{6699CC}
\definecolor{LightBlue}{HTML}{88CCEE}
\definecolor{DarkGreen}{HTML}{117733}
\definecolor{DarkRed}{HTML}{661100}
\definecolor{LightRed}{HTML}{CC6677}
\definecolor{LightPink}{HTML}{AA4466}
\definecolor{DarkPink}{HTML}{882255}
\definecolor{LightPurple}{HTML}{AA4499}

\definecolor{DarkBrown}{HTML}{604c38}
\definecolor{DarkTeal}{HTML}{23373b}
\definecolor{LightBrown}{HTML}{EB811B}
\definecolor{LightGreen}{HTML}{14B03D}

\usepackage{tikz,pgfplots}
\usetikzlibrary{calc}
\usetikzlibrary{arrows,automata}
\usetikzlibrary{positioning}
\usetikzlibrary{decorations.pathmorphing}
\usetikzlibrary{decorations.pathreplacing}
\usetikzlibrary{backgrounds, fit, shapes.symbols, chains, shapes.geometric, shapes.arrows}
\usetikzlibrary{crypto.symbols}
\pgfplotsset{compat=1.12}
\usetikzlibrary{graphs}
\tikzset{shadows=no}

%% Cache TiKZ pictures

\ifdefined\tikzcaching
\usetikzlibrary{external}
\tikzexternalize[prefix=build/]
\tikzset{external/up to date check=diff} % MD5 fails from within emacs
\fi

%
% SVG

\ifxetex
\newcommand{\executeiffilenewer}[3]{%
 {\immediate\write18{#3}} % hack
}
\else
\newcommand{\executeiffilenewer}[3]{%
 \ifnum\pdfstrcmp{\pdffilemoddate{#1}}%
 {\pdffilemoddate{#2}}>0%
 {\immediate\write18{#3}}\fi%
}
\fi

\newcommand{\includesvg}[2][1.0\textwidth]{%
 \executeiffilenewer{#1.svg}{#1.pdf}%
 {inkscape -z -D --file=#2.svg --export-pdf=#2.pdf --export-latex --export-area-page}%
 \def\svgwidth{#1} 
 \input{#2.pdf_tex}%
} 

%
% Metropolis Theme
% https://github.com/matze/mtheme.git
%

\usetheme{metropolis}
\metroset{color/block=fill}
\metroset{numbering=none}
\metroset{outer/progressbar=foot}
\metroset{titleformat=smallcaps}


%
% UTF-8 all the things
% 
\usepackage{unicodesymbols} % put this after m which loads fontspec

%
% BibLaTeX
%

\usepackage[
    backend=bibtex,
    style=alphabetic,
    citestyle=alphabetic,
]{biblatex}

\bibliography{../local.bib, ../master_refs.bib, ../TLS.bib}

\DeclareFieldFormat{title}{\alert{#1}}
\DeclareFieldFormat[book]{title}{\alert{#1}}
\DeclareFieldFormat[inproceedings]{title}{\alert{#1}}
\DeclareFieldFormat[article]{title}{\alert{#1}}
\DeclareFieldFormat[misc]{title}{\alert{#1}}

%
% Microtype
%

\IfFileExists{upquote.sty}{\usepackage{upquote}}{}
\IfFileExists{microtype.sty}{\usepackage{microtype}}{}


\setlength{\parindent}{0pt}
\setlength{\parskip}{6pt plus 2pt minus 1pt}
\setlength{\emergencystretch}{3em}  % prevent overfull lines
\setcounter{secnumdepth}{0}


% \AtBeginSection[] 
% {
% \begin{frame}<beamer>
% \frametitle{Outline}
% \tableofcontents[currentsection]
% \end{frame}
% }

% \AtBeginSubsection[] 
% {
% \begin{frame}<beamer>
% \frametitle{Outline}
% \tableofcontents[currentsubsection]
% \end{frame}
% }

%%% Local Variables:
%%% mode: latex
%%% End:
\def\enumworstfit{\(1/(2e)\, \beta \log(\beta) - \beta + 16.1\)}
\def\enumavgfit{\(1/8\,\beta \log(\beta) - 0.75\beta + 2.3\)}
\def\qenumworstfit{\(1/(4e)\, \beta \log(\beta) - 0.5\beta + 8\)}
\def\robl{\rowcolor{DarkBlue!20}}
\def\rore{\rowcolor{DarkRed!20}}
\def\rogr{\rowcolor{gray!20}}
\usetheme{default}
\author{Martin R. Albrecht}
\date{Information Security Group, Royal Holloway, University of London\\23 January 2020, Lattice Boot Camp @ Simons\vfill \begin{scriptsize}Based on joint work with Alex Davidson, Amit Deo, Benjamin R. Curtis, Damien Stehlé, Eamonn W. Postlethwaite, Elena Kirshanova, Fernando Virdia, Florian Göpfert, Gottfried Herold, John M. Schanck, Léo Ducas, Marc Stevens, Paul Kirchner, Pierre-Alain Fouque, Rachel Player, Sam Scott, Shi Bai, Thomas Wunderer, Vlad Gheorghiu and Weiqiang Wen as well as the works of many other authors.\end{scriptsize}}
\title{Algorithms for Lattice Problems: In Practice}
\hypersetup{
pdfauthor={Martin R. Albrecht},
pdftitle={Algorithms for Lattice Problems: In Practice},
pdfkeywords={},
pdfsubject={},
pdfcreator={Emacs 26.3 (Org mode 9.3.1)},
pdflang={English},
colorlinks,
citecolor=gray,
filecolor=gray,
linkcolor=gray,
urlcolor=gray
}
\begin{document}

\maketitle

\section{Introduction}
\label{sec:orgef8769b}
\begin{frame}[label={sec:orge2d9b86}]{NIST PQ Round 1: Selected Cost Estimates}
\rowcolors[]{3}{gray!20}{gray!10}

\begin{center}
\small{
\begin{tabular}{rrrrr}
\textbf{Cost Model} $\backslash$    \textbf{Scheme} & \textbf{Kyber} & \textbf{NewHope} & \textbf{NTRU HRSS} & \textbf{SNTRU'}\\
\hline
\(0.292\,β\)\footnotemark & 180 & 259 & 136 & 155\\
\enumworstfit \footnotemark & 456 & 738 & 313 & 370\\
\enumavgfit \footnotemark & 248 & 416 & 165 & 200\\
\hline
\(0.265\,\beta\)\textsuperscript{\ref{orgec23666}} & 163 & 235 & 123 & 140\\
\qenumworstfit & 228 & 369 & 157 & 187\\
\end{tabular}
\footnotetext[1]{\label{orgec23666}\fullcite{USENIX:ADPS16}}\footnotetext[2]{\label{org5a2122d}\fullcite{JMC:AlbPlaSco15}}\footnotetext[3]{\label{org0886df4}\fullcite{NISTPQC-R1:LOTUS17}}
}
\end{center}

\scriptsize{
Source: \fullcite{SCN:ACDDPP18}, \url{https://estimate-all-the-lwe-ntru-schemes.github.io/docs/}
}

\vspace{1em}
\end{frame}

\begin{frame}[label={sec:org35ba1be}]{Learning with Errors}
Given \((\mathbf{A},\vec{c})\), find \(\vec{s}\) when

\[
\left(\begin{array}{c}
\\
\\
\\ 
\vec{c} \\
\\
\\
\\
\end{array} \right) \equiv \left(
\begin{array}{ccc}
\leftarrow & n & \rightarrow \\
\\
\\ 
& \mathbf{A} & \\
\\
\\
\\
\end{array} \right) \cdot \left( \begin{array}{c}
\\\
\\
\vec{s} \\
\\
\\
\end{array} \right) + \left(
\begin{array}{c}
\\
\\
\\ 
\vec{e} \\
\\
\\
\\
\end{array} 
\right)
\]

for \(\vec{c} \in \ZZ_q^{m}\), \(\mathbf{A} \in \ZZ_q^{m \times n}\), and \(\vec{s} \in \ZZ^{n}\) and \(\vec{e} \in \ZZ^{m}\) having small coefficients.
\end{frame}

\begin{frame}[label={sec:orgd3751d7}]{(Matrix-)NTRU}
Let \(\mathbf{F}, \mathbf{G}\) be two \(n \times n\) matrices over \(\ZZ_q\) with short entries. Given
\[\mathbf{H} \equiv \mathbf{F}^{-1} \cdot \mathbf{G}\]
find (a small multiple of) \(\mathbf{F}\) or \(\mathbf{G}\).

\pause

\begin{block}{Note}
I will focus on LWE in this talk, but the techniques translate (with some modifications) to NTRU.
\end{block}
\end{frame}

\section{Primal Approach}
\label{sec:org0e84bdb}
\begin{frame}[label={sec:org0fa159d}]{Unique SVP Approach}
We can reformulate \(\vec{c} - \mathbf{A} \cdot \vec{s} \equiv \vec{e} \bmod q\)  over the Integers as:
\[
  \begin{pmatrix}
    q\mathbf{I} & -\mathbf{A}\\
    0 & \mathbf{I}\\
  \end{pmatrix} \cdot
  \begin{pmatrix}
    \mathbf{*}\\
    \mathbf{s}
  \end{pmatrix} +
  \begin{pmatrix}
    \vec{c}\\
    \vec{0}
  \end{pmatrix} = 
  \begin{pmatrix}
    \vec{e}\\
    \vec{s}
  \end{pmatrix}
\]
Alternatively:
\[
  \mathbf{B} = \begin{pmatrix}
    q\mathbf{I} & -\mathbf{A} & \vec{c}\\
    0 & \mathbf{I} & 0\\
    0 & 0 & 1\\
  \end{pmatrix}, \qquad
  \mathbf{B} \cdot
  \begin{pmatrix}
    \vec{*}\\
    \vec{s}\\
    1
  \end{pmatrix} = 
  \begin{pmatrix}
    \vec{e}\\
    \vec{s}\\
    1
  \end{pmatrix}
\]

In other words, there exists an integer-linear combination of the columns of \(\mathbf{B}\) that produces a vector with “unusually” small coefficients \(\rightarrow\) a unique shortest vector.
\end{frame}

\begin{frame}[label={sec:org560c679}]{Computational Problem}
\begin{block}{Unique Shortest Vector Problem}
Find a unique shortest vector amongst the integer combinations of the columns of:
\[
  \mathbf{B} = \begin{pmatrix}
    q\mathbf{I} & -\mathbf{A} & \vec{c}\\
    0 & \mathbf{I} & 0\\
    0 & 0 & 1\\
  \end{pmatrix}
\]
where \(\mat{B} \in \ZZ^{d \times d}\).
\end{block}

\begin{block}{Decision Variant}
Decide if \(\mat{B}\) has an unusually short vector.

\pause
\end{block}

\begin{alertblock}{NTRU}
For LWE we have (up to \(\pm\)) one such short vector. For NTRU we have \(n\).
\end{alertblock}
\end{frame}

\section{Lattice Reduction}
\label{sec:org1bf23cb}
\begin{frame}[label={sec:org21622dc}]{Lattice Volume}
The volume of a lattice is the volume of its fundamental parallelepiped.

\begin{center}
\includegraphics[width=0.8\linewidth]{./joop-vol3.pdf}
\end{center}

\tiny Picture Credit: Joop van de Pol
\end{frame}

\begin{frame}[label={sec:org47a5847}]{Gaussian Heuristic}
The shortest vector in the lattice has expected norm \[λ_1(Λ) ≈ \textnormal{gh}(d) \cdot \mathsf{Vol}(\Lambda)^{1/d} \approx \sqrt{\frac{d}{2 π e}} \cdot \mathsf{Vol}(\Lambda)^{1/d} .\]

\begin{block}{Unusually Shortest Vector}
When \(λ_1(Λ) \ll \sqrt{\frac{d}{2 π e}} \cdot \mathsf{Vol}(\Lambda)^{1/d}\).
\end{block}
\end{frame}

\begin{frame}[label={sec:orgbd3d18b}]{Length of Gram-Schmidt Vectors}
It will be useful to consider the lengths of the Gram-Schmidt vectors.

The vector \(\vec{b}^*_i\) is the orthogonal projection of \(\vec{b}_i\) to the space spanned by the vectors \(\vec{b}_0, \ldots, \vec{b}_{i-1}\).

\begin{columns}
\begin{column}{0.45\columnwidth}
Informally, this means taking out the contributions in the directions of previous vectors  \(\vec{b}_0, \ldots, \vec{b}_{i-1}\).
\end{column}

\begin{column}{0.45\columnwidth}
\begin{tikzpicture}
\pgfplotsset{width=\textwidth, height=0.6\textwidth}
\draw[->] (0,0) -- (3,1);
\node[] at (3.2,1.2) {$\vec{b}_0$};
\only<1>{\draw[->] (0,0) -- (1,2);}
\only<1>{\node[] at (1.2,2.2) {$\vec{b}_1$};}
\only<2>{\draw[->,color=lightgray] (0,0) -- (1,2);}
\only<2>{\node[color=lightgray] at (1.2,2.2) {$\vec{b}_1$};}
\only<2>{\draw[->,gray] (0,0) -- (-0.5,1.5);}
\only<2>{\node[] at (-0.3,1.7) {$\vec{b}^*_1$};}
\only<1>{\node[] at (-0.3,1.7) {\phantom{$\vec{b}^*_1$}};}
\end{tikzpicture}
\end{column}
\end{columns}
\end{frame}

\begin{frame}[label={sec:org14f5ae7},fragile]{Example}
 \lstset{language=sage,label= ,caption= ,captionpos=b,numbers=none}
\begin{lstlisting}
sage: A = IntegerMatrix.random(120, "qary", k=60, bits=20)[::-1]
sage: M = GSO.Mat(A); M.update_gso()
sage: line([(i,log(r_, 2)/2) for i, r_ in enumerate(M.r())], **plot_kwds)
\end{lstlisting}

\begin{center}
\includegraphics[width=.9\linewidth]{gram-schmidt-norms.png}
\end{center}
\end{frame}

\begin{frame}[label={sec:org5d03646},fragile]{Example - LLL}
 \lstset{language=sage,label= ,caption= ,captionpos=b,numbers=none}
\begin{lstlisting}
sage: A = LLL.reduction(A)
sage: M = GSO.Mat(A); M.update_gso()
sage: line([(i,log(r_, 2)/2) for i, r_ in enumerate(M.r())], **plot_kwds)
\end{lstlisting}

\begin{center}
\includegraphics[width=.9\linewidth]{gram-schmidt-norms-lll.png}
\end{center}
\end{frame}

\begin{frame}[label={sec:org852ead2}]{GSA}
\begin{center}
\includegraphics[width=.9\linewidth]{gram-schmidt-norms-lll.png}
\end{center}

\textbf{Geometric Series Assumption:} The shape after lattice reduction is a line with a flatter slope as lattice reduction gets stronger.\footfullcite{STACS:Schnorr03}
\end{frame}

\begin{frame}[label={sec:org1ccbd50}]{Strong Lattice Reduction: BKZ Algorithm}
\centering
\(
 \left(
     \begin{array}{ccccccccc}
                 &           &           &           &           &           &           &           &           \\
                 &           &           &           &           &           &           &           &           \\
                 &           &           &           &           &           &           &           &           \\
         \only<1-2>{\vec{b}_{0}}   \only<3->{{\color{LightRed} \vec{b}_{0}}}          &
         \only<1-5>{\vec{b}_{1}}   \only<6->{{\color{LightRed} \vec{b}_{1}}}          &
         \only<1-8>{\vec{b}_{2}}   \only<9->{{\color{LightRed} \vec{b}_{2}}}          &
         {\vec{b}_{3}}                                                             &
         {\vec{b}_{4}}                                                             &
         {\vec{b}_{5}}                                                             &
         {\vec{b}_{6}}                                                             &
         {\vec{b}_{7}}                                                             &
         \dots   \\
                 &           &           &           &           &           &           &           &           \\
                 &           &           &           &           &           &           &           &           \\
                 &           &           &           &           &           &           &           &
     \end{array}
        \right)
    \)
    \begin{tikzpicture}[remember picture, overlay]
      \tikzset{shift={(current page.center)},yshift=-1.5cm}
      \node[] at (0,0) (origin) {};
      {\color{DarkBlue} %
        \only<1-3>{%
          \draw (-.1,3) -- (-.1,2) {};
          \draw (-.1,1) -- (-.1,0) {};
          \draw (-3,3) -- (-3,2) {};
          \draw (-3,1) -- (-3,0) {};
          \draw[decorate,decoration={brace,amplitude=10pt}]
          (-3,3.2) -- (-.1,3.2) node [black,midway,yshift=.6cm]
          {$\beta = 5$};
          \only<2>{%
            \draw[decorate,decoration={brace,amplitude=10pt}]
            (-.1,-.2) -- (-3,-.2) {};
          }
        }
        \only<4-6>{%
          \draw (.6,3) -- (.6,2) {};
          \draw (.6,1) -- (.6,0) {};
          \draw (-2.3,3) -- (-2.3,2) {};
          \draw (-2.3,1) -- (-2.3,0) {};
          \draw[decorate,decoration={brace,amplitude=10pt}]
          (-2.3,3.2) -- (.6,3.2) node [black,midway,yshift=.6cm]
          {$\beta = 5$};
          \only<5>{%
            \draw[decorate,decoration={brace,amplitude=10pt}]
            (.6,-.2) -- (-2.3,-.2) {};
          }
        }
        \only<7-9>{%
          \draw (1.3,3) -- (1.3,2) {};
          \draw (1.3,1) -- (1.3,0) {};
          \draw (-1.6,3) -- (-1.6,2) {};
          \draw (-1.6,1) -- (-1.6,0) {};
          \draw[decorate,decoration={brace,amplitude=10pt}]
          (-1.6,3.2) -- (1.3,3.2) node [black,midway,yshift=.6cm]
          {$\beta = 5$};
          \only<8>{%
            \draw[decorate,decoration={brace,amplitude=10pt}]
            (1.3,-.2) -- (-1.6,-.2) {};
          }
        }
      }
      \node (oracle) at (-4,-1.8) {\includegraphics[scale=0.9]{oracle.png}};
      \only<2>{%
        \draw[->] (-2.8,-.5) to[in=70,out=160] (-4,-.8);
        \draw[->] (-3,-2) to [in=270,out=20] (-0.5,-.5);
      }
      \only<5>{%
        \draw[->] (-2.1,-.5) to[in=70,out=160] (-4,-.8);
        \draw[->] (-3,-2) to [in=270,out=20] (.2,-.5);
      }
      \only<8>{%
        \draw[->] (-1.4,-.5) to[in=70,out=160] (-4,-.8);
        \draw[->] (-3,-2) to [in=270,out=20] (.2,-.5);      
      }
      \node at (5, -2.5) {\tiny{Picture credit: Eamonn Postlethwaite}};
\end{tikzpicture}
\end{frame}

\begin{frame}[label={sec:orge599741}]{BKZ Algorithm}
\begin{algorithm}[H]
  \KwData{LLL-reduced lattice basis \(\mat{B}\)}
  \KwData{block size \(\beta\)}
  \SetKwFor{MRepeat}{repeat}{}{}
  \MRepeat{until no more change}{
    \For{\(\kappa \gets 0\) \KwTo{} \(d-1\)}{
        LLL  on local projected block \([\kappa,\ldots,\kappa+\beta-1]\)\; 
        \(\vec{v} \gets \) find shortest vector in local projected block \([\kappa,\ldots,\kappa+\beta-1]\)\;
        insert $\vec{v}$ into $\vec{B}$\;
    }
  }
\end{algorithm}
\end{frame}

\begin{frame}[label={sec:org3b62c30}]{Quality: Guarantees}
\begin{columns}[t]
\begin{column}{0.50\columnwidth}
\textbf{BKZ}

\begin{itemize}
\item \(\|\vec b_{0}\| \leq \sqrt{\gamma_{\beta}}^{\frac{d-1}{\beta-1} + 1} \cdot {\vol(\Lambda)}^{1/d}\) and
\item \(\|\vec b_{0}\| \leq \gamma_{\beta}^{\frac{d-1}{\beta-1}} \cdot \lambda_{1}(\Lambda)\)
\end{itemize}
\end{column}

\begin{column}{0.50\columnwidth}
\textbf{Slide}

\begin{itemize}
\item \(\|\vec b_{0}\| \leq \sqrt{(1+\epsilon)\cdot \gamma_{\beta}}^{\frac{d-1}{\beta-1}} \cdot {\vol(\Lambda)}^{1/d}\) and

\item \(\|\vec b_{0}\| \leq {\left((1+\epsilon)\cdot \gamma_{\beta}\right)}^{\frac{d-\beta}{\beta-1}} \cdot \lambda_{1}(\Lambda)\)
\end{itemize}
\end{column}
\end{columns}

\begin{table}[htbp]
\centering
\begin{tabular}{rrrrrrrrr}
\(\beta\) & 2 & 3 & 4 & 5 & 6 & 7 & 8 & 24\\
\hline
\(\gamma\)\textsubscript{\(\beta\)}\textsuperscript{1/(2(\(\beta\)-1))} & 1.074 & 1.059 & 1.059 & 1.053 & 1.052 & 1.050 & 1.050 & 1.031\\
\end{tabular}
\caption{Hermite’s constant \(\gamma_{\beta}\) in dimension \(\beta\).}

\end{table}

\scriptsize

\fullcite{SchEuc94}

\fullcite{STOC:GamNgu08}
\end{frame}

\begin{frame}[label={sec:orgef2e89c}]{Quality: Average I}
\begin{columns}[t]
\begin{column}{0.50\columnwidth}
\textbf{BKZ}

\begin{itemize}
\item \(\|\vec b_{0}\| \approx {\delta_{\beta}}^{{d-1}} \cdot {\vol(\Lambda)}^{1/d}\) or
\item \(\|\vec b_{0}\| \approx {\delta_{\beta}}^{2\cdot{(d-1)}} \cdot \lambda_{1}(\Lambda)\)
\end{itemize}
\end{column}

\begin{column}{0.50\columnwidth}
\textbf{Slide}

\begin{itemize}
\item \(\|\vec b_{0}\| \approx {\delta_{\beta}}^{{d-1}} \cdot {\vol(\Lambda)}^{1/d}\) or

\item \(\|\vec b_{0}\| \approx {\delta_{\beta}}^{2\cdot{(d-\beta)}} \cdot \lambda_{1}(\Lambda)\)
\end{itemize}
\end{column}
\end{columns}

\begin{center}
\begin{tabular}{rrrrrrrr}
\(\beta\) & 2 & 5 & 24 & 50 & 100 & 200 & 500\\
\hline
\(\delta\)\textsubscript{\(\beta\)} & 1.0219 & 1.0186 & 1.0142 & 1.0121 & 1.0096 & 1.0063 & 1.0034\\
\end{tabular}

\end{center}

\begin{itemize}
\item We have \(\delta_{\beta} = \textnormal{gh}(\beta)^{1/(\beta-1)}\) for \(\beta > 50\).
\item The slope under the \textbf{Geometric Series Assumption} is
\end{itemize}
\[\alpha_{\beta} = \delta_{\beta}^{-2}.\]
\end{frame}

\begin{frame}[label={sec:org7a61502}]{Quality: Average II}
\tikzset{external/export=true}
\tikzsetnextfilename{root-hermite-factor}
\begin{tikzpicture}
\pgfplotsset{width=\textwidth, height=0.4\textwidth}

\begin{axis}[xmin=0,xlabel={$\beta$},ylabel={$\delta_{\beta}$},legend pos=north east, legend style={fill=none},  yticklabel style={/pgf/number format/fixed, /pgf/number format/precision=4}]
         	
\addplot[black, thick] coordinates {
( 2, 1.02190) ( 5, 1.01862) (10, 1.01616) (15, 1.01485) 
(20, 1.01420) (25, 1.01342) (28, 1.01331) (40, 1.01295)
(50, 1.01206486355485) (60, 1.01145310214785) (70, 1.01083849117278)
(80, 1.01026264533039) (90, 1.00973613406057) (100, 1.00925872103633)
(110, 1.00882653150498) (120, 1.00843474281592) (130, 1.00807860284815)
(140, 1.00775378902354) (150, 1.00745650119215) (160, 1.00718344897388)
(170, 1.00693180103572) (180, 1.00669912477197) (190, 1.00648332800111)
(200, 1.00628260691082) (210, 1.00609540127612) (220, 1.00592035664374)
(230, 1.00575629268952) (240, 1.00560217684407) (250, 1.00545710232739)
};
\addlegendentry{$\delta_{\beta} = (\frac{\beta}{2\pi e} \cdot (\pi\, \beta)^{1/\beta} )^{\frac{1}{2(\beta-1)}}$};

\end{axis}
\end{tikzpicture}
\tikzset{external/export=false}

\scriptsize{

\fullcite{PhD:Chen13}

}
\end{frame}

\begin{frame}[allowframebreaks]{Behaviour in Practice: BKZ-60 in Dimension 180}
\tikzset{external/export=true}

\vspace{-0.8em}
\tikzsetnextfilename{bkz-behaviour-0-lll}
\begin{tikzpicture}
  \begin{axis}[ylabel=\(\log_2(\|\vec{b}_i^*\|)\),xlabel=\(i\),legend pos=north east,height=0.5\textwidth,ymin=3,ymax=16,xmin=0,xmax=180]
    \addplot+[black] table [x=i, y=gsa, col sep=comma]{data/bkz-60-180.csv};
    \addlegendentry{GSA};
    \addplot+[] table [x=i, y=lll, col sep=comma]{data/bkz-60-180.csv};
    \addlegendentry{LLL};
  \end{axis}
\end{tikzpicture}

\framebreak

\tikzsetnextfilename{bkz-behaviour-1}
\begin{tikzpicture}
  \begin{axis}[ylabel=\(\log_2(\|\vec{b}_i^*\|)\),xlabel=\(i\),legend pos=north east,height=0.5\textwidth,ymin=3,ymax=16,xmin=0,xmax=180]
    \addplot+[black] table [x=i, y=gsa, col sep=comma]{data/bkz-60-180.csv};
    \addlegendentry{GSA};
    \addplot+[] table [x=i, y=tour0, col sep=comma]{data/bkz-60-180.csv};
    \addlegendentry{Tour 0};
  \end{axis}
\end{tikzpicture}

\framebreak

\tikzsetnextfilename{bkz-behaviour-2}
\begin{tikzpicture}
  \begin{axis}[ylabel=\(\log_2(\|\vec{b}_i^*\|)\),xlabel=\(i\),legend pos=north east,height=0.5\textwidth,ymin=3,ymax=16,xmin=0,xmax=180]
    \addplot+[black] table [x=i, y=gsa, col sep=comma]{data/bkz-60-180.csv};
    \addlegendentry{GSA};
    \addplot+[] table [x=i, y=tour1, col sep=comma]{data/bkz-60-180.csv};
    \addlegendentry{Tour 1};
  \end{axis}
\end{tikzpicture}


\framebreak

\tikzsetnextfilename{bkz-behaviour-3}
\begin{tikzpicture}
  \begin{axis}[ylabel=\(\log_2(\|\vec{b}_i^*\|)\),xlabel=\(i\),legend pos=north east,height=0.5\textwidth,ymin=3,ymax=16,xmin=0,xmax=180]
    \addplot+[black] table [x=i, y=gsa, col sep=comma]{data/bkz-60-180.csv};
    \addlegendentry{GSA};
    \addplot+[] table [x=i, y=tour2, col sep=comma]{data/bkz-60-180.csv};
    \addlegendentry{Tour 2};
  \end{axis}
\end{tikzpicture}

\framebreak

\tikzsetnextfilename{bkz-behaviour-4}
\begin{tikzpicture}
  \begin{axis}[ylabel=\(\log_2(\|\vec{b}_i^*\|)\),xlabel=\(i\),legend pos=north east,height=0.5\textwidth,ymin=3,ymax=16,xmin=0,xmax=180]
    \addplot+[black] table [x=i, y=gsa, col sep=comma]{data/bkz-60-180.csv};
    \addlegendentry{GSA};
    \addplot+[] table [x=i, y=tour3, col sep=comma]{data/bkz-60-180.csv};
    \addlegendentry{Tour 3};
  \end{axis}
\end{tikzpicture}

\framebreak

\tikzsetnextfilename{bkz-behaviour-5}
\begin{tikzpicture}
  \begin{axis}[ylabel=\(\log_2(\|\vec{b}_i^*\|)\),xlabel=\(i\),legend pos=north east,height=0.5\textwidth,ymin=3,ymax=16,xmin=0,xmax=180]
    \addplot+[black] table [x=i, y=gsa, col sep=comma]{data/bkz-60-180.csv};
    \addlegendentry{GSA};
    \addplot+[] table [x=i, y=tour4, col sep=comma]{data/bkz-60-180.csv};
    \addlegendentry{Tour 4};
  \end{axis}
\end{tikzpicture}

\framebreak

\tikzsetnextfilename{bkz-behaviour-6}
\begin{tikzpicture}
  \begin{axis}[ylabel=\(\log_2(\|\vec{b}_i^*\|)\),xlabel=\(i\),legend pos=north east,height=0.5\textwidth,ymin=3,ymax=16,xmin=0,xmax=180]
    \addplot+[black] table [x=i, y=gsa, col sep=comma]{data/bkz-60-180.csv};
    \addlegendentry{GSA};
    \addplot+[] table [x=i, y=tour5, col sep=comma]{data/bkz-60-180.csv};
    \addlegendentry{Tour 5};
  \end{axis}
\end{tikzpicture}

\framebreak

\tikzsetnextfilename{bkz-behaviour-7}
\begin{tikzpicture}
  \begin{axis}[ylabel=\(\log_2(\|\vec{b}_i^*\|)\),xlabel=\(i\),legend pos=north east,height=0.5\textwidth,ymin=3,ymax=16,xmin=0,xmax=180]
    \addplot+[black] table [x=i, y=gsa, col sep=comma]{data/bkz-60-180.csv};
    \addlegendentry{GSA};
    \addplot+[] table [x=i, y=tour6, col sep=comma]{data/bkz-60-180.csv};
    \addlegendentry{Tour 6};
  \end{axis}
\end{tikzpicture}

\framebreak

\tikzsetnextfilename{bkz-behaviour-8}
\begin{tikzpicture}
  \begin{axis}[ylabel=\(\log_2(\|\vec{b}_i^*\|)\),xlabel=\(i\),legend pos=north east,height=0.5\textwidth,ymin=3,ymax=16,xmin=0,xmax=180]
    \addplot+[black] table [x=i, y=gsa, col sep=comma]{data/bkz-60-180.csv};
    \addlegendentry{GSA};
    \addplot+[] table [x=i, y=tour7, col sep=comma]{data/bkz-60-180.csv};
    \addlegendentry{Tour 7};
  \end{axis}
\end{tikzpicture}


\framebreak

\tikzsetnextfilename{bkz-behaviour-9}
\begin{tikzpicture}
  \begin{axis}[ylabel=\(\log_2(\|\vec{b}_i^*\|)\),xlabel=\(i\),legend pos=north east,height=0.5\textwidth,ymin=3,ymax=16,xmin=0,xmax=180]
    \addplot+[black] table [x=i, y=simulator, col sep=comma]{data/bkz-60-180.csv};
    \addlegendentry{Simulator};
    \addplot+[] table [x=i, y=tour7, col sep=comma]{data/bkz-60-180.csv};
    \addlegendentry{Tour 7};
  \end{axis}
\end{tikzpicture}

\tikzset{external/export=false}
\end{frame}

\begin{frame}[label={sec:org18afea7},fragile]{Try it at Home}
 \lstset{language=sage,label= ,caption= ,captionpos=b,numbers=none}
\begin{lstlisting}
from fpylll import *
from fpylll.algorithms.bkz2 import BKZReduction as BKZ2
A = IntegerMatrix.random(180, "qary", k=90, bits=20)
bkz = BKZ2(A)
bkz(BKZ.EasyParam(block_size=60))
\end{lstlisting}

\begin{description}
\item[{\url{https://github.com/fplll/fplll}}] C++ library
\item[{\url{https://github.com/fplll/fpylll}}] Python interface
\item[{\url{https://sagemath.org}}] FPyLLL is in Sage
\item[{\url{https://sagecell.sagemath.org/}}] Sage in your browser
\item[{\url{https://cocalc.com/}}] Sage worksheets in your browser
\end{description}
\end{frame}

\begin{frame}[label={sec:org7effa37}]{Success Condition for uSVP (Expectation)}
Can decide that \(\Lambda = \Lambda(\mat{B})\) has unusually short vector when

\vspace{1em}

\begin{columns}[t]
\begin{column}{0.45\columnwidth}
\textbf{BKZ}

\begin{itemize}
\item \({\delta_{\beta}}^{2\,(d-1)} \cdot \lambda_{1}(\Lambda) < {\delta_{\beta}}^{d-1} \cdot {\vol(\Lambda)}^{1/d}\)

\item \(\lambda_{1}(\Lambda) < {\delta_{\beta}}^{-d+1} \cdot {\vol(\Lambda)}^{1/d}\)
\end{itemize}
\end{column}


\begin{column}{0.45\columnwidth}
\textbf{Slide}

\begin{itemize}
\item \({\delta_{\beta}}^{2\cdot(d-\beta)} \cdot \lambda_{1}(\Lambda) < {\delta_{\beta}}^{d-1} \cdot {\vol(\Lambda)}^{1/d}\)
\item \(\lambda_{1}(\Lambda) < {\delta_{\beta}}^{\alert<3->{2\beta-d-1}} \cdot {\vol(\Lambda)}^{1/d}\)
\end{itemize}

\pause
\end{column}
\end{columns}

\begin{block}{“2016 Estimate”}
\[\alert<4>{\sqrt{\beta/d}} \cdot \norm{(\vec{e} \mid \vec{s} \mid 1)} \approx \sqrt{\beta} \cdot \sigma < \delta_{\beta}^{\alert<3->{2\beta-d-1}} \cdot {\vol(\Lambda)}^{1/d}\]

\scriptsize{

\fullcite{USENIX:ADPS16}

}
\end{block}
\end{frame}

\begin{frame}[label={sec:org54c5972}]{Success Condition for uSVP (Expectation)}
\tikzset{external/export=true}
\tikzsetnextfilename{usv-success-expectation}
\begin{tikzpicture}
\begin{axis}[/pgf/number format/.cd,fixed,ymin = 1,legend pos=north east,legend style={fill=white}, xlabel=,ylabel=$\log_2(\norm \cdot)$,width=\columnwidth, height=0.4\columnwidth, xmin = 1, xmax = 183,legend cell align=left,ymax=9]
%      \draw[->] (-3,0) -- (4.2,0) node[right] {$x$};
%      \draw[->] (0,-3) -- (0,4.2) node[above] {$y$};
\addplot[domain=1:183,smooth,variable=\x,black] plot ({\x},{log2(1.01170246711949^(-2*(\x-1)+183)*54.5751087741536)});
\addlegendentry{GSA for $\norm{\vec b_i^*}$}

\addplot[domain=1:183,samples=1000, smooth,variable=\x,darkgray,dotted,thick] plot ({\x},{log2( 3.19153824321146 * sqrt(183 - \x + 1) )});

\addlegendentry{length of projection of $(\vec{e},\vec{s},1)$}

\draw[dashed] (127,1) -- (127,820) node[pos = 0.06, right] {$d-\beta$};
\end{axis}
\end{tikzpicture}
\tikzset{external/export=false}

\scriptsize{

\fullcite{USENIX:ADPS16}  \phantom{Foo Foo Foo Foo Foo Foo Foo Foo Foo Foo Foo Foo Foo}

}
\end{frame}

\begin{frame}[label={sec:orgde9bff9}]{Success Condition for uSVP (Observed)}
\tikzset{external/export=true}
\tikzsetnextfilename{usv-success-observation}
\begin{tikzpicture}
\begin{axis}[/pgf/number format/.cd,fixed, ymin = 1,legend pos=north east, xlabel= ,ylabel=$\log_2(\norm \cdot)$,width=\columnwidth, height=0.4\columnwidth, xmin = 1, xmax = 183,legend cell align=left,ymax=9]
%      \draw[->] (-3,0) -- (4.2,0) node[right] {$x$};
%      \draw[->] (0,-3) -- (0,4.2) node[above] {$y$};

\addplot[gray,thick,x filter/.code={\pgfmathparse{\pgfmathresult+1.0}}] coordinates {
   (  0,  8.78) (  1,  8.78) (  2,  8.77) (  3,  8.72) (  4,  8.71) (  5,  8.69) (  6,  8.66) (  7,  8.63) (  8,  8.62) (  9,  8.59) ( 10,  8.54) ( 11,  8.53) ( 12,  8.51) ( 13,  8.47) ( 14,  8.43) ( 15,  8.39) ( 16,  8.36) ( 17,  8.34) ( 18,  8.30) ( 19,  8.28) ( 20,  8.24) ( 21,  8.20) ( 22,  8.16) ( 23,  8.13) ( 24,  8.10) ( 25,  8.07) ( 26,  8.04) ( 27,  7.99) ( 28,  7.96) ( 29,  7.94) ( 30,  7.91) ( 31,  7.88) ( 32,  7.84) ( 33,  7.79) ( 34,  7.76) ( 35,  7.73) ( 36,  7.69) ( 37,  7.65) ( 38,  7.61) ( 39,  7.59) ( 40,  7.55) ( 41,  7.52) ( 42,  7.48) ( 43,  7.44) ( 44,  7.39) ( 45,  7.37) ( 46,  7.33) ( 47,  7.31) ( 48,  7.27) ( 49,  7.24) ( 50,  7.21) ( 51,  7.18) ( 52,  7.15) ( 53,  7.09) ( 54,  7.07) ( 55,  7.03) ( 56,  7.00) ( 57,  6.97) ( 58,  6.95) ( 59,  6.91) ( 60,  6.87) ( 61,  6.83) ( 62,  6.79) ( 63,  6.74) ( 64,  6.72) ( 65,  6.67) ( 66,  6.64) ( 67,  6.62) ( 68,  6.59) ( 69,  6.55) ( 70,  6.52) ( 71,  6.46) ( 72,  6.44) ( 73,  6.40) ( 74,  6.38) ( 75,  6.34) ( 76,  6.31) ( 77,  6.28) ( 78,  6.24) ( 79,  6.21) ( 80,  6.15) ( 81,  6.13) ( 82,  6.09) ( 83,  6.06) ( 84,  6.02) ( 85,  6.00) ( 86,  5.97) ( 87,  5.92) ( 88,  5.88) ( 89,  5.86) ( 90,  5.82) ( 91,  5.78) ( 92,  5.75) ( 93,  5.73) ( 94,  5.71) ( 95,  5.66) ( 96,  5.64) ( 97,  5.59) ( 98,  5.55) ( 99,  5.51) (100,  5.47) (101,  5.43) (102,  5.41) (103,  5.36) (104,  5.36) (105,  5.31) (106,  5.28) (107,  5.25) (108,  5.23) (109,  5.18) (110,  5.13) (111,  5.09) (112,  5.04) (113,  5.01) (114,  5.00) (115,  4.96) (116,  4.92) (117,  4.86) (118,  4.83) (119,  4.79) (120,  4.77) (121,  4.72) (122,  4.68) (123,  4.66) (124,  4.63) (125,  4.60) (126,  4.56) (127,  4.52) (128,  4.50) (129,  4.45) (130,  4.43) (131,  4.40) (132,  4.36) (133,  4.34) (134,  4.30) (135,  4.27) (136,  4.24) (137,  4.22) (138,  4.18) (139,  4.16) (140,  4.12) (141,  4.09) (142,  4.06) (143,  4.03) (144,  4.01) (145,  3.95) (146,  3.91) (147,  3.89) (148,  3.85) (149,  3.81) (150,  3.77) (151,  3.75) (152,  3.71) (153,  3.66) (154,  3.62) (155,  3.59) (156,  3.55) (157,  3.51) (158,  3.47) (159,  3.43) (160,  3.39) (161,  3.37) (162,  3.29) (163,  3.27) (164,  3.23) (165,  3.19) (166,  3.13) (167,  3.08) (168,  3.03) (169,  2.99) (170,  2.94) (171,  2.89) (172,  2.84) (173,  2.79) (174,  2.76) (175,  2.72) (176,  2.68) (177,  2.65) (178,  2.61) (179,  2.58) (180,  2.51) (181,  2.54) (182,  2.56) };
\addlegendentry{Average for $\norm{\vec b_i^*}$}

  \addplot[black] coordinates {(  1, 5.453) (  2, 5.450) (  3, 5.449) (  4, 5.446) (  5, 5.442) (  6, 5.434) (  7, 5.430) (  8, 5.428) (  9, 5.424) ( 10, 5.416) ( 11, 5.411) ( 12, 5.407) ( 13, 5.402) ( 14, 5.397) ( 15, 5.392) ( 16, 5.388) ( 17, 5.385) ( 18, 5.383) ( 19, 5.380) ( 20, 5.375) ( 21, 5.366) ( 22, 5.358) ( 23, 5.355) ( 24, 5.352) ( 25, 5.350) ( 26, 5.345) ( 27, 5.341) ( 28, 5.336) ( 29, 5.332) ( 30, 5.327) ( 31, 5.322) ( 32, 5.317) ( 33, 5.312) ( 34, 5.307) ( 35, 5.305) ( 36, 5.299) ( 37, 5.296) ( 38, 5.290) ( 39, 5.285) ( 40, 5.279) ( 41, 5.276) ( 42, 5.273) ( 43, 5.267) ( 44, 5.261) ( 45, 5.255) ( 46, 5.252) ( 47, 5.248) ( 48, 5.241) ( 49, 5.237) ( 50, 5.233) ( 51, 5.230) ( 52, 5.222) ( 53, 5.217) ( 54, 5.209) ( 55, 5.206) ( 56, 5.204) ( 57, 5.197) ( 58, 5.190) ( 59, 5.182) ( 60, 5.175) ( 61, 5.166) ( 62, 5.157) ( 63, 5.151) ( 64, 5.144) ( 65, 5.139) ( 66, 5.132) ( 67, 5.123) ( 68, 5.117) ( 69, 5.111) ( 70, 5.108) ( 71, 5.105) ( 72, 5.099) ( 73, 5.087) ( 74, 5.082) ( 75, 5.078) ( 76, 5.074) ( 77, 5.063) ( 78, 5.057) ( 79, 5.052) ( 80, 5.041) ( 81, 5.026) ( 82, 5.021) ( 83, 5.013) ( 84, 5.001) ( 85, 4.996) ( 86, 4.988) ( 87, 4.970) ( 88, 4.963) ( 89, 4.956) ( 90, 4.949) ( 91, 4.941) ( 92, 4.937) ( 93, 4.929) ( 94, 4.925) ( 95, 4.915) ( 96, 4.909) ( 97, 4.898) ( 98, 4.887) ( 99, 4.875) (100, 4.860) (101, 4.846) (102, 4.830) (103, 4.824) (104, 4.815) (105, 4.806) (106, 4.796) (107, 4.791) (108, 4.780) (109, 4.759) (110, 4.750) (111, 4.741) (112, 4.729) (113, 4.714) (114, 4.699) (115, 4.685) (116, 4.680) (117, 4.668) (118, 4.659) (119, 4.651) (120, 4.641) (121, 4.628) (122, 4.619) (123, 4.605) (124, 4.590) (125, 4.577) (126, 4.567) (127, 4.558) (128, 4.545) (129, 4.537) (130, 4.525) (131, 4.506) (132, 4.489) (133, 4.480) (134, 4.471) (135, 4.459) (136, 4.443) (137, 4.424) (138, 4.412) (139, 4.404) (140, 4.392) (141, 4.374) (142, 4.363) (143, 4.342) (144, 4.316) (145, 4.291) (146, 4.268) (147, 4.242) (148, 4.221) (149, 4.198) (150, 4.174) (151, 4.128) (152, 4.088) (153, 4.073) (154, 4.041) (155, 4.024) (156, 4.006) (157, 3.972) (158, 3.952) (159, 3.929) (160, 3.896) (161, 3.875) (162, 3.797) (163, 3.744) (164, 3.702) (165, 3.675) (166, 3.643) (167, 3.592) (168, 3.552) (169, 3.515) (170, 3.455) (171, 3.411) (172, 3.367) (173, 3.313) (174, 3.246) (175, 3.188) (176, 3.054) (177, 2.936) (178, 2.866) (179, 2.704) (180, 2.464) (181, 2.141) (182, 1.682)};
\addlegendentry{Average for $\norm{\pi_i(\vec e,\vec s,1)}$}

\draw[dashed] (127,1) -- (127,820) node[pos = 0.06, right] {$d-\beta$};
\end{axis}
\end{tikzpicture}
\tikzset{external/export=false}

\scriptsize{

\fullcite{AC:AGVW17}

}
\end{frame}

\section{Solving SVP}
\label{sec:orga283f71}
\begin{frame}[label={sec:orgff93238}]{Solving SVP}
\begin{center}
\small{
\begin{tabular}{rrrrr}
\textbf{Cost Model} $\backslash$    \textbf{Scheme} & \textbf{Kyber} & \textbf{NewHope} & \textbf{NTRU HRSS} & \textbf{SNTRU'}\\
\hline
\rore \(0.292\,β\)\textsuperscript{\ref{orgec23666}} & 180 & 259 & 136 & 155\\
\robl \enumworstfit \textsuperscript{\ref{org5a2122d}} & 456 & 738 & 313 & 370\\
\robl \enumavgfit \textsuperscript{\ref{org0886df4}} & 248 & 416 & 165 & 200\\
\hline
\rore \(0.265\,\beta\)\textsuperscript{\ref{orgec23666}} & 163 & 235 & 123 & 140\\
\robl \qenumworstfit & 228 & 369 & 157 & 187\\
\end{tabular}
}
\end{center}

\begin{columns}[t]
\begin{column}{0.5\columnwidth}
{\color{LightRed} \textbf{Sieving} }


\begin{itemize}
\item Produce new, shorter vectors by considering sums and differences of existing vectors
\item \textbf{Time:} \(2^{\Theta(\beta)}\)
\item \textbf{Memory:} \(2^{\Theta(\beta)}\)
\end{itemize}
\end{column}

\begin{column}{0.5\columnwidth}
{\color{DarkBlue} \textbf{Enumeration} }

\begin{itemize}
\item Search through vectors smaller than a given bound: project down to 1-dim problem, lift to 2-dim problem …
\item \textbf{Time:} \(2^{\Theta(\beta \log \beta)}\)
\item \textbf{Memory:} \(\poly[\beta]\)
\end{itemize}
\end{column}
\end{columns}
\end{frame}

\begin{frame}[label={sec:org6aad1af}]{Enumeration Estimates}
The \(1/(2e)\) estimate extrapolates a dataset from \cite{PhD:Chen13}

\begin{tikzpicture}
    \begin{axis}[xmin=100,height=0.4\textwidth]
      \addplot table [x=d, y=Chen13, col sep=comma]{data/cn11-simulations.csv};
      \addlegendentry{simulation \cite{PhD:Chen13}};
      \addplot+ [domain=100:500, samples=250]{0.187*x*log2(x) -1.019*x + 16.1};
      \addlegendentry{\enumworstfit};
    \end{axis}
  \end{tikzpicture}
\end{frame}

\begin{frame}[label={sec:orgf655505}]{Extended Enumeration Simulation}
That estimate compared to our simulation

\begin{tikzpicture}
  \begin{axis}[xmin=100,height=0.4\textwidth]
    \addplot table [x=d, col sep=comma, y expr = log2(\thisrowno{2})]{data/fplll-simulations,qary.csv};
    \addlegendentry{FP(y)LLL simulation};
    \addplot+ [domain=100:500, samples=250]{0.187*x*log2(x) + -1.019*x + 16.1};
    \addlegendentry{\enumworstfit};
  \end{axis}
\end{tikzpicture}
\end{frame}

\begin{frame}[label={sec:orge1b2905}]{Enumeration Simulation vs Experiments}
Assuming 1 node \(\approx\) 100 cpu cycles:

\begin{tikzpicture}
  \begin{axis}[height=0.4\textwidth]
    \addplot table [x=d, col sep=comma, y expr = log2(\thisrowno{2} * 3.3 * 10.0^9/100.0)]{data/fplll-observations,qary,[one-tour-strombenzin.json].csv};
    \addlegendentry{FP(y)LLL: running time};
    \addplot table [x=d, col sep=comma, y expr = log2(\thisrowno{3}+1 )]{data/fplll-observations,qary,[one-tour-strombenzin.json].csv};
    \addlegendentry{FP(y)LLL: visited nodes};
    \addplot table [x=d, col sep=comma, y expr = log2(\thisrowno{2}), select coords between index={0}{97}]{data/fplll-simulations,qary.csv};
    \addlegendentry{FP(y)LLL simulation};
  \end{axis}
\end{tikzpicture}
\end{frame}

\begin{frame}[label={sec:orgac696a5}]{Enumeration Worst-Case Complexity}
\begin{center}
\small{
\begin{tabular}{rrrrr}
\textbf{Cost Model} $\backslash$    \textbf{Scheme} & \textbf{Kyber} & \textbf{NewHope} & \textbf{NTRU HRSS} & \textbf{SNTRU'}\\
\hline
\rogr \enumworstfit & 456 & 738 & 313 & 370\\
\enumavgfit & 248 & 416 & 165 & 200\\
\end{tabular}
}
\end{center}

\begin{quote}
“We obtain a new worst-case complexity upper bound, as well as the first worst-case complexity lower
bound, both of the order d of \(2^{O(d)} \cdot d^{\frac{d}{2e}}\) (up to polynomial factors) bit
operations, where \(d\) is the rank of the lattice.”\footnote{Full version of \fullcite{C:HanSte07}, available at \url{http://perso.ens-lyon.fr/damien.stehle/KANNAN\_EXTENDED.html}\label{org3d77782}}
\end{quote}
\end{frame}

\begin{frame}[label={sec:orgeecbcb0}]{Enumeration Heuristic Best-Case Complexity}
\begin{center}
\small{
\begin{tabular}{rrrrr}
\textbf{Cost Model} $\backslash$    \textbf{Scheme} & \textbf{Kyber} & \textbf{NewHope} & \textbf{NTRU HRSS} & \textbf{SNTRU'}\\
\hline
\enumworstfit & 456 & 738 & 313 & 370\\
\rogr \enumavgfit & 248 & 416 & 165 & 200\\
\end{tabular}
}
\end{center}

\begin{quote}
“Some authors favor the hypothesis that the average behaviour of an HKZ-reduced basis is rather a geometric decrease of the \(\|\vec{b}_i^{*}\|\)’s, i.e., roughly \(\|\vec{b}^*_i\| ≈ d^{\frac{i}{d}} \cdot \|\vec{b}_1\|\). With such a basis, solving SVP by Kannan’s algorithm would have a \(2^{O(d)} \cdot d^{\frac{d}{8}}\) complexity.”\textsuperscript{\ref{org3d77782}}
\end{quote}
\end{frame}

\begin{frame}[label={sec:orgb41408f}]{Enumeration Heuristic Best-Case Complexity}
\begin{center}
\small{
\begin{tabular}{rrrrr}
\textbf{Cost Model} $\backslash$    \textbf{Scheme} & \textbf{Kyber} & \textbf{NewHope} & \textbf{NTRU HRSS} & \textbf{SNTRU'}\\
\hline
\enumworstfit & 456 & 738 & 313 & 370\\
\rogr \enumavgfit & 248 & 416 & 165 & 200\\
\end{tabular}
}
\end{center}

\begin{quote}
“This suggests that, independently of the quality of the reduced basis, the complexity of enumeration will be at least \(d^\frac{d}{8}\) polynomial-time operations for many lattices.”\footfullcite{Nguyen10}
\phantom{foo}
\end{quote}
\end{frame}

\begin{frame}[label={sec:orgca758ad}]{\(1/8 = 0.125\) v \(1/(2e) \approx 0.184\)}
\begin{tikzpicture}
  \begin{axis}[xmin=-10, xmax=610, xlabel=\(i\),ylabel=\(\log_2 \|\vec{b}_i^*\|\),height=0.5\textwidth]
    \draw[fill=LightGreen!20!white,line width=0] (axis cs: 0,0) rectangle (axis cs: 200,12);
    \draw[fill=LightRed!20!white,line width=0] (axis cs: 400,0) rectangle (axis cs: 600,12);
    \addplot+[black] coordinates {
      (  0, 11.91) (  1, 11.89) (  2, 11.88) (  3, 11.86) (  4, 11.84) (  5, 11.82) (  6, 11.80)
      (  7, 11.79) (  8, 11.77) (  9, 11.75) ( 10, 11.73) ( 11, 11.71) ( 12, 11.69) ( 13, 11.68)
      ( 14, 11.66) ( 15, 11.64) ( 16, 11.62) ( 17, 11.60) ( 18, 11.59) ( 19, 11.57) ( 20, 11.55)
      ( 21, 11.53) ( 22, 11.51) ( 23, 11.50) ( 24, 11.48) ( 25, 11.46) ( 26, 11.44) ( 27, 11.42)
      ( 28, 11.40) ( 29, 11.39) ( 30, 11.37) ( 31, 11.35) ( 32, 11.33) ( 33, 11.31) ( 34, 11.30)
      ( 35, 11.28) ( 36, 11.26) ( 37, 11.24) ( 38, 11.22) ( 39, 11.21) ( 40, 11.19) ( 41, 11.17)
      ( 42, 11.15) ( 43, 11.13) ( 44, 11.11) ( 45, 11.10) ( 46, 11.08) ( 47, 11.06) ( 48, 11.04)
      ( 49, 11.02) ( 50, 11.01) ( 51, 10.99) ( 52, 10.97) ( 53, 10.95) ( 54, 10.93) ( 55, 10.92)
      ( 56, 10.90) ( 57, 10.88) ( 58, 10.86) ( 59, 10.84) ( 60, 10.83) ( 61, 10.81) ( 62, 10.79)
      ( 63, 10.77) ( 64, 10.75) ( 65, 10.74) ( 66, 10.72) ( 67, 10.70) ( 68, 10.68) ( 69, 10.66)
      ( 70, 10.64) ( 71, 10.63) ( 72, 10.61) ( 73, 10.59) ( 74, 10.57) ( 75, 10.55) ( 76, 10.54)
      ( 77, 10.52) ( 78, 10.50) ( 79, 10.48) ( 80, 10.46) ( 81, 10.45) ( 82, 10.43) ( 83, 10.41)
      ( 84, 10.39) ( 85, 10.37) ( 86, 10.36) ( 87, 10.34) ( 88, 10.32) ( 89, 10.30) ( 90, 10.28)
      ( 91, 10.27) ( 92, 10.25) ( 93, 10.23) ( 94, 10.21) ( 95, 10.19) ( 96, 10.18) ( 97, 10.16)
      ( 98, 10.14) ( 99, 10.12) (100, 10.10) (101, 10.09) (102, 10.07) (103, 10.05) (104, 10.03)
      (105, 10.01) (106, 10.00) (107,  9.98) (108,  9.96) (109,  9.94) (110,  9.92) (111,  9.91)
      (112,  9.89) (113,  9.87) (114,  9.85) (115,  9.84) (116,  9.82) (117,  9.80) (118,  9.78)
      (119,  9.76) (120,  9.75) (121,  9.73) (122,  9.71) (123,  9.69) (124,  9.67) (125,  9.66)
      (126,  9.64) (127,  9.62) (128,  9.60) (129,  9.58) (130,  9.57) (131,  9.55) (132,  9.53)
      (133,  9.51) (134,  9.49) (135,  9.48) (136,  9.46) (137,  9.44) (138,  9.42) (139,  9.40)
      (140,  9.39) (141,  9.37) (142,  9.35) (143,  9.33) (144,  9.31) (145,  9.30) (146,  9.28)
      (147,  9.26) (148,  9.24) (149,  9.22) (150,  9.21) (151,  9.19) (152,  9.17) (153,  9.15)
      (154,  9.13) (155,  9.12) (156,  9.10) (157,  9.08) (158,  9.06) (159,  9.04) (160,  9.03)
      (161,  9.01) (162,  8.99) (163,  8.97) (164,  8.95) (165,  8.93) (166,  8.92) (167,  8.90)
      (168,  8.88) (169,  8.86) (170,  8.84) (171,  8.83) (172,  8.81) (173,  8.79) (174,  8.77)
      (175,  8.75) (176,  8.73) (177,  8.72) (178,  8.70) (179,  8.68) (180,  8.66) (181,  8.64)
      (182,  8.62) (183,  8.61) (184,  8.59) (185,  8.57) (186,  8.55) (187,  8.53) (188,  8.51)
      (189,  8.50) (190,  8.48) (191,  8.46) (192,  8.44) (193,  8.42) (194,  8.40) (195,  8.38)
      (196,  8.37) (197,  8.35) (198,  8.33) (199,  8.31) (200,  8.29) (201,  8.27) (202,  8.25)
      (203,  8.24) (204,  8.22) (205,  8.20) (206,  8.18) (207,  8.16) (208,  8.14) (209,  8.12)
      (210,  8.11) (211,  8.09) (212,  8.07) (213,  8.05) (214,  8.03) (215,  8.01) (216,  8.00)
      (217,  7.98) (218,  7.96) (219,  7.94) (220,  7.92) (221,  7.90) (222,  7.89) (223,  7.87)
      (224,  7.85) (225,  7.83) (226,  7.81) (227,  7.80) (228,  7.78) (229,  7.76) (230,  7.74)
      (231,  7.72) (232,  7.71) (233,  7.69) (234,  7.67) (235,  7.65) (236,  7.63) (237,  7.62)
      (238,  7.60) (239,  7.58) (240,  7.56) (241,  7.55) (242,  7.53) (243,  7.51) (244,  7.49)
      (245,  7.48) (246,  7.46) (247,  7.44) (248,  7.42) (249,  7.40) (250,  7.39) (251,  7.37)
      (252,  7.35) (253,  7.33) (254,  7.32) (255,  7.30) (256,  7.28) (257,  7.26) (258,  7.25)
      (259,  7.23) (260,  7.21) (261,  7.19) (262,  7.18) (263,  7.16) (264,  7.14) (265,  7.12)
      (266,  7.11) (267,  7.09) (268,  7.07) (269,  7.05) (270,  7.04) (271,  7.02) (272,  7.00)
      (273,  6.98) (274,  6.97) (275,  6.95) (276,  6.93) (277,  6.91) (278,  6.90) (279,  6.88)
      (280,  6.86) (281,  6.84) (282,  6.83) (283,  6.81) (284,  6.79) (285,  6.77) (286,  6.76)
      (287,  6.74) (288,  6.72) (289,  6.70) (290,  6.69) (291,  6.67) (292,  6.65) (293,  6.63)
      (294,  6.62) (295,  6.60) (296,  6.58) (297,  6.56) (298,  6.55) (299,  6.53) (300,  6.51)
      (301,  6.49) (302,  6.47) (303,  6.46) (304,  6.44) (305,  6.42) (306,  6.40) (307,  6.39)
      (308,  6.37) (309,  6.35) (310,  6.33) (311,  6.32) (312,  6.30) (313,  6.28) (314,  6.26)
      (315,  6.24) (316,  6.23) (317,  6.21) (318,  6.19) (319,  6.17) (320,  6.15) (321,  6.14)
      (322,  6.12) (323,  6.10) (324,  6.08) (325,  6.06) (326,  6.05) (327,  6.03) (328,  6.01)
      (329,  5.99) (330,  5.97) (331,  5.95) (332,  5.94) (333,  5.92) (334,  5.90) (335,  5.88)
      (336,  5.86) (337,  5.84) (338,  5.83) (339,  5.81) (340,  5.79) (341,  5.77) (342,  5.75)
      (343,  5.73) (344,  5.71) (345,  5.69) (346,  5.68) (347,  5.66) (348,  5.64) (349,  5.62)
      (350,  5.60) (351,  5.58) (352,  5.56) (353,  5.54) (354,  5.52) (355,  5.50) (356,  5.48)
      (357,  5.46) (358,  5.44) (359,  5.42) (360,  5.41) (361,  5.39) (362,  5.37) (363,  5.35)
      (364,  5.33) (365,  5.31) (366,  5.29) (367,  5.27) (368,  5.25) (369,  5.23) (370,  5.21)
      (371,  5.19) (372,  5.16) (373,  5.14) (374,  5.12) (375,  5.10) (376,  5.08) (377,  5.06)
      (378,  5.04) (379,  5.02) (380,  5.00) (381,  4.98) (382,  4.96) (383,  4.93) (384,  4.91)
      (385,  4.89) (386,  4.87) (387,  4.85) (388,  4.82) (389,  4.80) (390,  4.78) (391,  4.76)
      (392,  4.73) (393,  4.71) (394,  4.69) (395,  4.67) (396,  4.64) (397,  4.62) (398,  4.60)
      (399,  4.57) (400,  4.55) (401,  4.54) (402,  4.53) (403,  4.51) (404,  4.50) (405,  4.49)
      (406,  4.47) (407,  4.46) (408,  4.45) (409,  4.44) (410,  4.42) (411,  4.41) (412,  4.40)
      (413,  4.38) (414,  4.37) (415,  4.36) (416,  4.34) (417,  4.33) (418,  4.32) (419,  4.30)
      (420,  4.29) (421,  4.28) (422,  4.26) (423,  4.25) (424,  4.24) (425,  4.22) (426,  4.21)
      (427,  4.19) (428,  4.18) (429,  4.17) (430,  4.15) (431,  4.14) (432,  4.12) (433,  4.11)
      (434,  4.10) (435,  4.08) (436,  4.07) (437,  4.05) (438,  4.04) (439,  4.02) (440,  4.01)
      (441,  3.99) (442,  3.98) (443,  3.96) (444,  3.95) (445,  3.94) (446,  3.92) (447,  3.90)
      (448,  3.89) (449,  3.87) (450,  3.86) (451,  3.84) (452,  3.83) (453,  3.81) (454,  3.80)
      (455,  3.78) (456,  3.77) (457,  3.75) (458,  3.73) (459,  3.72) (460,  3.70) (461,  3.69)
      (462,  3.67) (463,  3.65) (464,  3.64) (465,  3.62) (466,  3.60) (467,  3.59) (468,  3.57)
      (469,  3.55) (470,  3.54) (471,  3.52) (472,  3.50) (473,  3.49) (474,  3.47) (475,  3.45)
      (476,  3.43) (477,  3.42) (478,  3.40) (479,  3.38) (480,  3.36) (481,  3.35) (482,  3.33)
      (483,  3.31) (484,  3.29) (485,  3.27) (486,  3.25) (487,  3.24) (488,  3.22) (489,  3.20)
      (490,  3.18) (491,  3.16) (492,  3.14) (493,  3.12) (494,  3.10) (495,  3.08) (496,  3.06)
      (497,  3.04) (498,  3.02) (499,  3.00) (500,  2.98) (501,  2.96) (502,  2.94) (503,  2.92)
      (504,  2.90) (505,  2.88) (506,  2.86) (507,  2.84) (508,  2.82) (509,  2.80) (510,  2.78)
      (511,  2.75) (512,  2.73) (513,  2.71) (514,  2.69) (515,  2.67) (516,  2.64) (517,  2.62)
      (518,  2.60) (519,  2.58) (520,  2.55) (521,  2.53) (522,  2.51) (523,  2.48) (524,  2.46)
      (525,  2.43) (526,  2.41) (527,  2.39) (528,  2.36) (529,  2.34) (530,  2.31) (531,  2.29)
      (532,  2.26) (533,  2.23) (534,  2.21) (535,  2.18) (536,  2.16) (537,  2.13) (538,  2.10)
      (539,  2.08) (540,  2.05) (541,  2.02) (542,  1.99) (543,  1.96) (544,  1.94) (545,  1.91)
      (546,  1.88) (547,  1.85) (548,  1.82) (549,  1.79) (550,  1.76) (551,  1.73) (552,  1.70)
      (553,  1.67) (554,  1.63) (555,  1.61) (556,  1.60) (557,  1.57) (558,  1.53) (559,  1.52)
      (560,  1.48) (561,  1.45) (562,  1.40) (563,  1.38) (564,  1.34) (565,  1.31) (566,  1.28)
      (567,  1.24) (568,  1.22) (569,  1.16) (570,  1.13) (571,  1.10) (572,  1.06) (573,  1.02)
      (574,  0.98) (575,  0.93) (576,  0.89) (577,  0.85) (578,  0.80) (579,  0.79) (580,  0.73)
      (581,  0.69) (582,  0.65) (583,  0.61) (584,  0.56) (585,  0.52) (586,  0.47) (587,  0.44)
      (588,  0.40) (589,  0.35) (590,  0.29) (591,  0.26) (592,  0.21) (593,  0.15) (594,  0.11)
      (595,  0.07) (596,  0.01) (597, -0.04) (598, -0.06) (599, -0.06) 
    };
  \end{axis}
\end{tikzpicture}
\end{frame}

\begin{frame}[label={sec:orgafe03ae}]{Why we can’t have Nice Things}
\begin{enumerate}
\item We run enumeration many times each succeeding with low probability of success and re-randomise in between: this destroys the nice GSA-line shape
\begin{itemize}
\item Thus, before enumerating a local block, we run some local preprocessing with some block size \(\beta' < \beta\)
\end{itemize}
\item In the sandpile model,\footfullcite{C:HanPujSte11} as the algorithm proceeds through the indices \(i\), a “bump” accumulates from index \(i + 1\) onward.
\end{enumerate}
\end{frame}

\begin{frame}[label={sec:orge96c307}]{Idea: Overshoot Preprocessing (WIP)}
\begin{tikzpicture}
  \begin{axis}[xmin=-10, xmax=610, xlabel=,ylabel=\(\log_2 \|\vec{b}_i^*\|\),height=0.5\textwidth]
    \draw[fill=black!20!white,line width=0] (axis cs: 100,0) rectangle (axis cs: 400,12);
    \draw[fill=LightGreen!20!white,line width=0] (axis cs: 100,0) rectangle (axis cs: 300,12);
    \addplot+[black] coordinates {
      (  0, 11.91) (  1, 11.89) (  2, 11.88) (  3, 11.86) (  4, 11.84) (  5, 11.82) (  6, 11.80)
      (  7, 11.79) (  8, 11.77) (  9, 11.75) ( 10, 11.73) ( 11, 11.71) ( 12, 11.69) ( 13, 11.68)
      ( 14, 11.66) ( 15, 11.64) ( 16, 11.62) ( 17, 11.60) ( 18, 11.59) ( 19, 11.57) ( 20, 11.55)
      ( 21, 11.53) ( 22, 11.51) ( 23, 11.50) ( 24, 11.48) ( 25, 11.46) ( 26, 11.44) ( 27, 11.42)
      ( 28, 11.40) ( 29, 11.39) ( 30, 11.37) ( 31, 11.35) ( 32, 11.33) ( 33, 11.31) ( 34, 11.30)
      ( 35, 11.28) ( 36, 11.26) ( 37, 11.24) ( 38, 11.22) ( 39, 11.21) ( 40, 11.19) ( 41, 11.17)
      ( 42, 11.15) ( 43, 11.13) ( 44, 11.11) ( 45, 11.10) ( 46, 11.08) ( 47, 11.06) ( 48, 11.04)
      ( 49, 11.02) ( 50, 11.01) ( 51, 10.99) ( 52, 10.97) ( 53, 10.95) ( 54, 10.93) ( 55, 10.92)
      ( 56, 10.90) ( 57, 10.88) ( 58, 10.86) ( 59, 10.84) ( 60, 10.83) ( 61, 10.81) ( 62, 10.79)
      ( 63, 10.77) ( 64, 10.75) ( 65, 10.74) ( 66, 10.72) ( 67, 10.70) ( 68, 10.68) ( 69, 10.66)
      ( 70, 10.64) ( 71, 10.63) ( 72, 10.61) ( 73, 10.59) ( 74, 10.57) ( 75, 10.55) ( 76, 10.54)
      ( 77, 10.52) ( 78, 10.50) ( 79, 10.48) ( 80, 10.46) ( 81, 10.45) ( 82, 10.43) ( 83, 10.41)
      ( 84, 10.39) ( 85, 10.37) ( 86, 10.36) ( 87, 10.34) ( 88, 10.32) ( 89, 10.30) ( 90, 10.28)
      ( 91, 10.27) ( 92, 10.25) ( 93, 10.23) ( 94, 10.21) ( 95, 10.19) ( 96, 10.18) ( 97, 10.16)
      ( 98, 10.14) ( 99, 10.12) (100, 10.10) (101, 10.09) (102, 10.07) (103, 10.05) (104, 10.03)
      (105, 10.01) (106, 10.00) (107,  9.98) (108,  9.96) (109,  9.94) (110,  9.92) (111,  9.91)
      (112,  9.89) (113,  9.87) (114,  9.85) (115,  9.84) (116,  9.82) (117,  9.80) (118,  9.78)
      (119,  9.76) (120,  9.75) (121,  9.73) (122,  9.71) (123,  9.69) (124,  9.67) (125,  9.66)
      (126,  9.64) (127,  9.62) (128,  9.60) (129,  9.58) (130,  9.57) (131,  9.55) (132,  9.53)
      (133,  9.51) (134,  9.49) (135,  9.48) (136,  9.46) (137,  9.44) (138,  9.42) (139,  9.40)
      (140,  9.39) (141,  9.37) (142,  9.35) (143,  9.33) (144,  9.31) (145,  9.30) (146,  9.28)
      (147,  9.26) (148,  9.24) (149,  9.22) (150,  9.21) (151,  9.19) (152,  9.17) (153,  9.15)
      (154,  9.13) (155,  9.12) (156,  9.10) (157,  9.08) (158,  9.06) (159,  9.04) (160,  9.03)
      (161,  9.01) (162,  8.99) (163,  8.97) (164,  8.95) (165,  8.93) (166,  8.92) (167,  8.90)
      (168,  8.88) (169,  8.86) (170,  8.84) (171,  8.83) (172,  8.81) (173,  8.79) (174,  8.77)
      (175,  8.75) (176,  8.73) (177,  8.72) (178,  8.70) (179,  8.68) (180,  8.66) (181,  8.64)
      (182,  8.62) (183,  8.61) (184,  8.59) (185,  8.57) (186,  8.55) (187,  8.53) (188,  8.51)
      (189,  8.50) (190,  8.48) (191,  8.46) (192,  8.44) (193,  8.42) (194,  8.40) (195,  8.38)
      (196,  8.37) (197,  8.35) (198,  8.33) (199,  8.31) (200,  8.29) (201,  8.27) (202,  8.25)
      (203,  8.24) (204,  8.22) (205,  8.20) (206,  8.18) (207,  8.16) (208,  8.14) (209,  8.12)
      (210,  8.11) (211,  8.09) (212,  8.07) (213,  8.05) (214,  8.03) (215,  8.01) (216,  8.00)
      (217,  7.98) (218,  7.96) (219,  7.94) (220,  7.92) (221,  7.90) (222,  7.89) (223,  7.87)
      (224,  7.85) (225,  7.83) (226,  7.81) (227,  7.80) (228,  7.78) (229,  7.76) (230,  7.74)
      (231,  7.72) (232,  7.71) (233,  7.69) (234,  7.67) (235,  7.65) (236,  7.63) (237,  7.62)
      (238,  7.60) (239,  7.58) (240,  7.56) (241,  7.55) (242,  7.53) (243,  7.51) (244,  7.49)
      (245,  7.48) (246,  7.46) (247,  7.44) (248,  7.42) (249,  7.40) (250,  7.39) (251,  7.37)
      (252,  7.35) (253,  7.33) (254,  7.32) (255,  7.30) (256,  7.28) (257,  7.26) (258,  7.25)
      (259,  7.23) (260,  7.21) (261,  7.19) (262,  7.18) (263,  7.16) (264,  7.14) (265,  7.12)
      (266,  7.11) (267,  7.09) (268,  7.07) (269,  7.05) (270,  7.04) (271,  7.02) (272,  7.00)
      (273,  6.98) (274,  6.97) (275,  6.95) (276,  6.93) (277,  6.91) (278,  6.90) (279,  6.88)
      (280,  6.86) (281,  6.84) (282,  6.83) (283,  6.81) (284,  6.79) (285,  6.77) (286,  6.76)
      (287,  6.74) (288,  6.72) (289,  6.70) (290,  6.69) (291,  6.67) (292,  6.65) (293,  6.63)
      (294,  6.62) (295,  6.60) (296,  6.58) (297,  6.56) (298,  6.55) (299,  6.53) (300,  6.51)
      (301,  6.49) (302,  6.47) (303,  6.46) (304,  6.44) (305,  6.42) (306,  6.40) (307,  6.39)
      (308,  6.37) (309,  6.35) (310,  6.33) (311,  6.32) (312,  6.30) (313,  6.28) (314,  6.26)
      (315,  6.24) (316,  6.23) (317,  6.21) (318,  6.19) (319,  6.17) (320,  6.15) (321,  6.14)
      (322,  6.12) (323,  6.10) (324,  6.08) (325,  6.06) (326,  6.05) (327,  6.03) (328,  6.01)
      (329,  5.99) (330,  5.97) (331,  5.95) (332,  5.94) (333,  5.92) (334,  5.90) (335,  5.88)
      (336,  5.86) (337,  5.84) (338,  5.83) (339,  5.81) (340,  5.79) (341,  5.77) (342,  5.75)
      (343,  5.73) (344,  5.71) (345,  5.69) (346,  5.68) (347,  5.66) (348,  5.64) (349,  5.62)
      (350,  5.60) (351,  5.58) (352,  5.56) (353,  5.54) (354,  5.52) (355,  5.50) (356,  5.48)
      (357,  5.46) (358,  5.44) (359,  5.42) (360,  5.41) (361,  5.39) (362,  5.37) (363,  5.35)
      (364,  5.33) (365,  5.31) (366,  5.29) (367,  5.27) (368,  5.25) (369,  5.23) (370,  5.21)
      (371,  5.19) (372,  5.16) (373,  5.14) (374,  5.12) (375,  5.10) (376,  5.08) (377,  5.06)
      (378,  5.04) (379,  5.02) (380,  5.00) (381,  4.98) (382,  4.96) (383,  4.93) (384,  4.91)
      (385,  4.89) (386,  4.87) (387,  4.85) (388,  4.82) (389,  4.80) (390,  4.78) (391,  4.76)
      (392,  4.73) (393,  4.71) (394,  4.69) (395,  4.67) (396,  4.64) (397,  4.62) (398,  4.60)
      (399,  4.57) (400,  4.55) (401,  4.54) (402,  4.53) (403,  4.51) (404,  4.50) (405,  4.49)
      (406,  4.47) (407,  4.46) (408,  4.45) (409,  4.44) (410,  4.42) (411,  4.41) (412,  4.40)
      (413,  4.38) (414,  4.37) (415,  4.36) (416,  4.34) (417,  4.33) (418,  4.32) (419,  4.30)
      (420,  4.29) (421,  4.28) (422,  4.26) (423,  4.25) (424,  4.24) (425,  4.22) (426,  4.21)
      (427,  4.19) (428,  4.18) (429,  4.17) (430,  4.15) (431,  4.14) (432,  4.12) (433,  4.11)
      (434,  4.10) (435,  4.08) (436,  4.07) (437,  4.05) (438,  4.04) (439,  4.02) (440,  4.01)
      (441,  3.99) (442,  3.98) (443,  3.96) (444,  3.95) (445,  3.94) (446,  3.92) (447,  3.90)
      (448,  3.89) (449,  3.87) (450,  3.86) (451,  3.84) (452,  3.83) (453,  3.81) (454,  3.80)
      (455,  3.78) (456,  3.77) (457,  3.75) (458,  3.73) (459,  3.72) (460,  3.70) (461,  3.69)
      (462,  3.67) (463,  3.65) (464,  3.64) (465,  3.62) (466,  3.60) (467,  3.59) (468,  3.57)
      (469,  3.55) (470,  3.54) (471,  3.52) (472,  3.50) (473,  3.49) (474,  3.47) (475,  3.45)
      (476,  3.43) (477,  3.42) (478,  3.40) (479,  3.38) (480,  3.36) (481,  3.35) (482,  3.33)
      (483,  3.31) (484,  3.29) (485,  3.27) (486,  3.25) (487,  3.24) (488,  3.22) (489,  3.20)
      (490,  3.18) (491,  3.16) (492,  3.14) (493,  3.12) (494,  3.10) (495,  3.08) (496,  3.06)
      (497,  3.04) (498,  3.02) (499,  3.00) (500,  2.98) (501,  2.96) (502,  2.94) (503,  2.92)
      (504,  2.90) (505,  2.88) (506,  2.86) (507,  2.84) (508,  2.82) (509,  2.80) (510,  2.78)
      (511,  2.75) (512,  2.73) (513,  2.71) (514,  2.69) (515,  2.67) (516,  2.64) (517,  2.62)
      (518,  2.60) (519,  2.58) (520,  2.55) (521,  2.53) (522,  2.51) (523,  2.48) (524,  2.46)
      (525,  2.43) (526,  2.41) (527,  2.39) (528,  2.36) (529,  2.34) (530,  2.31) (531,  2.29)
      (532,  2.26) (533,  2.23) (534,  2.21) (535,  2.18) (536,  2.16) (537,  2.13) (538,  2.10)
      (539,  2.08) (540,  2.05) (541,  2.02) (542,  1.99) (543,  1.96) (544,  1.94) (545,  1.91)
      (546,  1.88) (547,  1.85) (548,  1.82) (549,  1.79) (550,  1.76) (551,  1.73) (552,  1.70)
      (553,  1.67) (554,  1.63) (555,  1.61) (556,  1.60) (557,  1.57) (558,  1.53) (559,  1.52)
      (560,  1.48) (561,  1.45) (562,  1.40) (563,  1.38) (564,  1.34) (565,  1.31) (566,  1.28)
      (567,  1.24) (568,  1.22) (569,  1.16) (570,  1.13) (571,  1.10) (572,  1.06) (573,  1.02)
      (574,  0.98) (575,  0.93) (576,  0.89) (577,  0.85) (578,  0.80) (579,  0.79) (580,  0.73)
      (581,  0.69) (582,  0.65) (583,  0.61) (584,  0.56) (585,  0.52) (586,  0.47) (587,  0.44)
      (588,  0.40) (589,  0.35) (590,  0.29) (591,  0.26) (592,  0.21) (593,  0.15) (594,  0.11)
      (595,  0.07) (596,  0.01) (597, -0.04) (598, -0.06) (599, -0.06) 
    };
  \end{axis}
\end{tikzpicture}
\begin{center}
Preprocessing in dimension \((1+c)\cdot\beta\) for enumeration in dimension \(\beta\).\footnote{Joint work with Shi Bai, Pierre-Alain Fouque, Paul Kirchner, Damien Stehlé and Weiqiang Wen}
\end{center}
\end{frame}

\begin{frame}[label={sec:orgdb9100e}]{Performance (WIP)}
\tikzset{external/export=true}
\tikzsetnextfilename{procrastinating-performance}
\tikzpicturedependsonfile{data/fplll-simulations,qary.csv}
\tikzpicturedependsonfile{data/fplll-block-simulations,qary,0.25.csv}
\begin{tikzpicture}
  \begin{axis}[xlabel=\(\beta\),height=0.5\textwidth]
    \addplot table [x=d, col sep=comma, y expr = log2(\thisrowno{2}), select coords between index={0}{500}]{data/fplll-simulations,qary.csv};
    \addlegendentry{\enumworstfit};

    \addplot table [x=d, col sep=comma, y expr = log2(\thisrowno{2}), , select coords between index={0}{500}]{data/fplll-block-simulations,qary,0.25.csv};
    \addlegendentry{simulation};

    \addplot+ [domain=20:500, samples=100]{0.125*x*log2(x) + -0.545*x + 10.0};
    \addlegendentry{\(0.125\,\beta\,\log_{2}(\beta) - 0.545\beta + 10.0\), for \(c=1/4\)};
  \end{axis}
\end{tikzpicture}
\tikzset{external/export=false}
\end{frame}

\begin{frame}[label={sec:org9136bdf}]{Sieving vs Enumeration}
\begin{center}
\small{
\begin{tabular}{rrrrr}
\textbf{Cost Model} $\backslash$    \textbf{Scheme} & \textbf{Kyber} & \textbf{NewHope} & \textbf{NTRU HRSS} & \textbf{SNTRU'}\\
\hline
\rore \(0.292\,β\) & 180 & 259 & 136 & 155\\
\robl \enumworstfit & 456 & 738 & 313 & 370\\
\robl \enumavgfit & 248 & 416 & 165 & 200\\
\end{tabular}
}
\end{center}

\begin{alertblock}{Crossover}
Sieving is asymptotically faster than enumeration, but does it beat enumeration in practical or cryptographic dimensions?
\end{alertblock}
\end{frame}

\begin{frame}[label={sec:org6972b9e}]{Sieving: Key Ideas I}
\tikzset{external/export=true}
\tikzsetnextfilename{sieving-idea-1}
\centering
\begin{tikzpicture}[scale=0.7, every node/.style={scale=0.7}]
  \coordinate (Origin)   at (0,0);

  \clip (-9,-5) rectangle (9,5);
  \pgftransformcm{1}{0.6}{0.7}{1}{\pgfpoint{0cm}{0cm}}

  \coordinate (Bone) at (0,2);
  \coordinate (Btwo) at (2,-2);

  \foreach \x in {-10,-9,...,10}{%
    \foreach \y in {-10,-9,...,10}{%
      \node[draw,circle,inner sep=1pt,fill] at (2*\x,2*\y) {};
    }
  }
  % \draw [ultra thick,-latex,red] (Origin)
  % -- (Bone) node [above left] {$b_1$};
  % \draw [ultra thick,-latex,red] (Origin)
  % -- (Btwo) node [below right] {$b_2$};

  \foreach \x in {-5,-3,2,4,5}{%
    \foreach \y in {-6,-4,1,4,5}{%
      \draw [ultra thick,-latex,DarkBlue] (Origin)
      -- (2*\x,2*\y) {};
    }
  }

  \node [left] at (Origin) {$\mathcal{O}$};
\end{tikzpicture}
\tikzset{external/export=false}
\end{frame}

\begin{frame}[label={sec:org7ca8886}]{Sieving: Key Ideas II}
\tikzset{external/export=true}
\tikzsetnextfilename{sieving-idea-2}
\centering
\begin{tikzpicture}[scale=0.7, every node/.style={scale=0.7}]
  \coordinate (Origin)   at (0,0);

  \clip (-9,-5) rectangle (9,5);
  \pgftransformcm{1}{0.6}{0.7}{1}{\pgfpoint{0cm}{0cm}}

  \coordinate (Bone) at (0,2);
  \coordinate (Btwo) at (2,-2);

  \foreach \x in {-10,-9,...,10}{%
    \foreach \y in {-10,-9,...,10}{%
      \node[draw,circle,inner sep=1pt,fill] at (2*\x,2*\y) {};
    }
  }

  \foreach \x in {-5,-3,2,4,5}{%
    \foreach \y in {-6,-4,1,4,5}{%
      \draw [ultra thick,-latex,DarkBlue] (Origin)
      -- (2*\x,2*\y) {};
    }
  }

  \draw [ultra thick,-latex,LightRed] (Origin) -- (-10,10) {};
  \draw [ultra thick,-latex,LightRed] (Origin) -- (-10,8) {};

  \node [left] at (Origin) {$\mathcal{O}$};
\end{tikzpicture}
\tikzset{external/export=false}
\end{frame}

\begin{frame}[label={sec:orge4adefe}]{Sieving: Key Ideas III}
\tikzset{external/export=true}
\tikzsetnextfilename{sieving-idea-3}
\centering
\begin{tikzpicture}[scale=0.7, every node/.style={scale=0.7}]
  \coordinate (Origin)   at (0,0);

  \clip (-9,-5) rectangle (9,5);
  \pgftransformcm{1}{0.6}{0.7}{1}{\pgfpoint{0cm}{0cm}}

  \coordinate (Bone) at (0,2);
  \coordinate (Btwo) at (2,-2);

  \foreach \x in {-10,-9,...,10}{%
    \foreach \y in {-10,-9,...,10}{%
      \node[draw,circle,inner sep=1pt,fill] at (2*\x,2*\y) {};
    }
  }

  \foreach \x in {-5,-3,2,4,5}{%
    \foreach \y in {-6,-4,1,4,5}{%
      \draw [ultra thick,-latex,DarkBlue] (Origin)
      -- (2*\x,2*\y) {};
    }
  }

  \draw [ultra thick,-latex,LightRed] (Origin) -- (-10,10) {};
  \draw [ultra thick,-latex,LightRed] (Origin) -- (-10,8) {};
  \draw [ultra thick,-latex,LightBrown] (Origin) -- (0,-2) {};

  \node [left] at (Origin) {$\mathcal{O}$};
\end{tikzpicture}
\tikzset{external/export=false}
\end{frame}

\begin{frame}[label={sec:org41336bb}]{Sieving: Popcount I}
\tikzset{external/export=true}
\tikzsetnextfilename{sieving-popcount-1}
\centering
\begin{tikzpicture}[scale=0.7, every node/.style={scale=0.7}]
  \coordinate (Origin)   at (0,0);

  \clip (-9,-5) rectangle (9,5);
  \pgftransformcm{1}{0.6}{0.7}{1}{\pgfpoint{0cm}{0cm}}

  \coordinate (Bone) at (0,2);
  \coordinate (Btwo) at (2,-2);

  \foreach \x in {-10,-9,...,10}{%
    \foreach \y in {-10,-9,...,10}{%
      \node[draw,circle,inner sep=1pt,fill] at (2*\x,2*\y) {};
    }
  }

  \foreach \x in {2,4,5}{%
    \foreach \y in {-6,-4,1,4,5}{%
      \draw [ultra thick,-latex,lightgray] (Origin)
      -- (2*\x,2*\y) {};
    }
  }


  \foreach \x in {-5,-3}{%
    \foreach \y in {-6,-4,1,4,5}{%
      \draw [ultra thick,-latex,DarkBlue] (Origin)
      -- (2*\x,2*\y) {};
    }
  }
  \draw[ultra thick] (0,-10) -- (0,10);

  \node [left] at (Origin) {$\mathcal{O}$};
\end{tikzpicture}
\tikzset{external/export=false}
\end{frame}

\begin{frame}[label={sec:orgbeaddd1}]{Sieving: Popcount II}
\begin{itemize}
\item For a given plane, denote a vector being on the “left” as 0, being on the “right” as 1.
\item This defines a 1-bit locality sensitive hash (LSH) function.
\item Consider many such hash functions and concatenate their output.
\item Two vectors are close if they agree on many bits of their hashes
\end{itemize}

\begin{block}{Comparison Operation}
XOR hash values and compute Hamming weight (“popcount”).
\end{block}
\end{frame}

\begin{frame}[label={sec:orgfcfdca8}]{Sieving: Buckets}
\tikzset{external/export=true}
\tikzsetnextfilename{sieving-buckets}
\centering
\begin{tikzpicture}[scale=0.7, every node/.style={scale=0.7}]
  \coordinate (Origin)   at (0,0);

  \clip (-9,-5) rectangle (9,5);
  \pgftransformcm{1}{0.6}{0.7}{1}{\pgfpoint{0cm}{0cm}}

  \coordinate (Bone) at (0,2);
  \coordinate (Btwo) at (2,-2);

  \foreach \x in {-10,-9,...,10}{%
    \foreach \y in {-10,-9,...,10}{%
      \node[draw,circle,inner sep=1pt,fill] at (2*\x,2*\y) {};
    }
  }

\draw[fill=LightGreen!20!white] (-20,10) -- (0,0) -- (20,10) -- (-100,100) {};

\foreach \x in {-5,-3,2,4,5}{%
    \foreach \y in {-6,-4, 1,4,5}{%
      \draw [ultra thick,-latex,DarkBlue] (Origin)
      -- (2*\x,2*\y) {};
    }
 }

  \node [left] at (Origin) {$\mathcal{O}$};
\end{tikzpicture}
\tikzset{external/export=false}
\end{frame}

\begin{frame}[label={sec:org47ea5ac}]{Sieving: Some Algorithms}
\begin{description}
\item[{Gauss}] Sample \((4/3)^{\beta/2 + o(\beta)}\) vectors, compare them pairwise if they reduce to something shorter. \textbf{Cost}: \((4/3)^{\beta + o(\beta)} \approx 2^{0.41\,\beta + o(\beta)}\).\footfullcite{SODA:MicVou10}
\item[{BGJ}] Split search space into “buckets”. \textbf{Cost}: \(2^{0.311\,\beta + o(\beta)}\).\footfullcite{EPRINT:BecGamJou15}
\item[{BDGL}] Use codes to decide which bucket to consider. \textbf{Cost}: \(2^{0.292\,\beta + o(\beta)}\). \footfullcite{SODA:BDGL16}
\end{description}
\end{frame}

\begin{frame}[label={sec:org9d3b50a}]{Sieving: G6K}
G6K \footfullcite{EC:ADHKPS19} is a Python/C++ framework for experimenting with sieving algorithms (inside and outside BKZ)
\begin{itemize}
\item Does not take the “oracle” view but considers sieves as stateful machines.
\item Implements several sieve algorithms\footnote{Gauss, NV, BGJ1 (BGJ with one level of filtration)} (but not BDGL)
\item Applies recent tricks and adds new tricks for improving performance of sieving
\end{itemize}
\end{frame}

\begin{frame}[label={sec:org1f00c55}]{Sieving: SVP}
\begin{center}
\begin{tikzpicture}
    \begin{semilogyaxis}[ylabel=seconds, xlabel=\(\beta\), legend style={fill=}, legend pos=north west, height=0.5\textwidth]
        \addplot+ [only marks] table [x=d, y=FPLLL, col sep=comma]{data/exact-svp.csv};
        \addlegendentry{BKZ + pruned enum (FPLLL)};
        \addplot+ [only marks] table [x=d, y=G6K, col sep=comma]{data/exact-svp.csv};
        \addlegendentry{G6K WorkOut};
    \end{semilogyaxis}
\end{tikzpicture}
Average time in seconds for solving exact SVP
\end{center}
\end{frame}

\begin{frame}[label={sec:org96f051c}]{Darmstadt HSVP\textsubscript{1.05} Challenges}
\begin{center}
\begin{tikzpicture}
  \begin{axis}[xlabel=\(\beta\),ylabel=\(\log_2(\textnormal{cycles})\),height=0.5\textwidth]
    \addplot table [x=d, col sep=comma, y expr = log2(100*\thisrowno{2}),, select coords between index={0}{50} ]{data/fplll-simulations,svp-challenge.csv};
    \addlegendentry{HSVP\(_{1.05}\) non-parallel enum sim};

    \addplot table [x=d, col sep=comma, y expr = log2(100*\thisrowno{2}), select coords between index={70}{166}]{data/fplll-simulations,qary.csv};
    \addlegendentry{SVP non-parallel enum sim};

    \addplot+ [only marks] table [unbounded coords=discard,x=d, col sep=comma, y expr = %
    log2(\thisrowno{3}*3600*2*10.0^9)%
    ]{data/svp-challenge-observations.csv};
    \addlegendentry{HoF:FK15};

    \addplot+ [only marks] table [unbounded coords=discard,x=d, col sep=comma, y expr = %
    log2(\thisrowno{4}*3600*2*10.0^9)%
    ]{data/svp-challenge-observations.csv};
    \addlegendentry{HoF:KT17};

    \addplot+ [only marks] table [unbounded coords=discard,x=d, col sep=comma, y expr = %
    log2(\thisrowno{5}*3600*2*10.0^9)%
    ]{data/svp-challenge-observations.csv};
    \addlegendentry{G6K};

  \end{axis}
\end{tikzpicture}

Estimated and reported costs for solving Darmstadt SVP Challenges.
\end{center}
\end{frame}

\begin{frame}[label={sec:org73697b9},fragile]{Try it at Home}
 \lstset{language=sage,label= ,caption= ,captionpos=b,numbers=none}
\begin{lstlisting}
from fpylll import IntegerMatrix, GSO, LLL
from fpylll.tools.bkz_stats import dummy_tracer
from g6k import Siever
from g6k.algorithms.bkz import pump_n_jump_bkz_tour

A = LLL.reduction(IntegerMatrix.random(180, "qary", k=90, bits=20))
g6k = Siever(A)

for b in range(20, 60+1, 10):
    pump_n_jump_bkz_tour(g6k, dummy_tracer, b, pump_params={"down_sieve": True})
\end{lstlisting}

\begin{description}
\item[{\url{https://github.com/fplll/g6k}}] C++ kernel + Python frontend
\end{description}
\end{frame}

\begin{frame}[label={sec:orgfa1fdea}]{Practical Sieving: Open Questions}
\begin{itemize}
\item Parallelism in non-uniform memory access (NUMA) architectures
\item Practical performance of asymptotically faster sieves
\item Dedicated hardware …
\end{itemize}
\end{frame}

\begin{frame}[label={sec:orgc816741}]{Quantum Estimates}
\begin{center}
\small{
\begin{center}
\begin{tabular}{lrrrrr}
\textbf{Type} & \textbf{Cost Model} $\backslash$ \textbf{Scheme} & \textbf{Kyber} & \textbf{NewHope} & \textbf{NTRU HRSS} & \textbf{SNTRU'}\\
\hline
\rore classical & \(0.292\,β\) & 180 & 259 & 136 & 155\\
\rore quantum & \(0.265\,\beta\) & 163 & 235 & 123 & 140\\
\hline
\robl classical & \enumworstfit & 456 & 738 & 313 & 370\\
\robl quantum & \qenumworstfit & 228 & 369 & 157 & 187\\
\end{tabular}

\end{center}
}
\end{center}


\begin{description}
\item[{{\color{LightRed} \textbf{Sieving} }}] Given some vector \(\vec{w}\) and a list of vectors \(L\), apply Grover’s algorithm to find \(\{\vec{v} \in L \textnormal{ s.t. } \|\vec{v} \pm \vec{w}\| \leq \|\vec{w}\|\}\).\footfullcite{PhD:Laarhoven15}

\item[{{\color{DarkBlue} \textbf{Enumeration} }}] Apply Montanaro’s quantum backtracking algorithm for quadratic speed-up.\footfullcite{EPRINT:AonNguShe18}
\end{description}
\end{frame}

\begin{frame}[label={sec:orgb306cdc}]{Quantum Sieving}
\begin{itemize}
\item A quantum sieve needs list of \(2^{0.2075 \beta}\) vectors before pairwise search with Grover

\item Fast sieves use that the search is structured, Grover does unstructured search
\begin{itemize}
\item Quantum Gauss Sieve \[2^{(0.2075 + \frac{1}{2} 0.2075)\, \beta + o(\beta)} = 2^{0.311\, \beta + o(\beta)} \textnormal{ time},\qquad 2^{0.2075\, \beta + o(\beta)} \textnormal{ memory}\]
\item Classical BGJ Sieve \footfullcite{EPRINT:BecGamJou15} \[\phantom{2^{(0.2075 + \frac{1}{2} 0.2075)\, \beta + o(\beta)} = }2^{0.311\, \beta + o(\beta)}\textnormal{ time}, \qquad 2^{0.2075\, \beta + o(\beta)} \textnormal{ memory}\]
\end{itemize}
\item Asymptotically fastest sieves have small lists and thus less Grover speed-up potential
\end{itemize}
\end{frame}

\begin{frame}[label={sec:orgae3235d}]{Implementing Quantum Algorithms for SVP}
\begin{columns}[t]
\begin{column}{0.5\columnwidth}
\textbf{Sieving}

\begin{itemize}
\item Major operation is to check whether two vectors reduce to some smaller vector
\item Can be implemented using the XOR and popcount trick \(\Rightarrow\) the quantum circuit is relatively small.
\item Sieving requires exponentially large quantum accessible RAM (qRAM). Not clear that this can be built efficiently (due to error correction being required).
\end{itemize}
\end{column}

\begin{column}{0.5\columnwidth}
\textbf{Enumeration}

\begin{itemize}
\item Enumeration requires higher precision floating point arithmetic.
\item Quantum circuit for enumeration is likely to be larger than for sieving.
\item But no exponential qRAM.
\end{itemize}
\end{column}
\end{columns}
\end{frame}

\begin{frame}[label={sec:orga96e46c}]{Implementing Quantum Algorithms for SVP: Sieving}
\begin{center}
\includegraphics[width=0.5\textwidth]{./popcount-quantum-circuit.pdf}
\end{center}
\end{frame}

\begin{frame}[label={sec:orgf94f8d2}]{Implementing Quantum Algorithms for SVP: Sieving (Underestimates)}
\tikzset{external/export=true}
\tikzsetnextfilename{quantum-sieving-bdgl-ge}
\tikzpicturedependsonfile{data/cost-estimate-list_decoding-classical.csv}
\tikzpicturedependsonfile{data/cost-estimate-list_decoding-ge19.csv}
\begin{tikzpicture}
  \begin{axis}[ylabel={$\log_{2}(\#ops)$}, height=0.5\textwidth,xlabel=,]
    \addplot+[] table [x=d, col sep=comma, y=log_cost] {data/cost-estimate-list_decoding-classical.csv};
    \addlegendentry{BDGL (c: RAM)};

    \addplot+[] [domain=64:1024] {0.2924*x};
    \addlegendentry{\(0.2924\,d\)};

    \addplot+[] table [x=d, col sep=comma, y=log_cost] {data/cost-estimate-list_decoding-ge19.csv};
    \addlegendentry{BDGL (q: Gidney and Ekerå 2019)};

    \addplot+[] [domain=64:1024] {0.2652*x};
    \addlegendentry{\(0.2652\,d\)};
  \end{axis}
\end{tikzpicture}
\tikzset{external/export=false}


\scriptsize{

\fullcite{EPRINT:AGPS19}

}
\end{frame}


\begin{frame}[label={sec:org66217b2}]{Quantum Algorithms Open Problems}
\begin{itemize}
\item A quantum circuit for enumeration.
\item Better algorithms than best classical + Grover.
\end{itemize}
\end{frame}

\begin{frame}[label={sec:org523adb0}]{A Word on Lower Bounds}
\begin{center}
\small{
\begin{tabular}{rrrrr}
\textbf{Cost Model} $\backslash$ \textbf{Scheme} & \textbf{Kyber} & \textbf{NewHope} & \textbf{NTRU HRSS} & \textbf{SNTRU'}\\
\hline
\rore \(0.292\,β\) & 180 & 259 & 136 & 155\\
\enumworstfit & 456 & 738 & 313 & 370\\
\rore \enumavgfit & 248 & 416 & 165 & 200\\
\hline
\rore \(0.265\,\beta\) & 163 & 235 & 123 & 140\\
\rore \qenumworstfit & 228 & 369 & 157 & 187\\
\end{tabular}
}
\end{center}

\begin{columns}[t]
\begin{column}{0.5\columnwidth}
These estimates ignore:

\begin{itemize}
\item (large) polynomial factors hidden in \(o(\beta)\)
\item MAXDEPTH of quantum computers
\item cost of a Grover iteration
\item cost of memory (access)
\end{itemize}
\end{column}

\begin{column}{0.5\columnwidth}
Thus:

\begin{itemize}
\item cannot claim parameters need to be adjusted when these estimates are lowered
\item careful about conclusions drawn: some attacks don’t work here but work in practice
\end{itemize}
\end{column}
\end{columns}
\end{frame}


\begin{frame}[label={sec:org5a9728f}]{Alternative Approaches}
\begin{description}
\item[{BKW}] combinatorial technique, relatively efficient for small secrets
\item[{Arora-Ge}] use Gröbner bases, asymptotically efficient , but large constants in the exponent
\item[{Hybrid Attack}] combine combinatorial techniques with lattice reduction
\end{description}

\begin{alertblock}{Rule of Thumb}
Don’t need to worry about these unless secret is unusually small (e.g. ternary) and/or sparse.
\end{alertblock}
\end{frame}

\begin{frame}[label={sec:org9217ca0},standout]{Fin}
\begin{center}
\Huge \alert{Thank You}
\end{center}
\end{frame}

\begin{frame}[allowframebreaks]{References}
\renewcommand*{\bibfont}{\scriptsize}
\printbibliography[heading=none]
\end{frame}
\end{document}