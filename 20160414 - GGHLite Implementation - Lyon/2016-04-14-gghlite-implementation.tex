% Created 2016-04-14 Thu 12:27
\documentclass[presentation,smaller]{beamer}
\usepackage[utf8]{inputenc}
\usepackage{fixltx2e}
\usepackage{graphicx}
\usepackage{grffile}
\usepackage{longtable}
\usepackage{wrapfig}
\usepackage{rotating}
\usepackage[normalem]{ulem}
\usepackage{amsmath}
\usepackage{textcomp}
\usepackage{amssymb}
\usepackage{capt-of}
\usepackage{hyperref}
\usepackage{color}
\usepackage{listings}
%
% Config
%

\usepackage{amsmath,amsfonts,amssymb,amsthm}
\usepackage{ifxetex,ifluatex}
\usepackage{fixltx2e} % provides \textsubscript
\usepackage{xspace}
\usepackage{graphicx}
\usepackage{comment}
\usepackage{url}
\usepackage{relsize}
\usepackage{booktabs}
\usepackage{tabularx}
\renewcommand*{\UrlFont}{\ttfamily\smaller\relax}
\usepackage{subfigure}
\usepackage[normalem]{ulem}
%
% Pseudocode
%

\usepackage[linesnumbered,ruled]{algorithm2e}

%
% Source Code Listings
%

\usepackage{listings}
\lstdefinelanguage{Sage}[]{Python}{morekeywords={True,False,sage,cdef,cpdef,ctypedef,self},sensitive=true}
\lstset{frame=none,
          showtabs=False,
          showspaces=False,
          showstringspaces=False,
          commentstyle={\color{gray}},
          keywordstyle={\color{mLightBrown}\textbf},
          stringstyle ={\color{mDarkBrown}},
          frame=single,
          basicstyle=\tt\scriptsize\relax,
          backgroundcolor=\color{gray!190!black},
          inputencoding=utf8,
          literate={…}{{\ldots}}1,
          belowskip=0.0em,
          }
\usepackage{lstlinebgrd}
%
% Tikz
%

% from pgfplotsthemetol.sty
\definecolor{DarkPurple}{HTML}{332288}
\definecolor{DarkBlue}{HTML}{6699CC}
\definecolor{LightBlue}{HTML}{88CCEE}
\definecolor{DarkGreen}{HTML}{117733}
\definecolor{DarkRed}{HTML}{661100}
\definecolor{LightRed}{HTML}{CC6677}
\definecolor{LightPink}{HTML}{AA4466}
\definecolor{DarkPink}{HTML}{882255}
\definecolor{LightPurple}{HTML}{AA4499}

\definecolor{DarkBrown}{HTML}{604c38}
\definecolor{DarkTeal}{HTML}{23373b}
\definecolor{LightBrown}{HTML}{EB811B}
\definecolor{LightGreen}{HTML}{14B03D}

\usepackage{tikz,pgfplots}
\usetikzlibrary{calc}
\usetikzlibrary{arrows,automata}
\usetikzlibrary{positioning}
\usetikzlibrary{decorations.pathmorphing}
\usetikzlibrary{decorations.pathreplacing}
\usetikzlibrary{backgrounds, fit, shapes.symbols, chains, shapes.geometric, shapes.arrows}
\usetikzlibrary{crypto.symbols}
\pgfplotsset{compat=1.12}
\usetikzlibrary{graphs}
\tikzset{shadows=no}

%% Cache TiKZ pictures

\ifdefined\tikzcaching
\usetikzlibrary{external}
\tikzexternalize[prefix=build/]
\tikzset{external/up to date check=diff} % MD5 fails from within emacs
\fi

%
% SVG

\ifxetex
\newcommand{\executeiffilenewer}[3]{%
 {\immediate\write18{#3}} % hack
}
\else
\newcommand{\executeiffilenewer}[3]{%
 \ifnum\pdfstrcmp{\pdffilemoddate{#1}}%
 {\pdffilemoddate{#2}}>0%
 {\immediate\write18{#3}}\fi%
}
\fi

\newcommand{\includesvg}[2][1.0\textwidth]{%
 \executeiffilenewer{#1.svg}{#1.pdf}%
 {inkscape -z -D --file=#2.svg --export-pdf=#2.pdf --export-latex --export-area-page}%
 \def\svgwidth{#1} 
 \input{#2.pdf_tex}%
} 

%
% Metropolis Theme
% https://github.com/matze/mtheme.git
%

\usetheme{metropolis}
\metroset{color/block=fill}
\metroset{numbering=none}
\metroset{outer/progressbar=foot}
\metroset{titleformat=smallcaps}


%
% UTF-8 all the things
% 
\usepackage{unicodesymbols} % put this after m which loads fontspec

%
% BibLaTeX
%

\usepackage[
    backend=bibtex,
    style=alphabetic,
    citestyle=alphabetic,
]{biblatex}

\bibliography{../local.bib, ../master_refs.bib, ../TLS.bib}

\DeclareFieldFormat{title}{\alert{#1}}
\DeclareFieldFormat[book]{title}{\alert{#1}}
\DeclareFieldFormat[inproceedings]{title}{\alert{#1}}
\DeclareFieldFormat[article]{title}{\alert{#1}}
\DeclareFieldFormat[misc]{title}{\alert{#1}}

%
% Microtype
%

\IfFileExists{upquote.sty}{\usepackage{upquote}}{}
\IfFileExists{microtype.sty}{\usepackage{microtype}}{}


\setlength{\parindent}{0pt}
\setlength{\parskip}{6pt plus 2pt minus 1pt}
\setlength{\emergencystretch}{3em}  % prevent overfull lines
\setcounter{secnumdepth}{0}


% \AtBeginSection[] 
% {
% \begin{frame}<beamer>
% \frametitle{Outline}
% \tableofcontents[currentsection]
% \end{frame}
% }

% \AtBeginSubsection[] 
% {
% \begin{frame}<beamer>
% \frametitle{Outline}
% \tableofcontents[currentsubsection]
% \end{frame}
% }

%%% Local Variables:
%%% mode: latex
%%% End:
%
% GGH Stuff
%

\usepackage{algorithmic}

\DeclareMathOperator{\res}{res}
\DeclareMathOperator{\NTT}{NTT}

\newcommand{\bigO}[1]{\ensuremath{\mathcal{O}\left( {#1} \right)}}

\newcommand{\ideal}[1]{\ensuremath{\left(#1\right)}\xspace}
\newcommand{\mmod}[2][q]{\ensuremath{{\left[#2\right]}_{#1}}\xspace}

\newcommand{\norm}[1]{\ensuremath{\mathcal{N}({#1})}\xspace}
\newcommand{\cycf}{\ensuremath{X^n+1}\xspace}

\newcommand{\Z}[1][blank]{\ensuremath{
    \ifthenelse{\equal{#1}{blank}}
    {\mathbb{Z}}
    {\mathbb{Z}\left[#1\right]}\xspace}}

\newcommand{\Q}[1][blank]{\ensuremath{
        \ifthenelse{\equal{#1}{blank}}
        {\mathbb{Q}}
        {\mathbb{Q}\left[{#1}\right]}\xspace}}

\newcommand{\R}[1][blank]{\ensuremath{
        \ifthenelse{\equal{#1}{blank}}
        {\mathbb{R}}
        {\mathbb{R}\left[{#1}\right]}\xspace}}

\newcommand{\Zq}[1][blank]{\ensuremath{
        \ifthenelse{\equal{#1}{blank}}
        {\mathbb{Z}_q}
        {\mathbb{Z}_q[#1]}\xspace}}

 \newcommand{\bt}[2][i]{\ensuremath{\tilde{b}^{({#2})}_{#1}}\xspace}
\renewcommand{\b}[2][i]{\ensuremath{b^{({#2})}_{#1}}\xspace}
\renewcommand{\u}[1][]{\ensuremath{u_{#1}}\xspace}
  \newcommand{\e}[1][]{\ensuremath{e_{#1}}\xspace}
  \newcommand{\pzt}{\ensuremath{p_{zt}}\xspace}
  \newcommand{\swtch}{\ensuremath{s_{wtch}}\xspace}

  \newcommand{\sample}{\ensuremath{\gets_{\$}}\xspace}

%%% Local Variables:
%%% mode: latex
%%% TeX-master: "2016-04-14-gghlite-implementation"
%%% End:

\usetheme{default}
\author{Martin R. Albrecht}
\date{2016-04-14}
\title{Implementing Operations in Power-of-2 Cyclotomic Rings}
\subtitle{Lattice Meeting}
\hypersetup{
 pdfauthor={Martin R. Albrecht},
 pdftitle={Implementing Operations in Power-of-2 Cyclotomic Rings},
 pdfkeywords={},
 pdfsubject={},
 pdfcreator={Emacs 24.5.1 (Org mode 8.3.4)}, 
 pdflang={English}}
\begin{document}

\maketitle
\begin{frame}{Outline}
\tableofcontents
\end{frame}


\section{GGH-like Multilinear Maps}
\label{sec:orgheadline13}

\begin{frame}[label={sec:orgheadline1}]{GGH-like Multilinear Maps}
\begin{itemize}
\item In 2013, Garg, Gentry and Halevi \footfullcite{EC:GarGenHal13} proposed a construction, relying on ideal lattices, of a graded encoding scheme that approximates a cryptographic multilinear map.

\item Shortly after, this construction was improved by Langlois, Stéhle and Steinfeld \footfullcite{EC:LanSteSte14}.

\item Implementing GGH-like schemes naively would not allow instantiating it for non-trivial parameter sizes.
\end{itemize}
\end{frame}

\begin{frame}[label={sec:orgheadline2}]{Paper}
\fullcite{AC:ACLL15}
\end{frame}

\begin{frame}[label={sec:orgheadline3}]{Wait, Aren’t Those All Broken?}
\begin{center}
\includegraphics[width=.9\linewidth]{./kitten-1.jpg}

\url{http://malb.io/are-graded-encoding-schemes-broken-yet.html}
\end{center}
\end{frame}

\begin{frame}[label={sec:orgheadline4}]{Attacks I}
\begin{block}{Key Exchange}
\fullcite{EPRINT:HuJia15aa}
\end{block}

\begin{itemize}
\item Polynomial-time attack using low-level encodings of zero.\footnote{Sage implementation: \url{https://martinralbrecht.wordpress.com/2015/04/13/}}
\end{itemize}
\end{frame}

\begin{frame}[label={sec:orgheadline5}]{Attacks II}
\begin{block}{Attacks without Low-Level Encodings of Zero}
\fullcite{EPRINT:CheJeoLee16}

\fullcite{EPRINT:AlbBaiDuc16}
\end{block}

\begin{itemize}
\item Polynomial-time attack for large levels of multilinearity \(κ\) without low-level encodings of zero.
\item Subexponential attack for large levels of multilinearity \(κ\) without low-level encodings of zero without using the zero-testing parameter.
\end{itemize}
\end{frame}

\begin{frame}[label={sec:orgheadline6}]{Attacks III}
\begin{block}{Indistinguishability Obfuscation}
\fullcite{EPRINT:MilSahZha16}
\end{block}

\begin{itemize}
\item Polynomial-time attack on several iO constructions \footfullcite{FOCS:GGHRSW13} \footfullcite{EC:BGKPS14} \footfullcite{CCS:AGIS14}
\end{itemize}
\end{frame}

\begin{frame}[label={sec:orgheadline7}]{What good is an implementation of GGH?}
GGH-like graded encodings schemes might be broken, but designers of lattice-based schemes might still be tempted to write: “Sample \(g \hookleftarrow D_{R, σ}\) until \({\mathcal{I}}= (g)\) is a prime ideal” or “Sample \(f \hookleftarrow D_{(g)+c, σ}\).“
\end{frame}

\begin{frame}[label={sec:orgheadline8}]{GGHLite}
\begin{itemize}
\item We work in the \(m\)-th cyclotomic ring for \(m\) a power of two.
\item It has degree \(n=m/2\) and we consider the representation \(R ≃ \Z[X]/(x^n+1)\).
\item We also consider \(R_q ≃ \Z_q[X]/(x^n+1)\) and \(R_g ≃ \Z[X]/(x^n+1,g)\).
\end{itemize}
\end{frame}

\begin{frame}[label={sec:orgheadline9}]{GGHLite: Instance Generation}
\begin{itemize}
\item \textbf{Instance generation.} Given security parameter \(λ\) and multilinearity parameter \(κ\), determine scheme parameters \(n\), \(q\), \(σ\), \(σ'\), \(ℓ_{g^{-1}}\), \(ℓ_b\), \(ℓ\) as in GGHLite \footfullcite{EC:LanSteSte14}. Then proceed as follows:

\begin{itemize}
\item Sample \(g \hookleftarrow D_{R, σ}\) until \(\|g^{-1}\| ≤ ℓ_{g^{-1}}\) and \({\mathcal{I}}= (g)\) is a \alert{prime ideal}. Define encoding domain \(R_g = R/\ideal{g}\).

\item Sample \(z_i \hookleftarrow  U(R_q)\) for all \(0 < i ≤ \kappa\).

\item Sample \(h \hookleftarrow  D_{R, \sqrt{q}}\) s.t. \(h\) and \(g\) are \alert{co-prime} and define the zero-testing parameter \(\pzt = \mmod{\frac{h}{g} \prod_{i=1}^κ z_i}\).

\item Return public parameters \({\sf params}=(n,q,\ell)\) and \(\pzt\).
\end{itemize}
\end{itemize}
\end{frame}

\begin{frame}[label={sec:orgheadline10}]{GGHLite: Encoding}
\begin{itemize}
\item \textbf{Encode at level-$0$.} Compute a \alert{small representative} \(e' = \mmod[g]{e}\) and \alert{sample an element} \(e'' \hookleftarrow D_{e'+{\mathcal{I}},σ'}\). Output \(e''\).

\item \textbf{Encode in group.} Given parameters \({\sf params}\), \(z_i\) and a level-\(0\) encoding \(e \in R\), output \(\mmod{e/z_i}\).
\end{itemize}
\end{frame}

\begin{frame}[label={sec:orgheadline11}]{GGHLite: Arithmetic \& Zero-Testing}
\begin{itemize}
\item \textbf{Adding encodings.} Given encodings \(u_1 = \mmod{c_1/\left(\prod_{i \in S} z_i\right)}\) and \(u_2 = \mmod{c_2/\left(\prod_{i ∈ S} z_i\right)}\) with \(S ⊆ \{1,\dots, κ\}\):

\begin{itemize}
\item Return \(u = \mmod{u_1 + u_2}\), an encoding of \(\mmod{c_1+c_2}\) in the group \(S\).
\end{itemize}

\item \textbf{Multiplying encodings.} Let \(S_1 ⊂ [κ]\), \(S_2 ⊂ [ κ ]\) with \(S_1 ∩ S_2 = ∅\), given an encoding \(u_1 = \mmod{c_1/\left(\prod_{i \in S_1} z_i\right)}\) and an encoding \(u_2 = \mmod{c_2/\left(\prod_{i \in S_2} z_i\right)}\):

\begin{itemize}
\item Return \(u = \mmod{u_1 · u_2}\), an encoding of \(\mmod{c_1 · c_2}\) in \(S_1 ∪ S_2\).
\end{itemize}

\item \textbf{Zero testing.} Given parameters \({\sf params}\), a zero-testing parameter \(\pzt\), and an encoding \(u = \mmod{c/\left(\prod_{i=0}^{κ-1} z_i\right)}\) in the group \([\kappa]\), return \(1\) if \(\|\mmod{\pzt u}\|_{\infty} < q^{3/4}\) and \(0\) else.
\end{itemize}
\end{frame}

\begin{frame}[fragile,label={sec:orgheadline12}]{liboz ⊂ gghlite-flint}
 \lstset{language=Python,label= ,caption= ,captionpos=b,numbers=none}
\begin{lstlisting}
gghlite-flint
|- applications   # 0.9k benchmarks, high-level applications, …
|- dgs            # 1.2k discrete Gaussian sampling over the Integers
|- dgsl           # 0.7k discrete Gaussian sampling over lattices
|- flint          # 250k we rely on flint
|- gghlite        # 1.5k instance generation, zero testing …
|- oz             # 2.2k operations in Z[x]/(x^n+1)
|- tests          # 1.1k tests!
\end{lstlisting}

\begin{center}
\url{https://bitbucket.org/malb/gghlite-flint}\\
\url{https://bitbucket.org/malb/dgs}
\end{center}
\end{frame}

\section{Multiplication}
\label{sec:orgheadline19}

\begin{frame}[label={sec:orgheadline14}]{Options}
\begin{itemize}
\item Naive multiplication takes \(\bigO{n^2}\).

\item Asymptotically fast multiplication:
\begin{itemize}
\item Reduce to multiplication in \(\Z[X]\)
\item Schönehage-Strassen algorithm for multiplying large integers in \(\bigO{n \log n \log\log n}\).
\item This is the strategy implemented in FLINT.
\item FLINT has highly optimised implementation of the Schönehage-Strassen algorithm.
\end{itemize}

\item We can also achieve \(\bigO{n\log n}\) by the Number-Theoretic Transform.
\end{itemize}
\end{frame}

\begin{frame}[label={sec:orgheadline15}]{Negative Wrapped Convolution}
\begin{theorem}[Negative Wrapped Convolution]
Let \(ω_n\) be an \(n\)th root of unity in \(\Zq\) and \(φ^2 = \omega_n\). Let
\[a = \sum_{i=0}^{n-1} a_i X^i \textnormal{ and } b = \sum_{i=0}^{n-1} b_i X^i \in \Zq[X]/(\cycf).\]
Let \(c = a ⋅ b \in \Zq[X]/(\cycf)\) and let
\[\overline{a} = (a_0, φa_1, \dots, φ^{n-1}a_{n-1})\]
and define \(\overline{b}\) and \(\overline{c}\) analogously. Then
\[\overline{c} = 1/n \cdot \NTT_{ω_n}^{-1}(\NTT_{ω_n}(\overline{a})\odot \NTT_{ω_n}(\overline{b})).\]
\end{theorem}
\end{frame}

\begin{frame}[fragile,label={sec:orgheadline16}]{NTT over machine words: Bit Reversal}
 \lstset{language=C,label= ,caption= ,captionpos=b,numbers=none}
\begin{lstlisting}
void _nmod_vec_oz_ntt(mp_ptr rop, const mp_ptr op, const mp_ptr w,
                      const size_t n, const nmod_t q) {
  const size_t k = n_flog(n,2);

  mp_ptr a = _nmod_vec_init(n);

  for (unsigned int i = 0; i < n; i++) {
    unsigned int r;
    r = (bit_reverse_table_256[i & 0xff] << 16) | \
      (bit_reverse_table_256[(i >>  8) & 0xff] << 8) | \
      (bit_reverse_table_256[(i >> 16) & 0xff]);
    r >>= (24 - k);
    a[r] = op[i];
  }
\end{lstlisting}
\end{frame}

\begin{frame}[fragile,label={sec:orgheadline17}]{NTT over machine words: main loop}
 \lstset{language=C,label= ,caption= ,captionpos=b,numbers=none}
\begin{lstlisting}
  mp_ptr b = _nmod_vec_init(n);

  const double ninv = n_precompute_inverse(q.n);
  for(size_t i=0; i<k; i++) {
    const mp_limb_t tkm = ~(((1UL)<<(k-1-i)) - 1);
    for(size_t j=0; j<n/2; j++) {
      const size_t pij = j & tkm;
      mp_limb_t tmp = n_mulmod_precomp(a[2*j+1], w[pij], q.n, ninv);
      b[j]      = n_addmod(a[2*j], tmp, q.n);
      b[j+n/2]  = n_submod(a[2*j], tmp, q.n);
    }
    if(i!=k-1)
      _nmod_vec_set(a, b, n);
  }
  _nmod_vec_set(rop, b, n);
  _nmod_vec_clear(b);
  _nmod_vec_clear(a);
}
\end{lstlisting}
\end{frame}

\begin{frame}[label={sec:orgheadline18}]{Avoiding Conversion}
\begin{itemize}
\item If we do many operations in \(\Zq[X]/(\cycf)\) we can avoid repeated conversions between coefficient and “evaluation” representation \(\left(f(1),f(ω_n),\dots,f(ω_n^{n-1})\right)\)
\item We convert encodings to their evaluation representation once on creation
\item We convert back only when running extraction.
\item This reduces the amortised cost from \(\bigO{n \log n}\) to \(\bigO{n}\).
\end{itemize}
\end{frame}


\section{Computing Algebraic Norms}
\label{sec:orgheadline25}

\begin{frame}[label={sec:orgheadline20}]{Computing Algebraic Norms: Resultants}
\begin{itemize}
\item During instance generation we have to compute the norm of \(g\).
\item We can compute norms in \(\Z[X]/(\cycf)\) by observing that \[\norm{f} = \res(f,\cycf).\]
\end{itemize}

\note{:B\(_{\text{note}}\):
Let \(β=q(α)\) where \(q(X)=\sum^{n−1}_{i=0} b_i X^i\) and all \(b_i ∈ \QQ\). Represent \(p(X)=∏^{n}_{i=1}(X−α_i)\) where the \(α_i\) are \(α\) and its conjugates. Then \(β_i=q(α_i)\) are \(β\) and its conjugates. By definition of the norm, \[N(β)=∏_{i=1}^n β_i =∏_{i=1}^n q(α_i) = \res(p,q).\]}
\end{frame}

\begin{frame}[label={sec:orgheadline21}]{Multi-modular Resultants}
\begin{itemize}
\item The usual strategy for computing resultants over the integers is to use a multi-modular approach.
\item We compute resultants modulo many small primes \(q_i\) and then combine the results using the Chinese Remainder Theorem.
\item Resultants modulo a prime \(q_i\) can be computed in \(\bigO{M(n}\log n)\) operations where \(M(n)\) is the cost of one multiplication in \(\Z_{q_i}[X]/(\cycf)\).
\item Overall cost \(\bigO{n \log^2 n}\) without specialisation.
\end{itemize}
\end{frame}

\begin{frame}[label={sec:orgheadline22}]{Faster Resultants}
\begin{itemize}
\item \(\res(f,\cycf) \bmod q_i\) can be rewritten as \[\prod_{(\cycf)(x) = 0} f(x) \bmod q_i,\] i.e. as evaluating \(f\) on all roots of \(\cycf\).

\item Picking \(q_i\) such that \(q_i \equiv 1 \bmod 2n\) this can be accomplished using the NTT reducing the cost mod \(q_i\) to \(\bigO{M(n})\) saving a factor of \(\log n\).
\end{itemize}
\end{frame}

\begin{frame}[fragile,label={sec:orgheadline23}]{Source Code: Main Loop}
 \lstset{language=C,label= ,caption= ,captionpos=b,numbers=none}
\begin{lstlisting}
void _fmpz_poly_oz_ideal_norm(fmpz_t norm, const fmpz_poly_t f,
                              const long n) {
  …
#pragma omp parallel for
  for (i = 0; i<num_primes; i++) {
    nmod_t mod;
    nmod_init(&mod, parr[i]);

    const int id = omp_get_thread_num();
    /* reduce polynomials modulo p */
    _fmpz_vec_get_nmod_vec(a[id], F, n, mod);
    /* compute resultant over Z/pZ */
    rarr[i] = _nmod_vec_oz_resultant(a[id], n, mod);
    flint_cleanup();
  }
  …
}
\end{lstlisting}
\end{frame}

\begin{frame}[fragile,label={sec:orgheadline24}]{Souce Code: Resultants mod \(q ≡ 1 \bmod 2n\)}
 \lstset{language=C,label= ,caption= ,captionpos=b,numbers=none}
\begin{lstlisting}
mp_limb_t _nmod_vec_oz_resultant(const mp_ptr a, long n, nmod_t q) {
  const mp_limb_t w_ = _nmod_nth_root(2*n, q.n);
  mp_ptr w = _nmod_vec_init(2*n);
  mp_ptr t = _nmod_vec_init(2*n);

  _nmod_vec_oz_set_powers(w, 2*n, w_, q);
  _nmod_vec_oz_ntt(t, a, w, 2*n, q);

  mp_limb_t acc = 1;
  for(int i=1; i<2*n; i+=2)
    acc = n_mulmod2_preinv(acc, t[i], q.n, q.ninv);

  _nmod_vec_clear(w);
  _nmod_vec_clear(t);
  return acc;
}
\end{lstlisting}
\end{frame}


\section{Primality}
\label{sec:orgheadline34}

\begin{frame}[fragile,label={sec:orgheadline26}]{Checking Primality}
 To check if \((g)\) is prime, compute the norm and check if prime. This is a sufficient but not necessary condition.

\lstset{language=C,label= ,caption= ,captionpos=b,numbers=none}
\begin{lstlisting}
int fmpz_poly_oz_ideal_is_probaprime(const fmpz_poly_t f,
                                     const long n,
                                     const mp_limb_t *primes) {
    …
    fmpz_t norm;
    fmpz_init(norm);
    fmpz_poly_oz_ideal_norm(norm, f, n, 0);
    r = fmpz_is_probabprime(norm);
    fmpz_clear(norm);
    …
  return r;
}
\end{lstlisting}
\end{frame}

\begin{frame}[fragile,label={sec:orgheadline27}]{Ruling out Common Factors Quickly}
 Before computing resultants, check if \(\res(g,\cycf) ≡ 0 \bmod q_i\) for several "interesting" primes \(q_i\).

\lstset{language=C,label= ,caption= ,captionpos=b,numbers=none}
\begin{lstlisting}
int fmpz_poly_oz_ideal_is_probaprime(const fmpz_poly_t f,
                                     const long n,
                                     const mp_limb_t *primes) {
  int r = fmpz_poly_oz_ideal_not_prime_factors(f, n, primes);
  if (r) {
    fmpz_t norm;
    fmpz_init(norm);
    fmpz_poly_oz_ideal_norm(norm, f, n, 0);
    r = fmpz_is_probabprime(norm);
    fmpz_clear(norm);
  }
  return r;
}
\end{lstlisting}
\end{frame}

\begin{frame}[fragile,label={sec:orgheadline28}]{Ruling out Common Factors Quickly}
 \lstset{language=C,label= ,caption= ,captionpos=b,numbers=none}
\begin{lstlisting}
int fmpz_poly_oz_ideal_not_prime_factors(const fmpz_poly_t f, long n,
                                         const mp_limb_t *primes) {
  nmod_poly_t a[num_threads], b[num_threads];
  int r[num_threads];

  for(size_t i=0; i<k; i+=num_threads) {
    if (k-i < (unsigned long)num_threads)
      num_threads = k-i;
#pragma omp parallel for
    for (int j=0; j<num_threads; j++) {
      mp_limb_t p = primes[1+i+j];
      fmpz_poly_get_nmod_poly(a[j], f);
      r[j] = nmod_poly_oz_resultant(a[j], n);
    }
    for(int j=0; j<num_threads; j++)
      if (r[j] == 0)
        return r[0];
  }
  return r[0];
}
\end{lstlisting}
\end{frame}

\begin{frame}[fragile,label={sec:orgheadline29}]{Common Factors}
 These primes are \(2\) and then all primes up to some bound with \(q_i \equiv 1 \bmod n\) because these occur with good probability as factors.

\lstset{language=C,label= ,caption= ,captionpos=b,numbers=none}
\begin{lstlisting}
int _gghlite_nsmall_primes(const gghlite_params_t self) {
  /* we try about 1% small primes first, where 1% relates to the total
     number of primes needed for multi-modular result */
  const long n = self->n;
  int nsp = ceil((log2(_gghlite_sigma(n)) + log2(n)/2.0)
                 * n/100.0/(FLINT_BITS -1));
  if (nsp < 20)
    nsp = 20;
  return nsp;
}
\end{lstlisting}
\end{frame}

\begin{frame}[label={sec:orgheadline30}]{Timings}
\begin{center}
\begin{tabular}{rrr}
\hline
\(n\) & \(\log σ\) & wall time\\
\hline
1024 & 15.1 & 0.54s\\
2048 & 16.2 & 3.03s\\
4096 & 17.3 & 20.99s\\
32768 & 20.4 & 1834.99s\\
\hline
\end{tabular}

\end{center}

Average time of checking primality of a single \((g)\) on Intel Xeon CPU E5--2667 v2 3.30GHz with 256GB of RAM using 16 cores.
\end{frame}


\begin{frame}[label={sec:orgheadline31}]{Verifying Co-Primality}
\begin{itemize}
\item When re-randomisation elements are required, then it is necessary that they generate all of \(\ideal{g}\), i.e. \[(\b[1]{1},\b[2]{1}) = (g).\]
\item When \(\b[i]{1} = \bt[i]{1}·g\) for \(0 < i \leq 2\) then this is equivalent to \[(\bt[1]{1}) + (\bt[2]{1}) = \Z[X]/(\cycf).\]
\item We check the sufficient but not necessary condition \[\gcd(\res(\bt[1]{1},X^n+1),\, \res(\bt[2]{1},X^n+1)) = 1,\] i.e. if the respective ideal norms are co-prime.
\end{itemize}
\end{frame}

\begin{frame}[label={sec:orgheadline32}]{Avoiding Resultants}
\begin{itemize}
\item Perform this check for every candidate pair \((\bt[1]{1},\bt[2]{1})\).
\item Compute two resultants and their gcd: \alert{expensive}.
\item But \[\gcd(\res(\bt[1]{1}, X^n+1),\, \res(\bt[2]{1},X^n+1)) \neq 1\] when \[\res(\bt[1]{1},X^n+1) = 0 = \res(\bt[2]{1},X^n+1) \bmod q_i\] for any modulus \(q_i\).
\end{itemize}

➡ Check this condition for several “interesting” primes and resample if this condition holds.
\end{frame}

\begin{frame}[label={sec:orgheadline33}]{Avoiding Resultants}
\begin{itemize}
\item After having ruled out small common prime factors it is quite unlikely that the gcd of the norms is not equal to one.
\item With good probability we will perform this expensive step only once as a final verification.
\end{itemize}

\begin{block}{Improvement}
A possible strategy is to sample \(m>2\) re-randomisers \(\b[i]{1}\) and to apply bounds on the probability of \(m\) elements \(\bt[i]{1}\) sharing a prime factor after excluding small prime factors.
\end{block}
\end{frame}


\section{Inverting in \(\Q[X]/(\cycf)\)}
\label{sec:orgheadline42}

\begin{frame}[label={sec:orgheadline35}]{GGH Motivation}
Instance generation relies on inversion in \(\Q[X]/(\cycf)\).

\begin{enumerate}
\item when sampling \(g\) we have to check that the norm of its inverse is bounded by \(ℓ_g\).
\item To set up our discrete Gaussian samplers we need to run many inversions in an iterative process.
\end{enumerate}
\end{frame}

\begin{frame}[label={sec:orgheadline36}]{Inverting in \(\Q[X]/(\cycf)\)}
\begin{itemize}
\item The core idea \footfullcite{ICALP:BDMM98} is similar to the FFT, i.e. to reduce the inversion of \(f\) to the inversion of an element of degree \(n/2\).

\item Since \(n\) is even, \(f(X)\) is invertible modulo \(\cycf\) if and only if \(f(-X)\) is also invertible.

\item By setting \[F(X^2) = f(X)f(-X) \bmod{\cycf},\] the inverse \(f^{-1}(X)\) of \(f(X)\) satisfies
\end{itemize}
\[F(X^2)\,f^{-1}(X) = f(-X) \bmod{\cycf}.\]
\end{frame}

\begin{frame}[label={sec:orgheadline37}]{Inverting in \(\Q[X]/(\cycf)\)}
\begin{itemize}
\item Let \[f^{-1}(X) = g(X) = G_e(X^2) + X G_o(X^2)\] and \[f(-X) = F_e(X^2) + X F_o(X^2).\]

\item We obtain \[F(X^2)(G_e(X^2) + X G_o(X^2)) = F_e(X^2) + X F_o(X^2) \bmod{\cycf}\] or equivalently
\begin{eqnarray*}
F(X^2) G_e(X^2) &=& F_e(X^2) \pmod{\cycf},\\
F(X^2) G_o(X^2) &=& F_o(X^2) \pmod{\cycf}
\end{eqnarray*}

\item Invert \(f(X)\) by inverting \(F(X^2)\) and multiplying at degree \(n/2\).
\item Recursively call the inversion of \(F(Y)\) modulo \((X^{n/2}+1)\) by setting \(Y=X^2\).
\end{itemize}
\end{frame}

\begin{frame}[label={sec:orgheadline38}]{GGH Motivation Revisited}
Instance generation relies on inversion in \(\Q[X]/(\cycf)\).

\begin{enumerate}
\item when sampling \(g\) we have to check that the norm of its inverse is bounded by \(ℓ_g\).
\item To set up our discrete Gaussian samplers we need to run many inversions in an iterative process.
\end{enumerate}

\begin{block}<2->{Approximates Suffice}
In the first case we only need to estimate the size of \(g^{-1}\) and in the second case inversion is a subroutine of an approximation algorithm.
\end{block}
\end{frame}

\begin{frame}[fragile,label={sec:orgheadline39}]{Truncation}
 \lstset{language=C,label= ,caption= ,captionpos=b,numbers=none}
\begin{lstlisting}
void fmpq_poly_truncate_prec(fmpq_poly_t op, const mp_bitcnt_t prec) {
  mpq_t *tmp_q = (mpq_t*)calloc(fmpq_poly_length(op), sizeof(mpq_t));
  mpf_t tmp_f; mpf_init2(tmp_f, prec);

  for (int i=0; i<fmpq_poly_length(op); i ++) {
    mpq_init(tmp_q[i]);
    fmpq_poly_get_coeff_mpq(tmp_q[i], op, i);
    mpf_set_q(tmp_f, tmp_q[i]);
    mpq_set_f(tmp_q[i], tmp_f);
  }
  fmpq_poly_set_array_mpq(op, (const mpq_t*)tmp_q, fmpq_poly_length(op));
  …
}
\end{lstlisting}

Calling \texttt{fmpq\_poly\_set\_array\_mpq} instead of setting each coefficient one-by-one avoids repeated GCD computations.
\end{frame}

\begin{frame}[fragile,label={sec:orgheadline40}]{Algorithm}
 \begin{algorithmic}
  \IF{$n=1$}
  \STATE $g_0 \gets f_0^{-1}$
  \ELSE
  \STATE $F(X^2)\gets f(X)f(-X) \bmod \cycf$
  \STATE $\tilde F(Y) = F(Y)$ \alert{truncated} to {\tt prec} bits of precision
  \STATE $G(Y)\gets \textnormal{InverseMod}(\tilde F(Y),q,n/2)$
  \STATE Set $F_e(X^2), F_o(X^2)$ such that $f(-X) = F_e(X^2)+X F_o(X^2)$
  \STATE $T_e(Y),\ T_o(Y) \gets G(Y)· F_e(Y),\ G(Y)· F_o(Y)$
  \STATE $f^{-1}(X)\gets T_e(X^2) + X T_o(X^2)$
  \STATE $\tilde f^{-1}(X) = f^{-1}(X)$ \alert{truncated} to {\tt prec} bits of precision
  \RETURN $\tilde f^{-1}(X)$
  \ENDIF
\end{algorithmic}

Approximate inverse of \(f(X) \mod \cycf\) using \texttt{prec} bits of precision
\end{frame}

\begin{frame}[label={sec:orgheadline41}]{Timings}
\begin{center}
\begin{tabular}{rrrrrr}
\hline
\(n\) & \(\log σ\) & xgcd & 160 & 160iter & \(\infty\)\\
\hline
4096 & 17.2 & 234.1s & 0.067s & 0.073s & 121.8s\\
8192 & 18.3 & 1476.8s & 0.195s & 0.200s & 755.8s\\
\hline
\end{tabular}

\end{center}

Inverting \(g \sample D_{\Z^n,σ}\) with FLINT's extended Euclidean algorithm ("xgcd"), our implementation with precision 160 ("160"), iterating our implementation until \(\Vert \tilde f^{-1}(X) · f(X) - 1\Vert < 2^{-160}\) ("160iter") and our implementation without truncation ("\(\infty\)") on Intel Core i7--4850HQ CPU at 2.30GHz, single core.
\end{frame}


\section{Small Remainders}
\label{sec:orgheadline46}

\begin{frame}[label={sec:orgheadline43}]{Small Remainders: Motivation}
\begin{itemize}
\item The Jigsaw Generator \footfullcite{FOCS:GGHRSW13} takes as input elements \(a_i\) in \(\Z_p\) where \(p = \norm{{\mathcal{I}}}\) and produces encodings with respect to some \(S_i\).
\item This algorithm produces some small representative of the coset \(a_i\) modulo \(\ideal{g}\) from large integers of size \(\approx {(σ\sqrt{n})}^n\).
\end{itemize}
\end{frame}

\begin{frame}[label={sec:orgheadline44}]{Small Remainders: HD}
\begin{itemize}
\item We can use Babai's trick and that \(g\) is small, i.e. compute
\[a_i - g ⋅ \lfloor g^{-1} ⋅ a_i \rceil \textnormal{ in } \Q[X]/(\cycf)\]
\item To produce sufficiently small elements, we need \(g^{-1}\) either exactly or with high precision.
\item Computing such a high quality approximation of \(g^{-1}\) is prohibitively expensive.
\end{itemize}
\end{frame}

\begin{frame}[label={sec:orgheadline45}]{Small Remainders: SD}
\begin{enumerate}
\item Rewrite \(a_i\) as \[a_i = \sum_{j=0}^{\lceil \log_2(a_i)/B\rceil} 2^{B⋅j}\cdot a_{ij}\] where \(a_{ij} < 2^B\) for some \(B\).
\item Compute small representatives for all \(2^{B⋅j}\) and \(a_{ij}\) using an approximation of \(g^{-1}\) with precision \(B\).
\item Multiply small representatives for \(2^{B⋅j}\) and \(a_{ij}\) and add up their products.
\end{enumerate}

This produces a somewhat short element which we then reduce using approximation of \(g^{-1}\) with precision \(B\) until its size does not decrease any more.
\end{frame}


\section{Discrete Gaussians}
\label{sec:orgheadline53}

\begin{frame}[label={sec:orgheadline47}]{Discrete Gaussian Sampling}
\begin{itemize}
\item We need to sample from the discrete Gaussian \(D_{(g),σ',c}\) where \(c\) is a small representative of a coset of \((g)\).

\item Fundamental building block is sampler over the Integers.
\end{itemize}
\end{frame}

\begin{frame}[fragile,label={sec:orgheadline48}]{\url{https://bitbucket.org/malb/dgs}}
 \begin{itemize}
\item Discrete Gaussian sampler over the integers for arbitrary precision using \texttt{MPFR} and \texttt{double} precision.
\item Implements rejection sampling from a uniform distribution with and without table (“online”) lookups \footfullcite{STOC:GenPeiVai08} and Ducas et al's sampler which samples from \(D_{\Z,kσ_2}\) where \(σ_2\) is a constant \footfullcite{C:DDLL13}.
\item Implementation automatically chooses the best algorithm based on \(σ\), \(c\) and \(\tau\) (tail cut).
\end{itemize}
\end{frame}

\begin{frame}[label={sec:orgheadline49}]{Timings}
\begin{center}
\begin{tabular}{l|rr|rr|rr|}
algorithm & \(σ\) & \(c\) & prec & samp./\(s\) & prec & samp./\(s\)\\
\hline
tabulated & 10000 & 1.0 & 53 & 660.000 & 160 & 310.000\\
tabulated & 10000 & 0.5 & 53 & 650.000 & 160 & 260.000\\
online & 10000 & 1.0 & 53 & 414.000 & 160 & 9.000\\
online & 10000 & 0.5 & 53 & 414.000 & 160 & 9.000\\
Alg 12 \cite{C:DDLL13} & 10000 & 1.0 & 53 & 350.000 & 160 & 123.000\\
\end{tabular}

\end{center}

Example timings for discrete Gaussian sampling over \(\Z\) on Intel Core i7--4850HQ CPU at 2.30GHz, single core.
\end{frame}

\begin{frame}[label={sec:orgheadline50}]{Sampling from \(D_{(g),σ',0}\)}
\begin{itemize}
\item Implemented naively this takes \(\bigO{n^3 \log n}\) operations even if we ignore issues of precision.
\item Following Léo’s thesis \footfullcite{PhD:Ducas13}, we implemented a variant of Peikert’s sampler \footfullcite{C:Peikert10}.
\end{itemize}
\end{frame}

\begin{frame}[label={sec:orgheadline51}]{Sampling from \(D_{(g),σ',0}\)}
\begin{enumerate}
\item Observe that \[D_{(g),σ',0} = g ⋅ D_{R,σ'· g^{-T}}\]

\item Compute approximate square-root \(\sqrt[appr]{\varSigma_2}\) of \[\varSigma_2 = σ'^2 ⋅ g^{-T} ⋅ g^{-1} - r^2 \textnormal{ with } r=2⋅ \lceil \sqrt{\log n}\, \rceil\]

\item Sample a vector \(x \sample {\mathbb{R}}^n\) from a standard normal distribution and interpret it as a polynomial in \(\Q[X]/(\cycf)\).

\item Compute \(y = \sqrt[appr]{\varSigma_2} \cdot x\) in \(\Q[X]/(\cycf)\) and return \(g ⋅ (\lfloor y \rceil_r)\), where \(\lfloor y \rceil_r\) denotes sampling a vector in \(\Z^n\) where the \(i\)-th component follows \(D_{\Z,r,y_i}\).
\end{enumerate}
\end{frame}

\begin{frame}[label={sec:orgheadline52}]{Sampling from \(D_{(g),σ',0}\): Sqrt}
\begin{enumerate}
\item Compute an approximate square root of \[\varSigma_2' = g^{-T} \cdot g^{-1}\] up to \(λ\) bits of precision.
\begin{itemize}
\item Precision: \(\log(n) + 4\,(\log (\sqrt{n}‖σ‖))\) bits.
\item If square root does not converge, double precision and start over.
\end{itemize}

\item Use this approximate square-root, scaled appropriately, as the initial value from which to start computing a square-root of \[\varSigma_2 = \alert{σ'^2} ⋅ g^{-T} ⋅ g^{-1} \alert{- r^2} \textnormal{ with } r=2⋅ \lceil \sqrt{\log n}\, \rceil\]

\item Terminate when the square is within distance \(2^{-2λ}\) to \(\varSigma_2\).

\item Converges quickly because initial candidate close to target.
\end{enumerate}
\end{frame}


\section{Approximate Square Roots}
\label{sec:orgheadline60}

\begin{frame}[label={sec:orgheadline54}]{Strategy}
\begin{itemize}
\item For some input element \(\varSigma\) we want to compute some element \(\sqrt[appr]{\varSigma} \in \Q[X]/(\cycf)\) such that \(\Vert \sqrt[appr]{\varSigma}⋅\sqrt[appr]{\varSigma} - \varSigma \Vert < 2^{-2λ}\).
\item We use iterative methods which iteratively refine the approximation of the square root similar to Newton's method.\footfullcite{PhD:Ducas13}
\item Computing approximate square roots of matrices is a well studied research area with many algorithms known in the literature.\footfullcite{Higham97}
\item All algorithms with global convergence invoke approximate inversions in \(\Q[X]/(\cycf)\) for which we call our inversion algorithm.
\end{itemize}
\end{frame}

\begin{frame}[label={sec:orgheadline55}]{Iterated Methods}
\begin{description}
\item[{Babylonian}] only one inversion, which allows lower precision.
\item[{Denman-Beavers}] converges faster in practice and can be parallelised on two cores.\footfullcite{DenBea76}
\item[{Padé iteration}] arbitrarily many cores, but workload on each core is greater than Denman-Beavers.\footfullcite{Higham97} Only better for us when more than 8 cores were used.
\end{description}
\end{frame}

\begin{frame}[label={sec:orgheadline56}]{Rapid Convergence}
\begin{itemize}
\item Quadratic convergence does not assure rapid convergence in practice because error can take many iterations to become small enough.
\item Speed-up convergence by scaling the operands appropriately in each loop.\footfullcite{Higham97}
\item Common scaling scheme: scale by the determinant, i.e. \(\res(f,\cycf)\) for some \(f \in \Q[X]/(\cycf)\).
\item Computing resultants in \(\Q[X]/(\cycf)\) reduces to computing resultants in \(\Z[X](\cycf)\).
\item Computing resultants in \(\Z[X]/(\cycf)\) can be expensive.
\end{itemize}
\end{frame}

\begin{frame}[label={sec:orgheadline57}]{Approximate Resultants}
\begin{itemize}
\item We are only interested in approximate determinant for scaling → compute with reduced precision.
\item Clear all but the most significant bit for each coefficient's numerator and denominator of \(f\) to produce \(f'\) and compute \(\res(f',\cycf)\).
\item Reduces the size of the integer representation to speed up the resultant computation.
\item With this optimisation scaling by an approximation of the determinant is both fast and precise enough to produce fast convergence.
\end{itemize}
\end{frame}

\begin{frame}[label={sec:orgheadline58}]{Sqrt Timing}
\begin{center}
\begin{tabular}{rrrrrr}
prec & \(n\) & \(\log σ'\) & it. & wall time & \(\log\left( {(\sqrt[appr]{Σ_2})}^2 - Σ_2\right)\)\\
\hline
160 & 1024 & 45.8 & 9 & 0.4s & -200\\
160 & 2048 & 49.6 & 9 & 0.9s & -221\\
160 & 4096 & 53.3 & 10 & 2.5s & -239\\
160 & 8192 & 57.0 & 10 & 8.6s & -253\\
160 & 16384 & 60.7 & 10 & 35.4s & -270\\
\end{tabular}

\end{center}

Approximate square roots of \(\varSigma_2 = σ'^2 \cdot g^{-T} \cdot g - r^2\) on Intel Core i7--4850HQ CPU at 2.30GHz, 2 cores for Denman-Beavers, 4 cores for estimating the scaling factor, one core for sampling.
\end{frame}


\begin{frame}[label={sec:orgheadline59}]{Fin}
\begin{center}
\begin{Huge}
\alert{Thank You}
\end{Huge}
\end{center}

\alert{Code} \url{https://bitbucket.org/malb/gghlite-flint}

\alert{Paper} \url{http://ia.cr/2014/928}
\end{frame}
\end{document}